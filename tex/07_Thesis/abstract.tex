At the end of the second decade of the XXI century, the transportation industry at large faces several challenges that will shape its evolution in the next decade and beyond. The first main challenge is the increasing public awareness and governmental action on climate change, which are increasing the pressure on the industrial sectors responsible for the greatest share of emissions, the transportation industry being one of them, to reduce their environmental footprint. A second challenge lies instead in the renewed push toward price reduction, due to increased competition (as for example the entry of private entities in the market for low-Earth orbit launchers) and innovative business models (like ride-sharing and ride-hailing in the automotive sector or low-cost carriers in civil aviation).\\
A common technical solution strategy to these challenges is the reduction of vehicles' structural mass, while keeping the payload mass constant. By reducing consumption, a reduced weight leads to reduced emissions in fossil-fuels powered vehicles and to increased autonomy in electrical vehicles. By reducing the quantity of materials required in structures, a weight reduction strategy favors a reduction of production costs and thus lower prices. Transportation is however a sector where safety is a paramount concern, and structures must satisfy strict requirements and validation procedures to guarantee their integrity and reliability during service life. This represents a significant constraint which limits the scope of weight reduction strategies.\\
In the last twenty years, the development of a novel type of Fiber-Reinforced Polymer Composite (FRPC) laminates, i.e. \emph{thin-ply} laminates, proposes a solution to these competing requirements (weight with to respect to structural integrity) by providing at the same time weight reduction and increased strength. Several experimental investigations have shown, in fact, that \emph{thin-ply} laminates are capable of delaying, and even suppress, the onset of transverse cracking. Transverse cracks are a kind of sub-critical damage and occur early in the failure process, causing the degradation of elastic properties and favoring other, often more critical, modes of damage (delaminations, fiber breaks). Delay and suppression of transverse cracks were already linked, at the of the 1970's, to the use of thinner plies. However, \emph{thin-plies} available today on the market are at least 10 times thinner than those studied in the 1970's. This changes the length scale of the problem, from millimeters to micrometers. At the microscale, transverse cracks are formed by several fiber/matrix interface cracks (or debonds) coalescing together. Understanding the mechanisms of transverse cracking delay and suppression in \emph{thin-ply} laminates requires detailed knowledge regarding onset of transverse cracking at the microscale, and thus the study of the mechanisms that favor or prevent debond initiation and growth.\\
The main objective of the present work is to investigate the influence of the microstructure on debond growth along the fiber arc direction. To this end, models of 2-dimensional Representative Volume Elements (RVEs) of UD composites and cross-ply laminates are developed. The Representative Volume Elements are characterized by different configurations of fibers and different damage states. Debond initiation is studied through the analysis of the distribution of stresses at the fiber/matrix interface in the absence of damage. Debond growth is characterized using the approach of Linear Elastic Fracture Mechanics (LEFM), specifically through the evaluation of the Mode I, Mode II and total Energy Release Rate (ERR). Displacement and stress fields are evaluated by means of the Finite Element Method (FEM) using the commercial solver Abaqus. The components of the Energy Release Rate are then evaluated using the Virtual Crack Closure Technique (VCCT), implemented in a custom Python routine. The elastic solution of the debonding problem presents two different regimes: the \emph{open crack} and the \emph{closed crack} regimes. In the latter, debond faces are in contact in a region of finite size at the debond tip; in the latter, the debond is everywhere open and no contact exists between the faces. In the \emph{open crack} regime, it is known that stress and displacement fields at the debond tip present an oscillating singularity. A convergence analysis of the VCCT in the context of the FEM solution is thus required to guarantee the validity of results and represents the first step of the work presented in this thesis. It is found that the total ERR does not depend on the size of elements at the debond tip, while the values of Mode I and Mode II ERR depend on element size in the \emph{open crack} or \emph{mixed mode} case. It is furthermore shown that Mode I and Mode II ERR do not converge, i.e. their asymptotic behavior for decreasing element size is not bounded. Thus, error reduction between successive iterations cannot be used to validate the solution and comparison with another method is required. Results obtained with the Boundary Element Method (BEM) and available in the literature are selected to this end. 
