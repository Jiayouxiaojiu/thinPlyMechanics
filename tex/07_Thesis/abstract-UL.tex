The main objective of the present work is to investigate the influence of the microstructure on debond growth along the fiber arc direction. To this end, models of 2-dimensional Representative Volume Elements (RVEs) of Uni-Directional (UD) composites and cross-ply laminates are developed. The Representative Volume Elements are characterized by different configurations of fibers and different damage states. Debond initiation is studied through the analysis of the distribution of stresses at the fiber/matrix interface in the absence of damage. Debond growth on the other hand is characterized using the approach of Linear Elastic Fracture Mechanics (LEFM), specifically through the evaluation of the Mode I, Mode II and total Energy Release Rate (ERR). Displacement and stress fields are evaluated by means of the Finite Element Method (FEM) using the commercial solver Abaqus. The components of the Energy Release Rate are then evaluated using the Virtual Crack Closure Technique (VCCT), implemented in a custom Python routine.\\
The elastic solution of the debonding problem presents two different regimes: the \emph{open crack} and the \emph{closed crack} behaviour. In the latter, debond faces are in contact in a region of finite size at the debond tip; in the latter, the debond is everywhere open and no contact exists between the faces. In the \emph{open crack} regime, it is known that stress and displacement fields at the debond tip present an oscillating singularity. A convergence analysis of the VCCT in the context of the FEM solution is thus required to guarantee the validity of results and represents the first step of the work presented in this thesis. It is found that the total ERR does not depend on the size of elements at the debond tip, while the values of Mode I and Mode II ERR depend on element size in the \emph{open crack} or \emph{mixed mode} case. It is furthermore shown that Mode I and Mode II ERR do not converge, i.e. their asymptotic behavior for decreasing element size is not bounded. Thus, error reduction between successive iterations cannot be used to validate the solution and comparison with another method is required. Results obtained with the Boundary Element Method (BEM), available in the literature, are selected to this end.\\
Debond growth under remote tensile loading is then studied in Representative Volume Elements of: UD composites of varying thickness, measured in terms of number of rows of fibers, from extremely thin (one fiber row) to thick ones; cross-ply laminates with a central $90^{\circ}$ ply of varying thickness, measured as well in terms of number of rows of fibers, from extremely thin (one fiber row) to thick ones; thick UD composites (modelled as infinite along the through-the-thickness direction). Different damage configurations are also considered, corresponding to different stages of transverse crack onset: non-interacting isolated debonds; interacting debonds distributed along the loading direction; debonds on consecutive fibers along the through-the-thickness direction. Among the most relevant results, it is found that neither the $90^{\circ}$ ply thickness nor the $0^{\circ}$ ply thickness influences debond ERR in cross-ply laminates, differently from what is observed for transverse cracks with the so-called ply-thickness and ply-block effects. On the other hand, debond interaction along the loading direction is shown to influence significantly the Energy Release Rate, but this interaction possesses a characteristic distance (in terms of number of undamaged fibers) that defines the region of influence between debonds.\\
Finally, an estimation of debond size at initiation and of debond maximum size is proposed based on arguments from stress analysis (for initiation) and on Griffith's criterion from LEFM (for propagation). For a debond in a cross-ply laminate, its maximum size is estimated to lie in the range $40^{\circ}-60^{\circ}$, which is in strong agreement with previous results from microscopic observations available in the literature.
