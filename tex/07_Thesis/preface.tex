I bought my first and current car, \emph{La Melanza}, in August 2015, just a few weeks before starting my doctoral studies at Lule\aa\ University of Technology and Universit\'e de Lorraine. Today, October 2019, \emph{La Melanza} has traveled $127'712$ kilometers. It has been, indeed, a long journey. One that has brought me to live in two different countries, France and Sweden, and to visit five more, Germany, Greece, Russia, Italy and Spain, for conferences, summer schools and exchanges. A journey in which I have learned a lot, made new friends and built a family. And, apparently, even managed to write a Ph.D. thesis! No such journey could be ventured alone, and here I would like to thank everyone who helped and supported me in these years.\\
It is common use to place supervisors at the top of the acknowledgements list, and I will not be any different. It is however not in adherence to custom, but with sincere gratitude that I place them here in the first place. Thus, many thanks to Prof. Janis Varna for accepting me as his Ph.D. student, sharing his knowledge, correcting my mistakes, pointing my efforts in the right direction, always being curious and passionate about research. Thanks to Prof. Zoubir Ayadi, for welcoming me in France and supporting me all along.\\
I then wish to thank all the members of the Polymeric Composite Group at LTU for welcoming me in Lule\aa, for showing me how to survive at $-30^{\circ}$, for the interesting discussions over a coffee and for their help to solve the problems in the lab: Johanna, Roberts, Patrik, Lennart, Zainab, Nawres, Hiba, Liva, Andrejs, Stephanie, Linqi.\\
I wish to thank also all the people that have helped me extricate myself in all the administrative needs that an international project requires, and have always answered with patience and a smile to my (at times many) questions: Birgitta, Fredrik, Marie-Louise, Christine, Martine, Nadine and Flavio.\\
And finally, my thoughts go to my family. To Scarlett, for ``the purest love in the world is between a grumpy dad and the pet he said he never wanted", and I guess I'm just another proof of it. To Levante, for forcing me to work in order to stay awake late at night guarding him, and for bringing already so much joy in my life. To Valentina, for following me in two different countries, for bringing so many beautiful things in my life and, every now and then, reminding me that there are worse things in life than a deadline for a paper (or a thesis!).\\

\noindent Lule\aa, October 2019\\
Luca Di Stasio
