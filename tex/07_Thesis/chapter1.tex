%%%%%%%%%%%%%%%%%%%%%%%%%%%%%%%%%%%%%%%%%%%%%%%%%%%%%%%%%%%%%%%%%%%%%%%
%      SEC. 1
%%%%%%%%%%%%%%%%%%%%%%%%%%%%%%%%%%%%%%%%%%%%%%%%%%%%%%%%%%%%%%%%%%%%%%%
\section{Vision 2030: challenges of the next decade and beyond for the transportation industry}

Passion and curiosity should always lie at the heart of the scientific activity, and that ought to be enough to define the value of a research effort. Time is the real arbiter of the significance of a piece of research, as many examples in the history of science show~\cite{Brush1967}\footnote{The Ising-Lenz model is one such example~\cite{Brush1967}. It was suggested by physicist Wilhelm Lenz to his doctoral student Ernst Ising to study phase transitions in ferromagnetic materials. Ising solved it analytically in 1D as part of his Ph.D. defense in 1925, but the solution for a 1D lattice did not show any phase transition and was thus regarded as a failure. Almost 20 years later, Onsager solved the 2D version of the model and showed the possibility of phase transitions in the Ising-Lenz model. The Ising-Lenz is now widly reknown in the statistical physics community and has been applied in several different fields.}.However, in these years of increasing mistrust towards scientific research and growing doubts on the value of universities and research institutes, it is worth to reflect on the place of one's own work. 


