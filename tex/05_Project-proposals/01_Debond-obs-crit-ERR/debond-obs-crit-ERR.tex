\documentclass[review]{elsarticle}

\usepackage{amsmath}
\usepackage{enumitem}
\usepackage{lineno,hyperref}
\modulolinenumbers[5]

\journal{Project proposal}

%%%%%%%%%%%%%%%%%%%%%%%
%% Elsevier bibliography styles
%%%%%%%%%%%%%%%%%%%%%%%
%% To change the style, put a % in front of the second line of the current style and
%% remove the % from the second line of the style you would like to use.
%%%%%%%%%%%%%%%%%%%%%%%

%% Numbered
%\bibliographystyle{model1-num-names}

%% Numbered without titles
%\bibliographystyle{model1a-num-names}

%% Harvard
%\bibliographystyle{model2-names.bst}\biboptions{authoryear}

%% Vancouver numbered
%\usepackage{numcompress}\bibliographystyle{model3-num-names}

%% Vancouver name/year
%\usepackage{numcompress}\bibliographystyle{model4-names}\biboptions{authoryear}

%% APA style
%\bibliographystyle{model5-names}\biboptions{authoryear}

%% AMA style
%\usepackage{numcompress}\bibliographystyle{model6-num-names}

%% `Elsevier LaTeX' style
\bibliographystyle{elsarticle-num}
%%%%%%%%%%%%%%%%%%%%%%%

\begin{document}

\begin{frontmatter}

\title{{\bf - Project proposal -}\\Microscopic observation and statistical analysis of initiation and propagation of the fiber/matrix interface crack}
%\tnotetext[mytitlenote]{Fully documented templates are available in the elsarticle package on \href{http://www.ctan.org/tex-archive/macros/latex/contrib/elsarticle}{CTAN}.}

%% Group authors per affiliation:
%\author{Luca Di Stasio\fnref{myfootnote}}
%\address{Radarweg 29, Amsterdam}
%\fntext[myfootnote]{Since 1880.}

%% or include affiliations in footnotes:
\author{Luca Di Stasio}
\author{Janis Varna}
%\ead[url]{www.elsevier.com}

%\author[mysecondaryaddress]{Global Customer Service\corref{mycorrespondingauthor}}
%\cortext[mycorrespondingauthor]{Corresponding author}
%\ead{support@elsevier.com}

%\address[nancy]{Universit\'e de Lorraine, EEIGM, IJL, 6 Rue Bastien Lepage, F-54010 Nancy, France}
%\address[lulea]{Lule\aa\ University of Technology, University Campus, SE-97187 Lule\aa, Sweden}

%\begin{abstract}

%\end{abstract}

%\begin{keyword}
%\texttt{elsarticle.cls}\sep \LaTeX\sep Elsevier \sep template
%\MSC[2010] 00-01\sep  99-00
%\end{keyword}

\end{frontmatter}

\linenumbers

\section{Introduction}

Although microscopic observations have been performed since , only few works (\cite{Correa2018,Zumaquero2018}) have attempted to statistically characterize the size of fiber/matrix interface crack in FRP UD and cross-ply laminates. Current methods allow for feasible and reliable measurements only in 2D, by observing the specimen on each side and eventually the two sides of a central longitudinal cross-section obtained after cutting and polishing. Coupled with numerical and analytical estimations of ERRs at the crack tips, a statistical analysis of debonds' geometry for different combinations of material configurations and loading could provide valuable insights into the mechanisms underpinning the onset of transverse cracks in FRP laminates.

\section{Objectives}
\begin{enumerate}
\item Determine the statistical distribution and statistical descriptors (mean, mode, median, variance) of
\begin{enumerate}[label=\alph*]
\item debond size,
\item angular position of debond's crack tips,
\item angular position of debond's mid-point,
\item angular position of kinks' start,
\item kinking angles,
\end{enumerate}
parameterized with respect to
\begin{enumerate}[label=\alph*]
\item fibers' material,
\item laminate lay-up,
\item level of applied strain.
\end{enumerate}

\item Investigate correlations between the quantities defined in 1 and the distribution of
\begin{enumerate}[label=\alph*]
\item fibers' radii,
\item angular position of closest fiber to debonded one,
\item distance of closest fiber to debonded one,
\item material,
\item lay-up,
\item level of applied strain.
\end{enumerate}

\item Measure the reduction in stiffness.
\item Measure the linear density of transverse cracks.
\item Measure the areal density of debonds.

\end{enumerate}

\section{Materials}
Glass-fiber and carbon-fiber cross-ply $[0_{m\cdot n}^{\circ},90_{n}^{\circ}]$ with $m=1,10$. 6 specimens for each lay-up and material combination, for a total of 24.

\section{Methods}
\begin{enumerate}
\item Manufacturing of laminates through manual lay-up, cutting and polishing of specimens.
\item Tensile tests in quasi-static conditions at $2\ \left[mm/min\right]$ reaching different levels of applied strain: $\left[0.4\%, 0.6\%, 0.8\%, 1.0\%\right]$.
\item Once a target level of strain is reached:
\begin{enumerate}[label=\alph*]
\item unload the specimen and then load again at $0.3\%$ to evaluate the reduction in stiffness;
\item count the transverse cracks visually and then with the optical microscope, with which measure the distance between cracks;
\item for each debond visible with the aid of the optical microscope, measure its fiber's radius, its angular size, crack tips' position, kinks' starting position, angular position and distance of the closest fiber.
\end{enumerate}
\item Analyse the data in Python, R, Excel or Matlab.
\end{enumerate}

\section{Expected outcomes}

\section{Audience}
2-3 students for Project Course or Master thesis.


\bibliography{mybibfile}


\end{document}
