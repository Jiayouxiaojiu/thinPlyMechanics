\documentclass[review]{elsarticle}

\usepackage{amsmath}
\usepackage{enumitem}
\usepackage{lineno,hyperref}
\modulolinenumbers[5]

\journal{Project proposal}

%%%%%%%%%%%%%%%%%%%%%%%
%% Elsevier bibliography styles
%%%%%%%%%%%%%%%%%%%%%%%
%% To change the style, put a % in front of the second line of the current style and
%% remove the % from the second line of the style you would like to use.
%%%%%%%%%%%%%%%%%%%%%%%

%% Numbered
%\bibliographystyle{model1-num-names}

%% Numbered without titles
%\bibliographystyle{model1a-num-names}

%% Harvard
%\bibliographystyle{model2-names.bst}\biboptions{authoryear}

%% Vancouver numbered
%\usepackage{numcompress}\bibliographystyle{model3-num-names}

%% Vancouver name/year
%\usepackage{numcompress}\bibliographystyle{model4-names}\biboptions{authoryear}

%% APA style
%\bibliographystyle{model5-names}\biboptions{authoryear}

%% AMA style
%\usepackage{numcompress}\bibliographystyle{model6-num-names}

%% `Elsevier LaTeX' style
\bibliographystyle{elsarticle-num}
%%%%%%%%%%%%%%%%%%%%%%%

\begin{document}

\begin{frontmatter}

\title{{\bf - Project proposal -}\\Microscopic observation and statistical analysis of initiation and propagation of the fiber/matrix interface crack}
%\tnotetext[mytitlenote]{Fully documented templates are available in the elsarticle package on \href{http://www.ctan.org/tex-archive/macros/latex/contrib/elsarticle}{CTAN}.}

%% Group authors per affiliation:
%\author{Luca Di Stasio\fnref{myfootnote}}
%\address{Radarweg 29, Amsterdam}
%\fntext[myfootnote]{Since 1880.}

%% or include affiliations in footnotes:
\author{Luca Di Stasio}
\author{Janis Varna}
%\ead[url]{www.elsevier.com}

%\author[mysecondaryaddress]{Global Customer Service\corref{mycorrespondingauthor}}
%\cortext[mycorrespondingauthor]{Corresponding author}
%\ead{support@elsevier.com}

%\address[nancy]{Universit\'e de Lorraine, EEIGM, IJL, 6 Rue Bastien Lepage, F-54010 Nancy, France}
%\address[lulea]{Lule\aa\ University of Technology, University Campus, SE-97187 Lule\aa, Sweden}

%\begin{abstract}

%\end{abstract}

%\begin{keyword}
%\texttt{elsarticle.cls}\sep \LaTeX\sep Elsevier \sep template
%\MSC[2010] 00-01\sep  99-00
%\end{keyword}

\end{frontmatter}

\linenumbers

\section{Introduction}

Only few works (\cite{Correa2018,Zumaquero2018}) have attempted to quantify the size of debonding

\section{Objectives}
\begin{enumerate}
\item Determine the statistical distribution and statistical descriptors (mean, mode, median, variance) of
\begin{enumerate}[label=\alph*]
\item debond size,
\item angular position of debond's crack tips,
\item angular position of debond's mid-point,
\item angular position of kinks' start,
\item kinking angles,
\end{enumerate}
parameterized with respect to
\begin{enumerate}[label=\alph*]
\item fibers' material,
\item laminate lay-up,
\item level of applied strain.
\end{enumerate}

\item Investigate correlations between the quantities defined in 1 and the distribution of
\begin{enumerate}[label=\alph*]
\item fibers' radii
\item angular position of closest fiber to debonded one
\item distance of closest fiber to debonded one
\item material
\item lay-up
\item level of applied strain
\end{enumerate}
\end{enumerate}

\section{Materials}
Glass-fiber and carbon-fiber cross-ply $[0_{m\cdot n}^{\circ},90_{n}^{\circ}]$ with $m=1,10$. 6 specimens for each lay-up and material combination, for a total of 24.
\section{Methods}

\section{Expected outcomes}

\section{Audience}
1-2 students for Project Course or Master thesis.


\bibliography{mybibfile}


\end{document}
