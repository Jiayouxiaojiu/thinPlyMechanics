\documentclass[12pt,a4paper]{article}
\usepackage{composites2019}

\begin{document}
\thispagestyle{empty}

\vspace*{-3.4cm}
\begin{table}[!h]
\begin{tabular}{r}
\hspace*{5.5cm} \scriptsize \textsf{7th ECCOMAS Thematic Conference on the Mechanical Response of Composites} \\
\hspace*{5.5cm} \scriptsize \textsf{ COMPOSITES 2019} \\
\hspace*{5.5cm} \tiny \textsf{A. Turon, P. Maimí \& M. Fagerström (Editors)}
\end{tabular}
\end{table}

\begin{center}
\title{ESTIMATING THE AVERAGE SIZE OF FIBER/MATRIX INTERFACE CRACKS IN UD AND CROSS-PLY LAMINATES}
\end{center}
\begin{center}
\textbf{\underline{Luca Di Stasio}$^{1,2,*}$, Janis Varna$^{1}$, Zoubir Ayadi$^{2}$} \\ [7pt]
\small{$^1$~Lule\aa\ University of Technology, University Campus, SE-97187 Lule\aa, Sweden}  \\  [2pt]  
\small{$^2$~Universit\'e de Lorraine, EEIGM, IJL, 6 Rue Bastien Lepage, F-54010 Nancy, France}  \\  [2pt]
\small{$^*$~\texttt{luca.di.stasio@ltu.se}} \\
\end{center}


\paragraph{Keywords:} Fiber Reinforced Polymer (FRP), Debonding, Linear Elastic Fracture Mechanics (LEFM).

\paragraph{Summary:} \textit{This document provides information and instructions for preparing the (optional) full-length paper for the COMPOSITES 2019 Conference (September 18-20, 2019 in Girona, Spain).}

\section{INTRODUCTION}

The Conference publication will consist of a pen drive containing papers of the contributions received and a printed Book of Abstracts containing a one page version of the accepted abstracts. The authors must submit a full-length paper (max. 12 pages) using the same format of this template. Submission of a full-length paper is not mandatory but authors are strongly encouraged to send it before June 27, 2019.

The deadline date for early registration date is April 30, 2019. Presenting authors must register by June 13, 2019. Papers with authors not registered by this date will be removed from the final program.
Registration closes on September 5, 2019. Further information can be found at the conference website: \texttt{www.composites2019.udg.edu}

\section{RVE MODELS AND FE DISCRETIZATION}

In this contribution, we analyze debond initiation and propagation in Representative Volume Elements (RVEs) of Uni-Directional (UD) composites and $\left[0_{m\cdot k\cdot2L}^{\circ},90_{k\cdot2L}^{\circ},0_{m\cdot k\cdot2L}^{\circ}\right]$ laminates. Given a global reference frame with axis $x$, $y$ and $z$, both types of composites are modeled as plates lying in the $x-y$ plane, with the through-the-thickness direction thus aligned with the $z$ axis. The UD composite $0^{\circ}$ direction is parallel to the $y$ axis, while the cross-ply $0^{\circ}$ direction is parallel to the $x$ axis. Both composites are loaded in tension along the $x$ axis, which thus corresponds to: transverse loading of the UD specimen; axial loading of the cross-ply specimen. In both composites, damage is present only in the form of fiber/matrix interface cracks, or debonds. In cross-plies, debonds are assumed to be present only in the central ${90^{\circ}}$.

\section{STRESS-BASED ANALYSIS OF DEBOND INITIATION}

\section{ENERGY-BASED ANALYSIS OF DEBOND PROPAGATION}

\section{CONCLUSIONS}

We are looking forward to receiving your contributions for this conference.

\section*{ACKNOWLEDGEMENTS}

Luca Di Stasio gratefully acknowledges the support of the European School of Materials (EUSMAT) through the DocMASE Doctoral Programme and the European Commission through the Erasmus Mundus Programme.

\bibliographystyle{unsrt}

%\begin{thebibliography}{10}
%
%\bibitem{Barbero} E.J. Barbero, \textit{Finite Element Analysis of Composite Materials}. CRC Press, Boca Raton, 2008.
%\bibitem{Pimenta} S. Pimenta, S.T. Pinho, The effect of recycling on the mechanical response of carbon fibres and their composites. \textit{Composite Structures}, \textbf{94}, 3669-3684, 2012.
%
%\end{thebibliography}

\bibliography{refs}

\end{document}

