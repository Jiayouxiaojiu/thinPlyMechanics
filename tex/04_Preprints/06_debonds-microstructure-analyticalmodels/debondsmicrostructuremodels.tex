\documentclass[review]{elsarticle}

\usepackage{amsmath}
\usepackage{subcaption}
\usepackage[usenames]{xcolor}
\usepackage{lineno,hyperref}
\modulolinenumbers[5]

\journal{Report}

%%%%%%%%%%%%%%%%%%%%%%%
%% Elsevier bibliography styles
%%%%%%%%%%%%%%%%%%%%%%%
%% To change the style, put a % in front of the second line of the current style and
%% remove the % from the second line of the style you would like to use.
%%%%%%%%%%%%%%%%%%%%%%%

%% Numbered
%\bibliographystyle{model1-num-names}

%% Numbered without titles
%\bibliographystyle{model1a-num-names}

%% Harvard
%\bibliographystyle{model2-names.bst}\biboptions{authoryear}

%% Vancouver numbered
%\usepackage{numcompress}\bibliographystyle{model3-num-names}

%% Vancouver name/year
%\usepackage{numcompress}\bibliographystyle{model4-names}\biboptions{authoryear}

%% APA style
%\bibliographystyle{model5-names}\biboptions{authoryear}

%% AMA style
%\usepackage{numcompress}\bibliographystyle{model6-num-names}

%% `Elsevier LaTeX' style
\bibliographystyle{elsarticle-num}
%%%%%%%%%%%%%%%%%%%%%%%

\begin{document}

\begin{frontmatter}

\title{Analysis of size, curvature and shape effects on the growth of the fiber/matrix interface crack in UD and cross-ply laminates based on Representative Volume Element (RVE) modeling}
%\tnotetext[mytitlenote]{Fully documented templates are available in the elsarticle package on \href{http://www.ctan.org/tex-archive/macros/latex/contrib/elsarticle}{CTAN}.}

%% Group authors per affiliation:
%\author{Luca Di Stasio\fnref{myfootnote}}
%\address{Radarweg 29, Amsterdam}
%\fntext[myfootnote]{Since 1880.}

%% or include affiliations in footnotes:
\author[nancy,lulea]{Luca Di Stasio}
\author[lulea]{Janis Varna}
\author[nancy]{Zoubir Ayadi}
%\ead[url]{www.elsevier.com}

%\author[mysecondaryaddress]{Global Customer Service\corref{mycorrespondingauthor}}
%\cortext[mycorrespondingauthor]{Corresponding author}
%\ead{support@elsevier.com}

\address[nancy]{Universit\'e de Lorraine, EEIGM, IJL, 6 Rue Bastien Lepage, F-54010 Nancy, France}
\address[lulea]{Lule\aa\ University of Technology, University Campus, SE-97187 Lule\aa, Sweden}

\begin{abstract}
\noindent
\textcolor{purple}{{\em Priority}: 2}\\
\textcolor{purple}{{\em Target journal(s)}: Composites Part B: Engineering, Composites Part A: Applied Science and Manufacturing, Composite Science and Technology, Composite Structures, Journal of Composite Materials, Composite Communications}\\
\end{abstract}

%\begin{keyword}
%\texttt{elsarticle.cls}\sep \LaTeX\sep Elsevier \sep template
%\MSC[2010] 00-01\sep  99-00
%\end{keyword}

\end{frontmatter}

\linenumbers

\section{Introduction}

\textcolor{blue}{
\begin{enumerate}
\item By recalling Buckingham's dimensional theorem, we recall that modeling size, shape, cruvature effects means finding analytical expression by which we can calculate ERR or, at least, given a base value we can calculate its change for a change in some reference quantity. Ex: ERR for debonds scales linearly with fiber radius. We recall the usefulness of such expressions: simple to use, quick and cheap calculations, provide insights on mechanics (ex: what happens to ERR if I use a fiber with a radius 2 times larger? ERR will 2 times as the base case).
\item This approach has been applied in the Fracture Mechanics literature in the form of the shape function: SIF (and ERR) can be expressed as $f(\sigma_{\inf},a)\cdot S$, where $f(\sigma_{\inf},a)$ is the solution for the straight crack in an infinite isotropic plate under transverse tension. $S$ is the shape function and represents the effect of different BCs, loading modes, and crack or plate geometry.
\item We then observe that for the fiber/interface crack a reference $G_{0}$ has been used, however we note that: there's no work that investigates the pros and cons of one formulation with respect to the other; there's no agreement on which formulation to use. We review briefly the different choices made since Toya.
\item We thus address this gap in the literature in this paper. We focus on the following questions: does a reference ERR exist with which we can parameterize results? Is there an analytical formulation (based on regression) for ERR for the fiber/matrix interface crack?
\item We conclude by summarizing the structure of the paper.
\end{enumerate}
}

\section{Homogenized models}

\subsection{Straight crack in an infinite homogenized ply under transverse loading}\label{subsec:straightcrack}

\textcolor{blue}{Why do we recall this case?\\Because at first approximation, the debond under remote transverse tension can be modeled as a crack of size $R_{f}\sin{\left(\Delta\theta\right)}$ in a homogenized ply}

\subsection{Semi-circular crack in an infinite homogenized ply under transverse loading}

\textcolor{blue}{Why do we recall this case?\\Because as a second approximation, the debond under remote transverse tension can be modeled as a semi-circular crack in a homogenized ply\\Here we also plot the different components of the solution and reflect on their meachnical meaning}

\section{The analytical solution of the fiber/matrix interface crack problem}

\textcolor{blue}{We recall the solution and then we plot the different components and show that, at first approximation, the solution can be expressed as $A\sin{\left(B\Delta\theta+C\right)+D}$. We observe that $G_{dim}\sim G_{0}\sin{\left(\Delta\theta\right)}$ means that $G_{dim}\sim R_{f}\sin{\left(\Delta\theta\right)}$, which is a size effect and means that the debond in the infinite matrix behaves closely to the case in~\ref{subsec:straightcrack}.}

\section{Modeling size effects}

\textcolor{blue}{We compare mode I and mode II ERR from FEM simulations and compare to the value corresponding to a straight crack in an infinite homogenized ply.}

\section{Modeling curvature effects}

\textcolor{blue}{We compare mode I and mode II ERR from FEM simulations and compare to the value corresponding to a semi-circular crack in an infinite homogenized ply, the part correpsonding only to the curvature.}

\section{Modeling shape effects}

\textcolor{blue}{We compare mode I and mode II ERR from FEM simulations and compare to the value corresponding to the full solution for a semi-circular crack in an infinite homogenized ply.}

\section{An analytical model of the fiber/matrix interface crack}

\textcolor{blue}{Based on previous considerations, we suggest an analytical regression-based expression for the energy release rate of the fiber/matrix interface crack.}

\section{Conclusions \& Outlook}

\end{document}
