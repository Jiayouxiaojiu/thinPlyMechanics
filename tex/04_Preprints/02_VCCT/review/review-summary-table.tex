\begin{longtable}{ccccXX}
\bf{N.}&\bf{Status}&\bf{Reviewer}&\bf{Line(s)/Section(s)}&\bf{Observation(s)}&\bf{Action(s)}\\
&\textcolor{red}{\xmark}\textrightarrow To-Do&&&&\\
&\textcolor{orange}{$\blacksquare$}\textrightarrow In Progress&&&&\\
&\textcolor{green}{\cmark}\textrightarrow Reviewed&&&&\\
\toprule
\bottomrule
1&\textcolor{red}{\xmark}&1&Introduction&The issues related to mesh dependency of the modal contributions to the energy release rate in bi-material interface crack problems are long known in the literature. In this respect, some additional references could be usefully added in the Introduction: 
1) I.S. Raju, J.H. Crews Jr., M.A. Aminpour, Convergence of strain energy release rate components for Edge-Delaminated composite laminates, Engineering Fracture Mechanics 30 (3) (1988) 383-396. doi: 10.1016/0013-7944(88)90196-8.
2) A. Agrawal, A.M. Karlsson, Obtaining mode mixity for a bimaterial interface crack using the virtual crack closure technique, International Journal of Fracture 141 (1-2) (2006) 75-98. doi:10.1007/s10704-006-0069-4.
3)  R. Krueger, K. Shivakumar, I.S. Raju, Fracture mechanics analyses for interface crack problems - A review, in: Proceedings of the 54th AIAA/ASME/ASCE/AHS/ASC Structures, Structural Dynamics, and Materials Conference, 2013. doi:10.2514/6.2013-1476.\\
\midrule
2&\textcolor{green}{\cmark}&1&Section 3& To help understanding, it would be useful to add a figure of the mesh in the neighbourhood of the crack tip showing the delta-sized elements (e.g., in two cases with m = 1 and m = 2) with the indices p and q.&\textit{In Section 3:} added a figure for first order elements and one figure for second order elements showing the $\delta$-sized elements at the crack tip and $p,q$-based indexing of the nodes; added reference in the text.\\
\midrule
3&\textcolor{green}{\cmark}&1&Section 3& If I correctly understand, the permutation matrix is denoted as $P_{\pi}$ in Eq. (7) and as D in Eq. (8). Please, check and unify notation.&\textit{In Section 3:}  Eq. (8) modified to agree with  Eq. (7); the correct notation is indeed $P_{\pi}$.\\
\midrule
4&\textcolor{red}{\xmark}&1&Section 3&The comparison between Eq. (11) and a similar one in the paper by Valvo (2012) is useful, but should be revised:
a - the two relations are different because of an additional term in the submitted manuscript. Indeed, Eq. (11) relates the forces at the crack-tip node with the relative displacements at adjacent nodes (on the same mesh). Instead, Valvo (2012) relates forces and displacements at the same crack-tip node (on two different meshes);
b - the linear relationship between forces and displacements is not "assumed a priori" in the cited paper, but is an immediate consequence of the linear elastic model;
c - also according to Valvo (2012), the crack-tip stiffness matrix is not diagonal in general because of Poisson's effect, as well as material mismatch and geometric asymmetry. The matrix is diagonalised to obtain the mode I and mode II contributions to the energy release rate;
d - lastly, consider also the following development:
P.S. Valvo, A further step towards a physically consistent virtual crack closure technique, International Journal of Fracture 192 (2) (2015) 235-244. doi:10.1007/s10704-015-0007-4.\\
\midrule
5&\textcolor{green}{\cmark}&1&Section 3&In Eq. (12), an operator called Tr appears for the first time, but is not defined. I presume it is the Trace of the matrix, but for clarity it would be better to specify in the manuscript. In Eq. (12), also check the order of the arguments, which maybe have to be rearranged to yield a 2x2 matrix (as in the following equations).&\textit{In Section 3:} added explanation of $Tr$ symbol and \emph{Trace} operator; re-ordered arguments in Eq. (12) to get a 2x2 matrix in agreement with the rest of the discussion.\\
\midrule
6&\textcolor{green}{\cmark}&1&Conclusions& In the Conclusions, the comments about the cited part of the Abaqus User's Manual should be revised. The Manual gives suggestions on how to improve the accuracy of stress and strain calculations in the neighbourhood of the crack tip, as well as of the J-integral (total energy release rate). At least from the cited part, there is no direct reference to VCCT computation of mode I and mode II contributions to energy release rate, which are known to be mesh dependent quantities.&\textit{In Conclusions: }the discussion has been re-formulated to focus on what Abaqus proposes for its VCCT-based crack propagation technique. Notice that this discussion is not meant as a critique \textit{per se} to Abaqus software or documentation, but it is meant to promote reflection in the practitioner to avoid black-box thinking when working in Fracture Mechanics (and numerical techniques in general).\\
\midrule
7&\textcolor{green}{\cmark}&1&Appendix A&In Appendix A, there is an attemp to cover at the same time both two-dimensional (d = 2) and three-dimensional (d = 3) problems. However, to achieve fully this goal, also matrices in Eqs. (A.1), (A.2), ..., (A.11) should have optional additional columns and rows. This would only add up to confusion. For the sake of clarity, it would be better to focus on the two-dimensional problem only (as concerns equations) and simply state that extension to three-dimensional problems is straightforward.&\textit{In Appendix A: }Section A.1 has been revised, by clarifying which expression is the general formulation for an arbitrary number of dimensions and which is instead an exemplification of the structure of the mathematical object in the 2D setting, which is the one of interest for the rest of the paper.\\
\midrule
8&\textcolor{green}{\cmark}&1&Appendix A&Again in Appendix A, correct:
- "$u_{r,M}, u_{r,F}$ the x-displacement" $\rightarrow$ "$u_{r,M}, u_{r,F}$ the r-displacement";
- "$u_{theta,M}, u_{theta,F}$ the y-displacement" $\rightarrow$ "$u_{theta,M}, u_{theta,F}$ the theta-displacement".&\textit{In Appendix A: }corrections made in Section A.2.\\
\midrule
\midrule
9&\textcolor{red}{\xmark}&2&introduction/conclusions&In particular, it seems that the paper fails to establish the proper connection with the state of the art literature on VCCT applied to bimaterial cracks. [...] In conclusion, the reviewer suggests to broaden the focus so to consider the relevant contributions from the existing literature such as those provided by Kruger (2014), and to point out the main novelties of their approach wrt the existing ones. A further aspects is that GI and GII computed via VCCT for bimaterial interface cracks are strongly dependent on the assumed crack extension length. Hence, it would be advisable that the Authors puts into evidence this aspect.  Another possible extension could be to compare the GI/GII calculations with those computed by means of techniques other than the VCCT, perhaps starting from the horizontal crack case so to have reference results. Finally, the authors could suggest a modified VCCT overcoming the drawbacks of existing revised VCCT such as Valvo (2012) or  S. Wang et al. (2013).\\
\end{longtable}