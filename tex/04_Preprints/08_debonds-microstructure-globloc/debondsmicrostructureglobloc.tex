\documentclass[review]{elsarticle}

\usepackage{amsmath}
\usepackage[makeroom]{cancel}
\usepackage{lineno,hyperref}
\modulolinenumbers[5]

\journal{Composites Part A}

%%%%%%%%%%%%%%%%%%%%%%%
%% Elsevier bibliography styles
%%%%%%%%%%%%%%%%%%%%%%%
%% To change the style, put a % in front of the second line of the current style and
%% remove the % from the second line of the style you would like to use.
%%%%%%%%%%%%%%%%%%%%%%%

%% Numbered
%\bibliographystyle{model1-num-names}

%% Numbered without titles
%\bibliographystyle{model1a-num-names}

%% Harvard
%\bibliographystyle{model2-names.bst}\biboptions{authoryear}

%% Vancouver numbered
%\usepackage{numcompress}\bibliographystyle{model3-num-names}

%% Vancouver name/year
%\usepackage{numcompress}\bibliographystyle{model4-names}\biboptions{authoryear}

%% APA style
%\bibliographystyle{model5-names}\biboptions{authoryear}

%% AMA style
%\usepackage{numcompress}\bibliographystyle{model6-num-names}

%% `Elsevier LaTeX' style
\bibliographystyle{elsarticle-num}
%%%%%%%%%%%%%%%%%%%%%%%

\newdefinition{prn}{Principle}

\begin{document}

\begin{frontmatter}

\title{Stiffness reduction in UD and cross-ply laminates due to fiber/matrix interface cracks}
%\tnotetext[mytitlenote]{Fully documented templates are available in the elsarticle package on \href{http://www.ctan.org/tex-archive/macros/latex/contrib/elsarticle}{CTAN}.}

%% Group authors per affiliation:
%\author{Luca Di Stasio\fnref{myfootnote}}
%\address{Radarweg 29, Amsterdam}
%\fntext[myfootnote]{Since 1880.}

%% or include affiliations in footnotes:
\author[lulea]{Luca Di Stasio}
\author[lulea]{Janis Varna}
%\author[nancy]{Zoubir Ayadi}
%\ead[url]{www.elsevier.com}

%\author[mysecondaryaddress]{Global Customer Service\corref{mycorrespondingauthor}}
%\cortext[mycorrespondingauthor]{Corresponding author}
%\ead{support@elsevier.com}

\address[lulea]{Lule\aa\ University of Technology, University Campus, SE-97187 Lule\aa, Sweden}
%\address[nancy]{Universit\'e de Lorraine, EEIGM, IJL, 6 Rue Bastien Lepage, F-54010 Nancy, France}


\begin{abstract}

\end{abstract}

\begin{keyword}
Fiber Reinforced Polymer Composite (FRPC) \sep Debonding \sep Finite element analysis (FEA)
\end{keyword}

\end{frontmatter}

\linenumbers

\section{Introduction}

Main ref~\cite{Varna2018}

\section{Derivation of constitutive relations}

\subsection{Reference frames}

\begin{description}
\item[Local reference frame of $k$-th layer: ] index $1$ is the in-plane longitudinal or fiber or $0^{\circ}$-direction; index $2$ is the in-plane transverse or $90^{\circ}$-direction; index $3$ is the out-of-plane or through-the-thickness direction.
\item[Global reference frame of laminate: ] index $x$ is the in-plane longitudinal or laminate $0^{\circ}$direction; index $y$ is the in-plane transverse direction; index $z$ is the out-of-plane or through-the-thickness direction.
\end{description}

\subsection{Crack density}

\begin{prn}
Normalized volume of cracks $V_{an}$ is the ratio of cracked volume $V_{a}$ to material volume $V$

\begin{equation}
V_{an}=\frac{V_{a}}{V}
\end{equation}

$V_{a}$ is equal to the product of total crack surface $S_{C}$ and average crack opening $u_{a}$

\begin{equation}
V_{an}=\frac{S_{C}u_{a}}{V}=\frac{S_{C}}{V}u_{a}=\rho_{C}u_{a}
\end{equation}

The ratio $\frac{S_{C}}{V}$ has a size of $\frac{1}{length}$ and correspond to the crack density $\rho_{C}$. It means: product of crack density and average crack opening is equal to normalized volume of cracks.
\end{prn}

Applying the previous Principle to debonds, we have:

\begin{equation}
\begin{aligned}
\rho_{D}=\frac{\text{total area of debonds}}{\text{total layer volume}}=&\frac{n_{D}wR_{f}\Delta\theta}{L_{lam}wt_{90^{\circ}}}=\frac{n_{D}\cancel{w}}{L_{lam}\cancel{w}t_{90^{\circ}}}R_{f}\Delta\theta=\frac{1}{n2L}\frac{1}{k2L}R_{f}\Delta\theta=\\=&\frac{1}{nk4L^{2}}R_{f}\Delta\theta=\frac{V_{f}}{nk\pi R_{f}^{2}}R_{f}\Delta\theta=\frac{V_{f}}{nkR_{f}}\frac{\Delta\theta}{\pi}
\end{aligned}
\end{equation}

\subsection{Vakulenko-Kachanov tensor}

In the local reference frame of $k$-th layer, the outer normal at crack faces has components:

\begin{equation}
n_{1}=0\quad n_{2}\neq0\quad n_{3}\neq0
\end{equation}

while crack face displacement has components:

\begin{equation}
u_{1}=0\quad u_{2}\neq0\quad u_{3}\neq0
\end{equation}

Definition of Vakulenko-Kachanov tensor:

\begin{equation}
\beta_{ij}=\frac{1}{V_{k}}\int_{S_{C}}\frac{1}{2}\left(u_{i}n_{j}+u_{j}n_{i}\right)dS
\end{equation}

Expand the expression for each component and simplify based on the fact that $u_{1}=0$:

\begin{equation}
\begin{aligned}
\beta_{11}=&\frac{1}{V_{k}}\int_{S_{C}}\frac{1}{2}\left(u_{1}n_{1}+u_{1}n_{1}\right)dS=\frac{1}{V_{k}}\int_{S_{C}}\cancelto{0}{u_{1}}n_{1}dS=0\\
\beta_{22}=&\frac{1}{V_{k}}\int_{S_{C}}\frac{1}{2}\left(u_{2}n_{2}+u_{2}n_{2}\right)dS=\frac{1}{V_{k}}\int_{S_{C}}u_{2}n_{2}dS\\
\beta_{33}=&\frac{1}{V_{k}}\int_{S_{C}}\frac{1}{2}\left(u_{3}n_{3}+u_{3}n_{3}\right)dS=\frac{1}{V_{k}}\int_{S_{C}}u_{3}n_{3}dS\\
\beta_{12}=&\frac{1}{V_{k}}\int_{S_{C}}\frac{1}{2}\left(\cancelto{0}{u_{1}}n_{2}+u_{2}\cancelto{0}{n_{1}}\right)dS=0\\
\beta_{13}=&\frac{1}{V_{k}}\int_{S_{C}}\frac{1}{2}\left(\cancelto{0}{u_{1}}n_{3}+u_{3}\cancelto{0}{n_{1}}\right)dS=0\\
\beta_{23}=&\frac{1}{V_{k}}\int_{S_{C}}\frac{1}{2}\left(u_{2}n_{3}+u_{3}n_{2}\right)dS\\
\beta_{21}=&\frac{1}{V_{k}}\int_{S_{C}}\frac{1}{2}\left(u_{2}\cancelto{0}{n_{1}}+\cancelto{0}{u_{1}}n_{2}\right)dS=0\\
\beta_{31}=&\frac{1}{V_{k}}\int_{S_{C}}\frac{1}{2}\left(u_{3}\cancelto{0}{n_{1}}+\cancelto{0}{u_{1}}n_{3}\right)dS=0\\
\beta_{32}=&\frac{1}{V_{k}}\int_{S_{C}}\frac{1}{2}\left(u_{3}n_{2}+u_{2}n_{3}\right)dS=\beta_{23}\\
\end{aligned}
\end{equation}

Only 3 independent components of the tensor $\beta_{ij}$ remain: $\beta_{22}$, $\beta_{33}$ and $\beta_{23}$.\\
Split total crack surface $S_{C}$ into total matrix crack surface $S_{C}^{m}$ and total fiber crack surface $S_{C}^{f}$ and remember that $n_{i}^{f}=-n_{i}^{m}$ for $i=2,3$

\begin{equation}\label{eq:betagensurf}
\begin{aligned}
\beta_{22}=&\frac{1}{V_{k}}\int_{S_{C}}u_{2}n_{2}dS=\frac{1}{V_{k}}\left[\int_{S_{C}^{m}}u_{2}^{m}n_{2}^{m}dS+\int_{S_{C}^{f}}u_{2}^{f}n_{2}^{f}dS\right]=\\
=&\frac{1}{V_{k}}\left[\int_{S_{C}^{m}}u_{2}^{m}n_{2}^{m}dS+\int_{S_{C}^{f}}u_{2}^{f}\left(-n_{2}^{m}\right)dS\right]\\
\beta_{33}=&\frac{1}{V_{k}}\int_{S_{C}}u_{3}n_{3}dS=\frac{1}{V_{k}}\left[\int_{S_{C}^{m}}u_{3}^{m}n_{3}^{m}dS+\int_{S_{C}^{f}}u_{3}^{f}n_{3}^{f}dS\right]=\\
=&\frac{1}{V_{k}}\left[\int_{S_{C}^{m}}u_{3}^{m}n_{3}^{m}dS+\int_{S_{C}^{f}}u_{3}^{f}\left(-n_{3}^{m}\right)dS\right]\\
\beta_{23}=&\frac{1}{V_{k}}\int_{S_{C}}\left(u_{2}n_{3}+u_{3}n_{2}\right)dS=\\
=&\frac{1}{V_{k}}\left[\int_{S_{C}^{m}}\left(u_{2}^{m}n_{3}^{m}+u_{3}^{m}n_{2}^{m}\right)dS+\int_{S_{C}^{f}}\left(u_{2}^{f}n_{3}^{f}+u_{3}^{f}n_{2}^{f}\right)dS\right]=\\
=&\frac{1}{V_{k}}\left[\int_{S_{C}^{m}}u_{2}^{m}n_{3}^{m}dS+\int_{S_{C}^{f}}u_{2}^{f}\left(-n_{3}^{m}\right)dS+\int_{S_{C}^{m}}u_{3}^{m}n_{2}^{m}dS+\int_{S_{C}^{f}}u_{3}^{f}\left(-n_{2}^{m}\right)dS\right]\\
\end{aligned}
\end{equation}

The total matrix debonded surface $S_{C}^{m}$ is equal to the total fiber debonded surface $S_{C}^{f}$ and equal to:

\begin{equation}\label{eq:cracksurfaces}
S_{C}^{m}=S_{C}^{f}=n_{D}R_{f}\Delta\theta
\end{equation}

With Eq.~\ref{eq:cracksurfaces}, we can recast Eq.~\ref{eq:betagensurf} as

\begin{equation}\label{eq:betarho}
\begin{aligned}
\beta_{22}=&\frac{1}{V_{k}}\left[n_{D}R_{f}w\int_{0}^{\Delta\theta}\left(u_{2}^{m}-u_{2}^{f}\right)n_{2}^{m}d\theta\right]=\\
=&\frac{1}{L_{lam}wt_{90^{\circ}}}\left[n_{D}R_{f}w\int_{0}^{\Delta\theta}\left(u_{2}^{m}-u_{2}^{f}\right)n_{2}^{m}d\theta\right]=\\
=&\frac{1}{L_{lam}}\frac{n_{D}R_{f}}{t_{90^{\circ}}}\left[\int_{0}^{\Delta\theta}\left(u_{2}^{m}-u_{2}^{f}\right)n_{2}^{m}d\theta\right]=\\
=&\rho_{D}\left[\frac{1}{\Delta\theta}\int_{0}^{\Delta\theta}\left(u_{2}^{m}-u_{2}^{f}\right)n_{2}^{m}d\theta\right]\\
\beta_{33}=&\rho_{D}\left[\frac{1}{\Delta\theta}\int_{0}^{\Delta\theta}\left(u_{3}^{m}-u_{3}^{f}\right)n_{3}^{m}d\theta\right]\\
\beta_{23}=&\rho_{D}\left[\frac{1}{\Delta\theta}\int_{0}^{\Delta\theta}\left(u_{2}^{m}-u_{2}^{f}\right)n_{3}^{m}d\theta+\frac{1}{\Delta\theta}\int_{0}^{\Delta\theta}\left(u_{3}^{m}-u_{3}^{f}\right)n_{2}^{m}d\theta\right]
\end{aligned}
\end{equation}

We can express the displacement jumps at the interface as a function of the local Crack Opening Displacement (COD) and Crack Sliding Displacement (CSD) as

\begin{equation}
\begin{aligned}
u_{2}^{m}-u_{2}^{f}=&\left(u_{r}^{m}-u_{r}^{f}\right)\cos{\left(\theta\right)}-\left(u_{\theta}^{m}-u_{\theta}^{f}\right)\sin{\left(\theta\right)}=\\
=&COD\left(\theta\right)\cos{\left(\theta\right)}-CSD\left(\theta\right)\sin{\left(\theta\right)}\\
u_{3}^{m}-u_{3}^{f}=&\left(u_{r}^{m}-u_{r}^{f}\right)\sin{\left(\theta\right)}+\left(u_{\theta}^{m}-u_{\theta}^{f}\right)\cos{\left(\theta\right)}=\\
=&COD\left(\theta\right)\sin{\left(\theta\right)}+CSD\left(\theta\right)\cos{\left(\theta\right)}\\
\end{aligned}
\end{equation}

where $\theta$ is the local angular coordinate at the interface. We can similarly express $n_{2}^{m}$ and $n_{3}^{m}$ as a function of $\theta$:

\begin{equation}
\begin{aligned}
n_{2}^{m}&=\cos{\left(\theta\right)}-\sin{\left(\theta\right)}\\
n_{3}^{m}&=\sin{\left(\theta\right)}+\cos{\left(\theta\right)}
\end{aligned}
\end{equation}

Thus, Eq.~\ref{eq:betarho} becomes

\begin{equation}
\begin{aligned}
&\beta_{22}=\\=\rho_{D}&\frac{1}{\Delta\theta}\int_{0}^{\Delta\theta}\left[COD\left(\theta\right)\left(\cos^{2}{\left(\theta\right)}-\cos{\left(\theta\right)}\sin{\left(\theta\right)}\right)-CSD\left(\theta\right)\left(\sin{\left(\theta\right)}\cos{\left(\theta\right)}-\sin^{2}{\left(\theta\right)}\right)\right]d\theta=\\
\\=\rho_{D}&\frac{1}{2\Delta\theta}\int_{0}^{\Delta\theta}\left[COD\left(\theta\right)\left(1+\cos\left(2\theta\right)-\sin\left(2\theta\right)\right)+CSD\left(\theta\right)\left(1-\cos\left(2\theta\right)-\sin\left(2\theta\right)\right)\right]d\theta=\\
\\=\rho_{D}&\frac{1}{\Delta\theta}\int_{0}^{\Delta\theta}\left[\frac{COD\left(\theta\right)+CSD\left(\theta\right)}{2}\left(1-\sin\left(2\theta\right)\right)+\frac{COD\left(\theta\right)-CSD\left(\theta\right)}{2}\cos\left(2\theta\right)\right]d\theta\\
&\beta_{33}=\\
=\rho_{D}&\frac{1}{\Delta\theta}\int_{0}^{\Delta\theta}\left[COD\left(\theta\right)\left(\sin{\left(\theta\right)}\cos{\left(\theta\right)}+\sin^{2}{\left(\theta\right)}\right)+CSD\left(\theta\right)\left(\cos^{2}{\left(\theta\right)}+\cos{\left(\theta\right)}\sin{\left(\theta\right)}\right)\right]d\theta=\\
=\rho_{D}&\frac{1}{2\Delta\theta}\int_{0}^{\Delta\theta}\left[COD\left(\theta\right)\left(1+\sin\left(2\theta\right)-\cos\left(2\theta\right)\right)+CSD\left(\theta\right)\left(1+\sin\left(2\theta\right)+\cos\left(2\theta\right)\right)\right]d\theta=\\
=\rho_{D}&\frac{1}{\Delta\theta}\int_{0}^{\Delta\theta}\left[\frac{COD\left(\theta\right)+CSD\left(\theta\right)}{2}\left(1+\sin\left(2\theta\right)\right)-\frac{COD\left(\theta\right)-CSD\left(\theta\right)}{2}\cos\left(2\theta\right)\right]d\theta\\
&\beta_{23}=\\
&=\rho_{D}\frac{1}{\Delta\theta}\int_{0}^{\Delta\theta}COD\left(\theta\right)\left(2\sin{\left(\theta\right)}\cos{\left(\theta\right)}+\cos^{2}{\left(\theta\right)}-\sin^{2}\left(\theta\right)\right)+\\
&\qquad-\rho_{D}\frac{1}{\Delta\theta}\int_{0}^{\Delta\theta}CSD\left(\theta\right)\left(\sin^{2}{\left(\theta\right)}-\cos^{2}{\left(\theta\right)}+2\cos{\left(\theta\right)}\sin{\left(\theta\right)}\right)d\theta=\\
=\rho_{D}&\frac{1}{\Delta\theta}\int_{0}^{\Delta\theta}2\left[\frac{COD\left(\theta\right)+CSD\left(\theta\right)}{2}\cos\left(2\theta\right)+\frac{COD\left(\theta\right)-CSD\left(\theta\right)}{2}\sin\left(2\theta\right)\right]d\theta\\
\end{aligned}
\end{equation}

\subsection{Analytical modeling of $COD\left(\theta\right)$ and $CSD\left(\theta\right)$}

The Crack Opening Displacement (COD) and Crack Sliding Displacement (CSD) are in general a function of $\theta$, the angular coordinate along the crack which varies between $0$ and $\Delta\theta$. Without making any approximation, the Crack Opening Displacement (COD) and Crack Sliding Displacement (CSD) can be expressed as the sum of their average value and a term, respectively $\delta COD\left(\theta\right)$ and $\delta CSD\left(\theta\right)$, that represents the variation of the function from its average:

\begin{equation}
\begin{aligned}
COD\left(\theta\right)&=COD_{avg}+\delta COD\left(\theta\right)\\
CSD\left(\theta\right)&=CSD_{avg}+\delta CSD\left(\theta\right).
\end{aligned}
\end{equation}

By defining $\Delta\Psi$

\begin{equation}
\Delta\Psi=\min\left(\Delta\theta,\Delta\Phi\right),
\end{equation}

we introduce at this point an approximation and assume that the functions $\delta COD\left(\theta\right)$ and $\delta CSD\left(\theta\right)$ can be expressed as the product of the maximum value of the displacement and a function, respectively $f\left(\theta-\frac{\Delta\Psi}{2}\right)$ and $g\left(\theta-\frac{\Delta\theta}{2}\right)$:

\begin{equation}
\begin{aligned}
COD\left(\theta\right)&=COD_{avg}+\delta COD\left(\theta\right)=COD_{avg}+COD_{max}f\left(\theta-\frac{\Delta\Psi}{2}\right)\\
CSD\left(\theta\right)&=CSD_{avg}+\delta CSD\left(\theta\right)=CSD_{avg}+CSD_{max}g\left(\theta-\frac{\Delta\theta}{2}\right),
\end{aligned}
\end{equation}

where $f\left(\theta-\frac{\Delta\Psi}{2}\right)$ and $g\left(\theta-\frac{\Delta\theta}{2}\right)$ are assumed to be odd functions over their respective integration domain $\left[0,\Delta\Psi\right]$ and $\left[0,\Delta\theta\right]$

\begin{equation}\label{eq:integcondition}
\int_{0}^{\Delta\theta}f\left(\theta-\frac{\Delta\Psi}{2}\right)d\theta=0\quad\int_{0}^{\Delta\theta}g\left(\theta-\frac{\Delta\theta}{2}\right)d\theta=0.
\end{equation}

We assume the two functions $f\left(\theta-\frac{\Delta\Psi}{2}\right)$ and $g\left(\theta-\frac{\Delta\theta}{2}\right)$ to be two odd polynomials of degree $2n-1$:

\begin{equation}
\begin{aligned}
f\left(\theta-\frac{\Delta\Psi}{2}\right)&=
\begin{cases}
&\sum_{k=0}^{n-1}a_{2k+1}\left(\theta-\frac{\Delta\Psi}{2}\right)^{2k+1}\quad0\leq\theta\leq\Delta\Psi\\
&0\quad otherwise
\end{cases}\\
g\left(\theta-\frac{\Delta\theta}{2}\right)&=
\begin{cases}
&\sum_{k=0}^{n-1}b_{2k+1}\left(\theta-\frac{\Delta\theta}{2}\right)^{2k+1}\quad0\leq\theta\leq\Delta\theta\\
&0\quad otherwise
\end{cases}
\end{aligned}
\end{equation}

which satisfy by construction the conditions expressed in Equation~\ref{eq:integcondition}. The coefficients $a_{2k+1}$ and $b_{2k+1}$ are determined by imposing that

\begin{equation}
\begin{aligned}
COD\left(\Delta\Psi\right)&=0\\
CSD\left(\Delta\theta\right)&=0.
\end{aligned}
\end{equation}

The explicit construction of the polynomials $f\left(\theta-\frac{\Delta\Psi}{2}\right)$ and $g\left(\theta-\frac{\Delta\theta}{2}\right)$ for $n=1,2,3$ (or degree $2n-1=1,3,5$) is reported in \ref{app:fandgexplicit}.

\begin{equation}
\begin{aligned}
&\beta_{22}=\\
=\rho_{D}&\frac{1}{\Delta\theta}\int_{0}^{\Delta\theta}\left[\frac{COD\left(\theta\right)+CSD\left(\theta\right)}{2}\left(1-\sin\left(2\theta\right)\right)+\frac{COD\left(\theta\right)-CSD\left(\theta\right)}{2}\cos\left(2\theta\right)\right]d\theta\\
&\beta_{33}=\\
=\rho_{D}&\frac{1}{\Delta\theta}\int_{0}^{\Delta\theta}\left[\frac{COD\left(\theta\right)+CSD\left(\theta\right)}{2}\left(1+\sin\left(2\theta\right)\right)-\frac{COD\left(\theta\right)-CSD\left(\theta\right)}{2}\cos\left(2\theta\right)\right]d\theta\\
&\beta_{23}=\\
=\rho_{D}&\frac{1}{\Delta\theta}\int_{0}^{\Delta\theta}2\left[\frac{COD\left(\theta\right)+CSD\left(\theta\right)}{2}\cos\left(2\theta\right)+\frac{COD\left(\theta\right)-CSD\left(\theta\right)}{2}\sin\left(2\theta\right)\right]d\theta\\
\end{aligned}
\end{equation}

\begin{equation}
\begin{aligned}
\frac{1}{\Delta\theta}&\int_{0}^{\Delta\theta}COD\left(\theta\right)d\theta=\\
=&\frac{1}{\Delta\theta}\int_{0}^{\Delta\theta}\left(COD_{avg}+COD_{max}\sum_{k=0}^{n-1}a_{2k+1}\left(\theta-\frac{\Delta\Psi}{2}\right)^{2k+1}\right)d\theta=\\
=&\frac{1}{\Delta\theta}\int_{0}^{\Delta\theta}COD_{avg}d\theta+\\
+&\frac{1}{\Delta\theta}\int_{0}^{\Delta\Psi}COD_{max}\sum_{k=0}^{n-1}a_{2k+1}\left(\theta-\frac{\Delta\Psi}{2}\right)^{2k+1}d\theta+\\
+&\frac{1}{\Delta\theta}\cancelto{0}{\int_{\Delta\Psi}^{\Delta\theta}COD_{max}\sum_{k=0}^{n-1}a_{2k+1}\left(\theta-\frac{\Delta\Psi}{2}\right)^{2k+1}d\theta}=\\
=&\frac{1}{\Delta\theta}\left[COD_{avg}\theta\right]\Bigg\rvert_{0}^{\Delta\theta}+\frac{1}{\Delta\theta}\left[COD_{max}\sum_{k=0}^{n-1}\frac{a_{2k+1}}{2\left(k+1\right)}\left(\theta-\frac{\Delta\Psi}{2}\right)^{2\left(k+1\right)}\right]\Bigg\rvert_{0}^{\Delta\Psi}=\\
=&COD_{avg}+COD_{max}\sum_{k=0}^{n-1}\frac{a_{2k+1}}{2\left(k+1\right)\Delta\theta}\left(\left(\frac{\Delta\Psi}{2}\right)^{2\left(k+1\right)}-\left(-\frac{\Delta\Psi}{2}\right)^{2\left(k+1\right)}\right)=\\
=&COD_{avg}
\end{aligned}
\end{equation}

\begin{equation}
\begin{aligned}
\frac{1}{\Delta\theta}&\int_{0}^{\Delta\theta}CSD\left(\theta\right)d\theta=CSD_{avg}
\end{aligned}
\end{equation}

\begin{equation}
\footnotesize
\begin{aligned}
&\frac{1}{\Delta\theta}\int_{0}^{\Delta\theta}COD\left(\theta\right)\sin\left(2\theta\right)d\theta=\\
=&\frac{1}{\Delta\theta}\int_{0}^{\Delta\theta}\left(COD_{avg}+COD_{max}\sum_{k=0}^{n-1}a_{2k+1}\theta^{2k+1}\right)\sin\left(2\theta\right)d\theta=\\
=&-\frac{1}{2\Delta\theta}COD_{avg}\left[\cos\left(2\theta\right)\right]\Bigg\rvert_{0}^{\Delta\theta}+\\
+&\frac{1}{\Delta\theta}\left[COD_{max}\sum_{i=0}^{n}\left(-\frac{1}{2}\right)^{i+1}\sin(\frac{1+mod\left(i,2\right)}{2}\pi-2\theta)\left(\sum_{k=0}^{n-i}a_{k}\left(n-\left(k+1\right)\right)!\theta^{k}\right)\right]\Bigg\rvert_{0}^{\Delta\Psi}=\\
=&\frac{1-\cos\left(2\Delta\theta\right)}{2\Delta\theta}COD_{avg}+\\
+&\frac{1}{\Delta\theta}COD_{max}\sum_{i=0}^{n}\left(-\frac{1}{2}\right)^{i+1}\sin\left(\frac{1+mod\left(i,2\right)}{2}\pi-2\Delta\Psi\right)\left(\sum_{k=0}^{n-i}a_{k}\left(n-\left(k+1\right)\right)!\Delta\Psi^{k}\right)
\end{aligned}
\end{equation}

\bibliography{refs}

\appendix
\section{Explicit expressions for $f\left(\theta\right)$ and $g\left(\theta\right)$}\label{app:fandgexplicit}

In the following, recall that

\begin{equation}
\Delta\Psi=\min\left(\Delta\theta,\Delta\Phi\right).
\end{equation}

\begin{description}
\item[$\mathbf{n=1}$]

\begin{equation}
\begin{aligned}
f\left(\theta-\frac{\Delta\Psi}{2}\right)&=\sum_{k=0}^{0}a_{2k+1}\left(\theta-\frac{\Delta\Psi}{2}\right)^{2k+1}=a_{1}\left(\theta-\frac{\Delta\Psi}{2}\right)\\
g\left(\theta-\frac{\Delta\theta}{2}\right)&=\sum_{k=0}^{0}b_{2k+1}\left(\theta-\frac{\Delta\theta}{2}\right)^{2k+1}=b_{1}\left(\theta-\frac{\Delta\theta}{2}\right)
\end{aligned}
\end{equation}

\begin{equation}
\begin{aligned}
\int_{0}^{\Delta\Psi}&f\left(\theta-\frac{\Delta\Psi}{2}\right)d\theta=\int_{0}^{\Delta\Psi}a_{1}\left(\theta-\frac{\Delta\Psi}{2}\right) d\theta=\left[\frac{a_{1}}{2}\theta^{2}-a_{1}\frac{\Delta\Psi}{2}\theta\right]\Bigg\rvert_{0}^{\Delta\Psi}=0\quad\forall a_{1}\\
\int_{0}^{\Delta\theta}&g\left(\theta-\frac{\Delta\theta}{2}\right)d\theta=\int_{0}^{\Delta\theta}b_{1}\left(\theta-\frac{\Delta\theta}{2}\right) d\theta=\left[\frac{b_{1}}{2}\theta^{2}-b_{1}\frac{\Delta\theta}{2}\theta\right]\Bigg\rvert_{0}^{\Delta\theta}=0\quad\forall b_{1}
\end{aligned}
\end{equation}

\begin{equation}
\begin{aligned}
&COD_{avg}+COD_{max}a_{1}\left(\Delta\Psi-\frac{\Delta\Psi}{2}\right)=0\rightarrow a_{1}=-\frac{2}{\Delta\Psi}\frac{COD_{avg}}{COD_{max}}\\
&CSD_{avg}+CSD_{max}b_{1}\left(\Delta\theta-\frac{\Delta\theta}{2}\right)=0\rightarrow b_{1}=-\frac{2}{\Delta\theta}\frac{CSD_{avg}}{CSD_{max}}
\end{aligned}
\end{equation}

\begin{equation}
\begin{aligned}
\sum_{i=0}^{1}&\left(-\frac{1}{2}\right)^{i+1}\sin\left(\frac{1+mod\left(i,2\right)}{2}\pi-2\Delta\Psi\right)\left(\sum_{k=0}^{1-i}a_{k}\left(1-\left(k+1\right)\right)!\Delta\Psi^{k}\right)=\\
=&\left(-\frac{1}{2}\right)^{1}\sin\left(\frac{1+mod\left(0,2\right)}{2}\pi-2\Delta\Psi\right)\left(\sum_{k=0}^{1}a_{k}\left(1-\left(k+1\right)\right)!\Delta\Psi^{k}\right)+\\
+&\left(-\frac{1}{2}\right)^{2}\sin\left(\frac{1+mod\left(1,2\right)}{2}\pi-2\Delta\Psi\right)\left(\sum_{k=0}^{0}a_{k}↑\left(1-\left(k+1\right)\right)!\Delta\Psi^{k}\right)=\\
=&-\frac{1}{2}\cos\left(2\Delta\Psi\right)\left(\sum_{k=0}^{1}a_{k}\left(1-\left(k+1\right)\right)!\Delta\Psi^{k}\right)+\\
&+\frac{1}{4}\sin\left(2\Delta\Psi\right)\left(\sum_{k=0}^{0}a_{k}\left(1-\left(k+1\right)\right)!\Delta\Psi^{k}\right)=\\
=&-\frac{1}{2}\cos\left(2\Delta\Psi\right)\left(a_{0}+a_{1}\left(n-\left(k+1\right)\right)!\Delta\Psi^{k}\right)+\\
&+\frac{1}{4}\sin\left(2\Delta\Psi\right)\left(\sum_{k=0}^{0}a_{k}\left(n-\left(k+1\right)\right)!\Delta\Psi^{k}\right)=\\
\end{aligned}
\end{equation}

\item[$\mathbf{n=2}$]

\begin{equation}
\begin{aligned}
f\left(\theta-\frac{\Delta\Psi}{2}\right)&=\sum_{k=0}^{1}a_{2k+1}\left(\theta-\frac{\Delta\Psi}{2}\right)^{2k+1}=a_{1}\left(\theta-\frac{\Delta\Psi}{2}\right)+a_{3}\left(\theta-\frac{\Delta\Psi}{2}\right)^{3}\\
g\left(\theta-\frac{\Delta\theta}{2}\right)&=\sum_{k=0}^{1}b_{2k+1}\left(\theta-\frac{\Delta\theta}{2}\right)^{2k+1}=b_{1}\left(\theta-\frac{\Delta\theta}{2}\right)+b_{3}\left(\theta-\frac{\Delta\theta}{2}\right)^{3}\\
\end{aligned}
\end{equation}

\item[$\mathbf{n=3}$]

\begin{equation}
\begin{aligned}
f\left(\theta-\frac{\Delta\Psi}{2}\right)&=\sum_{k=0}^{2}a_{2k+1}\left(\theta-\frac{\Delta\Psi}{2}\right)^{2k+1}=\\&=a_{1}\left(\theta-\frac{\Delta\Psi}{2}\right)+a_{3}\left(\theta-\frac{\Delta\Psi}{2}\right)^{3}+a_{5}\left(\theta-\frac{\Delta\Psi}{2}\right)^{5}\\
g\left(\theta-\frac{\Delta\theta}{2}\right)&=\sum_{k=0}^{1}b_{2k+1}\left(\theta-\frac{\Delta\theta}{2}\right)^{2k+1}=\\&=b_{1}\left(\theta-\frac{\Delta\theta}{2}\right)+b_{3}\left(\theta-\frac{\Delta\theta}{2}\right)^{3}+b_{5}\left(\theta-\frac{\Delta\theta}{2}\right)^{5}\\
\end{aligned}
\end{equation}
\end{description}

\end{document}
