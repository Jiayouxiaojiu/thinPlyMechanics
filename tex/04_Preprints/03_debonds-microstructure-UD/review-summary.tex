\documentclass{article}

\usepackage{amssymb}
\usepackage{booktabs}
\usepackage[a4paper,margin=1in,landscape]{geometry}
\usepackage{ltxtable}
\usepackage{pifont}
\usepackage{tabularx}
\usepackage{textcomp}
\usepackage{xcolor}

\newcommand{\cmark}{\ding{52}}
\newcommand{\xmark}{\ding{55}}

\begin{document}
\LTXtable{\textwidth}{review-summary-table}
%\begin{table}[!t]
%\begin{tabularx}{\textwidth}{ccccXcX}
%\bf{N.}&\bf{Status}&\bf{Reviewer}&\bf{Line(s)}&\bf{Observation}&\bf{Action}&\bf{Comments}\\
%&\textcolor{red}{\xmark}\textrightarrow To-Do&&&&&\\
%&\textcolor{orange}{$\blacksquare$}\textrightarrow In Progress&&&&&\\
%&\textcolor{green}{\cmark}\textrightarrow Reviewed&&&&&\\
%\toprule
%\midrule
%1&\textcolor{green}{\cmark}&1&title&The title should be changed: terms “debond-debond” and “debond-free” are not mentioned in the text, only in the title.&\textit{Changed}&The expressions \textit{debond-debond} and \textit{debond-free boundary} are in effect mentioned only in the title. They were meant to be synthetic expressions to summarize the two main points of discussion of the paper. In order to avoid confusion and clarify the message of the article, the expressions \textit{debond-debond and debond-free boundary interactions} have been changed to \textit{effect of the fiber volume fraction and of the distance to the free surface and to non-adjacent debonds}.\\
%\midrule
%2&\textcolor{green}{\cmark}&1&78&The authors used “square packing of fibers” (Line 78). This should be commented since other patterns are also used - the authors themselves refer to results for “a hexagonal cluster” (Line 71).&\textit{Added}&Added comment motivating the choice of a square-packing configuration.\\
%\midrule
%3&\textcolor{red}{\xmark}&1&177-178&Lines 177-178: “Due to its appearance, frictionless contact is considered between the two crack faces to allow free sliding and avoid interpenetration.” The strength of this assumption and its effect on the obtained results should be discussed.&&\\
%\midrule
%4&\textcolor{red}{\xmark}&1&196-197&Lines 196-197: “…it is assumed that their response lies always in the linear elastic domain.” The level of maximum local strains for cases with high volume fractions of fibers could justify (or not) this assumption.&&\\
%\midrule
%5&\textcolor{green}{\cmark}&1&359&Line 359: “101st” and “201st” instead of “101th” and “201th”.&\textit{Changed}&\\
%\midrule
%6&\textcolor{red}{\xmark}&1&487&Line 487: “The fiber volume fraction is the same everywhere…” is misleading since a range of this parameter was studied.”&&\\
%\midrule
%7&\textcolor{red}{\xmark}&1&491-494&Conclusion 1 (Lines 491-494) should also provide some idea on the magnitude of this “characteristic distance between debonds which defines the transition to a non-interactive solution”, mentioning also that it was “outside the range studied” (Line 384).&&\\
%\midrule
%8&\textcolor{red}{\xmark}&2&title&although in Section 2.4 the authors present a successful model validation, it would be useful for the reader to mention the amount of degrees of freedom the developed model requires.&&\\
%\end{tabularx}
%\end{table}

\end{document}