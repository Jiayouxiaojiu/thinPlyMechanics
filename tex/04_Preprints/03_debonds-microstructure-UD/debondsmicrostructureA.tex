\documentclass[review]{elsarticle}

\usepackage{amsmath}
\usepackage{subcaption}
\usepackage[usenames]{xcolor}
\usepackage{lineno,hyperref}
\modulolinenumbers[5]

\journal{TBA}

%%%%%%%%%%%%%%%%%%%%%%%
%% Elsevier bibliography styles
%%%%%%%%%%%%%%%%%%%%%%%
%% To change the style, put a % in front of the second line of the current style and
%% remove the % from the second line of the style you would like to use.
%%%%%%%%%%%%%%%%%%%%%%%

%% Numbered
%\bibliographystyle{model1-num-names}

%% Numbered without titles
%\bibliographystyle{model1a-num-names}

%% Harvard
%\bibliographystyle{model2-names.bst}\biboptions{authoryear}

%% Vancouver numbered
%\usepackage{numcompress}\bibliographystyle{model3-num-names}

%% Vancouver name/year
%\usepackage{numcompress}\bibliographystyle{model4-names}\biboptions{authoryear}

%% APA style
%\bibliographystyle{model5-names}\biboptions{authoryear}

%% AMA style
%\usepackage{numcompress}\bibliographystyle{model6-num-names}

%% `Elsevier LaTeX' style
\bibliographystyle{elsarticle-num}
%%%%%%%%%%%%%%%%%%%%%%%

\begin{document}

\begin{frontmatter}

\title{Effect of uniform distributions of bonded and debonded fibers on the growth of the fiber/matrix interface crack in UD laminates with different fiber contents under transverse loading}
%\tnotetext[mytitlenote]{Fully documented templates are available in the elsarticle package on \href{http://www.ctan.org/tex-archive/macros/latex/contrib/elsarticle}{CTAN}.}

%% Group authors per affiliation:
%\author{Luca Di Stasio\fnref{myfootnote}}
%\address{Radarweg 29, Amsterdam}
%\fntext[myfootnote]{Since 1880.}

%% or include affiliations in footnotes:
\author[nancy,lulea]{Luca Di Stasio}
\author[lulea]{Janis Varna}
\author[nancy]{Zoubir Ayadi}
%\ead[url]{www.elsevier.com}

%\author[mysecondaryaddress]{Global Customer Service\corref{mycorrespondingauthor}}
%\cortext[mycorrespondingauthor]{Corresponding author}
%\ead{support@elsevier.com}

\address[nancy]{Universit\'e de Lorraine, EEIGM, IJL, 6 Rue Bastien Lepage, F-54010 Nancy, France}
\address[lulea]{Lule\aa\ University of Technology, University Campus, SE-97187 Lule\aa, Sweden}

\begin{abstract}
\noindent
\textcolor{purple}{{\em Priority}: 1}\\
\textcolor{purple}{{\em Type}: long article}\\
\textcolor{purple}{{\em Target journal(s)}: Composites Part B: Engineering, Composites Part A: Applied Science and Manufacturing, Composite Science and Technology, Composite Structures, Journal of Composite Materials, Composite Communications}\\

\end{abstract}

%\begin{keyword}
%\texttt{elsarticle.cls}\sep \LaTeX\sep Elsevier \sep template
%\MSC[2010] 00-01\sep  99-00
%\end{keyword}

\end{frontmatter}

\linenumbers

\section{Introduction}

\textcolor{blue}{
\begin{enumerate}
\item We start with a few lines devoted to the spread tow technology and thin plies: what they are, what can be done, what are the possible applications.
\item By quoting the relevant references, we report on the observation that one of the main beneficial mechanisms in thin ply is the retardation of transverse crack propagation. We then enlarge by reporting the microscopical observations by Saito, in which debonds where also observed. We observe that available microscopic observations are just a few and mainly in 2D.
\item Propagation of transverse cracks has been widely investigated both analytically and numerically
\item Initiation at the level of fiber/matrix interface is instead a less researched subject.
\item cohesive elements are a possible choice, but have some drawbacks, which makes a LEFM approach valuable
\item With regard to LEFM studies of laminates under transverse loading, models can be found in the literature about: the single fiber in infinite matrix under different mode of loading, the effect of adjacent fibers on a fiber in infinite matrix under different mode of loading, the single fiber in an equivalent composite in transverse tension, the effect of adjacent fibers on a fiber in an equivalent composite in transverse tension.
\item For initiation of transverse cracking at the fiber/matrix interface in UD laminates under transverse tension, there is thus a gap regarding: the effect of fiber volume fraction; the interaction of debonded and bonded fibers in micro-structured assemblies, i.e. no homogenization. This article addresses these two points.
\item We conclude the introduction with a summary of the article's structure.
\end{enumerate}
}

\section{RVE models \& FE discretization}

\subsection{Models of Representative Volume Element(RVE)}

\textcolor{blue}{We start by describing the different idealized micro-structures considered and the corresponding repeating element or RVE used to model them.}

\begin{figure}[!h]
\centering
    \begin{subfigure}[b]{0.45\textwidth}
        %\includegraphics[width=\textwidth]{}
        \caption{A debonded fiber every 2 fully bonded ones.}\label{subfig:every2}
    \end{subfigure} ~
    \begin{subfigure}[b]{0.45\textwidth}
        %\includegraphics[width=\textwidth]{}
        \caption{Central debonded fiber with 1 fiber each side.}\label{subfig:1eachside}
    \end{subfigure}

    \begin{subfigure}[b]{0.45\textwidth}
        %\includegraphics[width=\textwidth]{}
        \caption{A debonded fiber every 4 fully bonded ones.}\label{subfig:every4}
    \end{subfigure} ~
    \begin{subfigure}[b]{0.45\textwidth}
        %\includegraphics[width=\textwidth]{}
        \caption{Central debonded fiber with 2 fibers each side.}\label{subfig:2eachside}
    \end{subfigure}

    \begin{subfigure}[b]{0.45\textwidth}
        %\includegraphics[width=\textwidth]{}
        \caption{A debonded fiber every 6 fully bonded ones.}\label{subfig:every6}
    \end{subfigure} ~
    \begin{subfigure}[b]{0.45\textwidth}
        %\includegraphics[width=\textwidth]{}
        \caption{Central debonded fiber with 3 fibers each side.}\label{subfig:3eachside}
    \end{subfigure}
\caption{Models of single-ply laminates with a single layer of fibers and debonds repeating at different distances (left column), and corresponding Representative Volume Elements (right column) with symmetry applied on the lower boundary line. The interface crack is represented in red.}\label{fig:fibersOnSideModels}
\end{figure}

\begin{figure}[!h]
\centering
    \begin{subfigure}[b]{0.45\textwidth}
        %\includegraphics[width=\textwidth]{}
        \caption{3 layers with a central line of debonded fibers.}\label{subfig:3layers}
    \end{subfigure} ~
    \begin{subfigure}[b]{0.45\textwidth}
        %\includegraphics[width=\textwidth]{}
        \caption{Central debonded fiber with 1 fiber above.}\label{subfig:1above}
    \end{subfigure}

    \begin{subfigure}[b]{0.45\textwidth}
        %\includegraphics[width=\textwidth]{}
        \caption{5 layers with a central line of debonded fibers.}\label{subfig:5layers}
    \end{subfigure} ~
    \begin{subfigure}[b]{0.45\textwidth}
        %\includegraphics[width=\textwidth]{}
        \caption{Central debonded fiber with 2 fibers above.}\label{subfig:1above}
    \end{subfigure}

    \begin{subfigure}[b]{0.45\textwidth}
        %\includegraphics[width=\textwidth]{}
        \caption{7 layers with a central line of debonded fibers.}\label{subfig:7layers}
    \end{subfigure} ~
    \begin{subfigure}[b]{0.45\textwidth}
        %\includegraphics[width=\textwidth]{}
        \caption{Central debonded fiber with 3 fibers above.}\label{subfig:1above}
    \end{subfigure}
\caption{Models of single-ply laminates with multiple layers of fibers and a central line of debonded fibers (left column), and corresponding Representative Volume Elements (right column) with symmetry applied on the lower boundary line. The interface crack is represented in red.}\label{fig:fibersOnTopModels}
\end{figure}

\begin{figure}[!h]
\centering
    \begin{subfigure}[b]{0.45\textwidth}
        %\includegraphics[width=\textwidth]{}
        \caption{3 layers with a debonded fiber every 2 fully bonded ones in the central line of fibers.}\label{subfig:3layersevery2}
    \end{subfigure} ~
    \begin{subfigure}[b]{0.45\textwidth}
        %\includegraphics[width=\textwidth]{}
        \caption{Central debonded fiber with 1 fiber on each side and 1 above.}\label{subfig:1eachside1above}
    \end{subfigure}

    \begin{subfigure}[b]{0.45\textwidth}
        %\includegraphics[width=\textwidth]{}
        \caption{3 layers with a debonded fiber every 4 fully bonded ones in the central line of fibers.}\label{subfig:3layersevery4}
    \end{subfigure} ~
    \begin{subfigure}[b]{0.45\textwidth}
        %\includegraphics[width=\textwidth]{}
        \caption{Central debonded fiber with 2 fibers on each side and 1 above.}\label{subfig:2eachside1above}
    \end{subfigure}

    \begin{subfigure}[b]{0.45\textwidth}
        %\includegraphics[width=\textwidth]{}
        \caption{5 layers with a debonded fiber every 4 fully bonded ones in the central line of fibers.}\label{subfig:5layersevery4}
    \end{subfigure} ~
    \begin{subfigure}[b]{0.45\textwidth}
        %\includegraphics[width=\textwidth]{}
        \caption{Central debonded fiber with 2 fibers on each side and 2 above.}\label{subfig:2eachside2above}
    \end{subfigure}

    \begin{subfigure}[b]{0.45\textwidth}
        %\includegraphics[width=\textwidth]{}
        \caption{3 layers with a debonded fiber every 6 fully bonded ones in the central line of fibers.}\label{subfig:3layersevery6}
    \end{subfigure} ~
    \begin{subfigure}[b]{0.45\textwidth}
        %\includegraphics[width=\textwidth]{}
        \caption{Central debonded fiber with 3 fibers on each side and 1 above.}\label{subfig:3eachside1above}
    \end{subfigure}
\caption{Models of single-ply laminates with multiple layers of fibers with debonds repeating at different distances in the central line of fibers (left column), and corresponding Representative Volume Elements (right column) with symmetry applied on the lower boundary line.}\label{fig:fibersOnSideAndTopModels}
\end{figure}

\begin{figure}[!h]
\centering
    \begin{subfigure}[b]{0.45\textwidth}
        %\includegraphics[width=\textwidth]{}
        \caption{Single layer of debonded fibers.}\label{subfig:singlelayerdebfibers}
    \end{subfigure} ~
    \begin{subfigure}[b]{0.45\textwidth}
        %\includegraphics[width=\textwidth]{}
        \caption{Element with a single debonded fiber and free boundary on top.}\label{subfig:free}
    \end{subfigure}

    \begin{subfigure}[b]{0.45\textwidth}
        %\includegraphics[width=\textwidth]{}
        \caption{Infinite number of layers of debonded fibers.}\label{subfig:inflayersdebfibers}
    \end{subfigure} ~
    \begin{subfigure}[b]{0.45\textwidth}
        %\includegraphics[width=\textwidth]{}
        \caption{Element with a single debonded fiber and coupling of vertical displacements along the upper boundary.}\label{subfig:coupling}
    \end{subfigure}

\caption{Models of single-ply laminates with all the fibers debonded  (left column), and corresponding Representative Volume Elements (right column) with symmetry applied on the lower boundary line.}\label{fig:allFibersDebonded}
\end{figure}

\subsection{Finite Element (FE) discretization}

\textcolor{blue}{We describe the model implemented: schematic + description of parameters, formulation (LEFM, frictionless contact, VCCT, J-Integral), implementation of BCs, mesh.}

\begin{figure}[!h]
\centering
    \begin{subfigure}[b]{0.45\textwidth}
        %\includegraphics[width=\textwidth]{}
        \caption{Schematic of the model with its main parameters.}\label{subfig:modelschem}
    \end{subfigure} ~
    \begin{subfigure}[b]{0.45\textwidth}
        %\includegraphics[width=\textwidth]{}
        \caption{Detail of the mesh in the crack tip's neighborhood.}\label{subfig:meshdetail}
    \end{subfigure}

\caption{Details and main parameters of the Finite Element model.}\label{fig:FEmodel}
\end{figure}

\subsection{Validation of the model}

\textcolor{blue}{We mention once in this paper the validation of the model with respect to BEM results.}

\begin{figure}[!h]
\centering
%\includegraphics[width=\textwidth]{}
\caption{Validation of the single fiber model for the infinite matrix case with respect to the BEM solution in \cite{}.}\label{fig:validation}
\end{figure}

\section{Results \& Discussion}

\subsection{Effect of Fiber Volume Fraction}

\textcolor{blue}{Effect of fiber volume fraction: we show graphics of ERR vs $\Delta\theta$, one curve for each $V_{f}$, one graphic for each selected BC. Selected BC: free, coupling, some examples with fibers. The effect is similar for all the different BC cases, it's enough to show some of them to exemplify.}

\begin{figure}[!h]
\centering
    \begin{subfigure}[b]{0.45\textwidth}
        %\includegraphics[width=\textwidth]{}
        \caption{Single fiber model with free boundary on top.}\label{subfig:volfracfreeMI}
    \end{subfigure} ~
    \begin{subfigure}[b]{0.45\textwidth}
        %\includegraphics[width=\textwidth]{}
        \caption{Single fiber model with coupling of vertical displacements along the upper boundary.}\label{subfig:volfraccouplingMI}
    \end{subfigure}

    \begin{subfigure}[b]{0.45\textwidth}
        %\includegraphics[width=\textwidth]{}
        \caption{1 fiber each side.}\label{subfig:volfrac1eachsideMI}
    \end{subfigure} ~
    \begin{subfigure}[b]{0.45\textwidth}
        %\includegraphics[width=\textwidth]{}
        \caption{1 fiber above.}\label{subfig:volfrac1aboveMI}
    \end{subfigure}

    \begin{subfigure}[b]{0.45\textwidth}
        %\includegraphics[width=\textwidth]{}
        \caption{5 fibers each side.}\label{subfig:volfrac5eachsideMI}
    \end{subfigure} ~
    \begin{subfigure}[b]{0.45\textwidth}
        %\includegraphics[width=\textwidth]{}
        \caption{5 fibers above.}\label{subfig:volfrac5aboveMI}
    \end{subfigure}

    \begin{subfigure}[b]{0.45\textwidth}
        %\includegraphics[width=\textwidth]{}
        \caption{10 fibers each side.}\label{subfig:volfrac10eachsideMI}
    \end{subfigure} ~
    \begin{subfigure}[b]{0.45\textwidth}
        %\includegraphics[width=\textwidth]{}
        \caption{10 fibers above.}\label{subfig:volfrac10aboveMI}
    \end{subfigure}

    \begin{subfigure}[b]{0.45\textwidth}
        %\includegraphics[width=\textwidth]{}
        \caption{1 fiber each side, 1 above.}\label{subfig:volfrac1eachside1aboveMI}
    \end{subfigure} ~
    \begin{subfigure}[b]{0.45\textwidth}
        %\includegraphics[width=\textwidth]{}
        \caption{3 fibers each side, 1 above.}\label{subfig:volfrac3eachside1aboveMI}
    \end{subfigure}

    \begin{subfigure}[b]{0.45\textwidth}
        %\includegraphics[width=\textwidth]{}
        \caption{2 fibers each side, 2 above.}\label{subfig:volfrac2eachside2aboveMI}
    \end{subfigure} ~
    \begin{subfigure}[b]{0.45\textwidth}
        %\includegraphics[width=\textwidth]{}
        \caption{5 fibers each side, 2 above.}\label{subfig:volfrac5eachside2aboveMI}
    \end{subfigure}

\caption{A view of the effect of fiber volume fraction on Mode I ERR across different models.}\label{fig:volumefractionMI}
\end{figure}

\begin{figure}[!h]
\centering
    \begin{subfigure}[b]{0.45\textwidth}
        %\includegraphics[width=\textwidth]{}
        \caption{Single fiber model with free boundary on top.}\label{subfig:volfracfreeMII}
    \end{subfigure} ~
    \begin{subfigure}[b]{0.45\textwidth}
        %\includegraphics[width=\textwidth]{}
        \caption{Single fiber model with coupling of vertical displacements along the upper boundary.}\label{subfig:volfraccouplingMII}
    \end{subfigure}

    \begin{subfigure}[b]{0.45\textwidth}
        %\includegraphics[width=\textwidth]{}
        \caption{1 fiber each side.}\label{subfig:volfrac1eachsideMII}
    \end{subfigure} ~
    \begin{subfigure}[b]{0.45\textwidth}
        %\includegraphics[width=\textwidth]{}
        \caption{1 fiber above.}\label{subfig:volfrac1aboveMII}
    \end{subfigure}

    \begin{subfigure}[b]{0.45\textwidth}
        %\includegraphics[width=\textwidth]{}
        \caption{5 fibers each side.}\label{subfig:volfrac5eachsideMII}
    \end{subfigure} ~
    \begin{subfigure}[b]{0.45\textwidth}
        %\includegraphics[width=\textwidth]{}
        \caption{5 fibers above.}\label{subfig:volfrac5aboveMII}
    \end{subfigure}

    \begin{subfigure}[b]{0.45\textwidth}
        %\includegraphics[width=\textwidth]{}
        \caption{10 fibers each side.}\label{subfig:volfrac10eachsideMII}
    \end{subfigure} ~
    \begin{subfigure}[b]{0.45\textwidth}
        %\includegraphics[width=\textwidth]{}
        \caption{10 fibers above.}\label{subfig:volfrac10aboveMII}
    \end{subfigure}

    \begin{subfigure}[b]{0.45\textwidth}
        %\includegraphics[width=\textwidth]{}
        \caption{1 fiber each side, 1 above.}\label{subfig:volfrac1eachside1aboveMII}
    \end{subfigure} ~
    \begin{subfigure}[b]{0.45\textwidth}
        %\includegraphics[width=\textwidth]{}
        \caption{3 fibers each side, 1 above.}\label{subfig:volfrac3eachside1aboveMII}
    \end{subfigure}

    \begin{subfigure}[b]{0.45\textwidth}
        %\includegraphics[width=\textwidth]{}
        \caption{2 fibers each side, 2 above.}\label{subfig:volfrac2eachside2aboveMII}
    \end{subfigure} ~
    \begin{subfigure}[b]{0.45\textwidth}
        %\includegraphics[width=\textwidth]{}
        \caption{5 fibers each side, 2 above.}\label{subfig:volfrac5eachside2aboveMII}
    \end{subfigure}

\caption{A view of the effect of fiber volume fraction on Mode II ERR across different models.}\label{fig:volumefractionMII}
\end{figure}

\subsection{Interaction between debonds in UD laminates with a single layer of fibers}

\textcolor{blue}{We start with a simpler (1 parameter: number of fibers in the horizontal directions) but more extreme model: one line of fibers. What's the effect on $G_{I}$ and $G_{II}$? It increases them: a compliant element in the middle of two stiffer ones. Reference to Kies strain magnification.}

\begin{figure}[!h]
\centering
    \begin{subfigure}[b]{0.45\textwidth}
        %\includegraphics[width=\textwidth]{}
        \caption{$V_{f}=30\%$.}\label{subfig:sidefiber30MI}
    \end{subfigure} ~
    \begin{subfigure}[b]{0.45\textwidth}
        %\includegraphics[width=\textwidth]{}
        \caption{$V_{f}=50\%$.}\label{subfig:sidefiber50MI}
    \end{subfigure}

    \begin{subfigure}[b]{0.45\textwidth}
        %\includegraphics[width=\textwidth]{}
        \caption{$V_{f}=60\%$.}\label{subfig:sidefiber60MI}
    \end{subfigure} ~
    \begin{subfigure}[b]{0.45\textwidth}
        %\includegraphics[width=\textwidth]{}
        \caption{$V_{f}=65\%$.}\label{subfig:sidefiber65MI}
    \end{subfigure}

\caption{Effect of the interaction between debonds appearing at regular intervals on Mode I ERR in a single-ply laminate with a single layer of fibers at different levels of fiber volume fraction $V_{f}$.}\label{fig:sidefibersMI}
\end{figure}

\begin{figure}[!h]
\centering
    \begin{subfigure}[b]{0.45\textwidth}
        %\includegraphics[width=\textwidth]{}
        \caption{$V_{f}=30\%$.}\label{subfig:sidefiber30MII}
    \end{subfigure} ~
    \begin{subfigure}[b]{0.45\textwidth}
        %\includegraphics[width=\textwidth]{}
        \caption{$V_{f}=50\%$.}\label{subfig:sidefiber50MII}
    \end{subfigure}

    \begin{subfigure}[b]{0.45\textwidth}
        %\includegraphics[width=\textwidth]{}
        \caption{$V_{f}=60\%$.}\label{subfig:sidefiber60MII}
    \end{subfigure} ~
    \begin{subfigure}[b]{0.45\textwidth}
        %\includegraphics[width=\textwidth]{}
        \caption{$V_{f}=65\%$.}\label{subfig:sidefiber65MII}
    \end{subfigure}

\caption{Effect of the interaction between debonds appearing at regular intervals on Mode II ERR in a single-ply laminate with a single layer of fibers at different levels of fiber volume fraction $V_{f}$.}\label{fig:sidefibersMII}
\end{figure}

\subsection{Influence of layers of fully bonded fibers on debond's growth in a centrally located line of debonded fibers}

\textcolor{blue}{We then move to a ply with multiple lines of fibers and only debonded fibers in the central one (still only 1 parameter: number of fibers in vertical direction, but bit closer to real plies). No significant effect.}

\begin{figure}[!h]
\centering
    \begin{subfigure}[b]{0.45\textwidth}
        %\includegraphics[width=\textwidth]{}
        \caption{$V_{f}=30\%$.}\label{subfig:sidefiber30MI}
    \end{subfigure} ~
    \begin{subfigure}[b]{0.45\textwidth}
        %\includegraphics[width=\textwidth]{}
        \caption{$V_{f}=50\%$.}\label{subfig:sidefiber50MI}
    \end{subfigure}

    \begin{subfigure}[b]{0.45\textwidth}
        %\includegraphics[width=\textwidth]{}
        \caption{$V_{f}=60\%$.}\label{subfig:sidefiber60MI}
    \end{subfigure} ~
    \begin{subfigure}[b]{0.45\textwidth}
        %\includegraphics[width=\textwidth]{}
        \caption{$V_{f}=65\%$.}\label{subfig:sidefiber65MI}
    \end{subfigure}

\caption{Influence of layers of fully bonded fibers on debond's growth in Mode I ERR in a centrally located line of debonded fibers at different levels of fiber volume fraction $V_{f}$.}\label{fig:abovefibersMI}
\end{figure}

\begin{figure}[!h]
\centering
    \begin{subfigure}[b]{0.45\textwidth}
        %\includegraphics[width=\textwidth]{}
        \caption{$V_{f}=30\%$.}\label{subfig:sidefiber30MII}
    \end{subfigure} ~
    \begin{subfigure}[b]{0.45\textwidth}
        %\includegraphics[width=\textwidth]{}
        \caption{$V_{f}=50\%$.}\label{subfig:sidefiber50MII}
    \end{subfigure}

    \begin{subfigure}[b]{0.45\textwidth}
        %\includegraphics[width=\textwidth]{}
        \caption{$V_{f}=60\%$.}\label{subfig:sidefiber60MII}
    \end{subfigure} ~
    \begin{subfigure}[b]{0.45\textwidth}
        %\includegraphics[width=\textwidth]{}
        \caption{$V_{f}=65\%$.}\label{subfig:sidefiber65MII}
    \end{subfigure}

\caption{Influence of layers of fully bonded fibers on debond's growth in Mode II ERR in a centrally located line of debonded fibers at different levels of fiber volume fraction $V_{f}$.}\label{fig:abovefibersMII}
\end{figure}

\subsection{Interaction between debonds in UD laminates with multiple layers of fibers}

\textcolor{blue}{Finally models that are closer to real laminates and are more complex (2 parameters: number of fibers along the horizontal direction, number of layers in the vertical one).}

\begin{figure}[!h]
\centering
    \begin{subfigure}[b]{0.45\textwidth}
        %\includegraphics[width=\textwidth]{}
        \caption{$V_{f}=30\%$.}\label{subfig:abovefiber30MI}
    \end{subfigure} ~
    \begin{subfigure}[b]{0.45\textwidth}
        %\includegraphics[width=\textwidth]{}
        \caption{$V_{f}=50\%$.}\label{subfig:abovefiber50MI}
    \end{subfigure}

    \begin{subfigure}[b]{0.45\textwidth}
        %\includegraphics[width=\textwidth]{}
        \caption{$V_{f}=60\%$.}\label{subfig:abovefiber60MI}
    \end{subfigure} ~
    \begin{subfigure}[b]{0.45\textwidth}
        %\includegraphics[width=\textwidth]{}
        \caption{$V_{f}=65\%$.}\label{subfig:abovefiber65MI}
    \end{subfigure}

\caption{Effect of the interaction between debonds appearing at regular intervals on Mode I ERR in a single-ply laminate with multiple layers of fibers at different levels of fiber volume fraction $V_{f}$.}\label{fig:sideabovefibersMI}
\end{figure}

\begin{figure}[!h]
\centering
    \begin{subfigure}[b]{0.45\textwidth}
        %\includegraphics[width=\textwidth]{}
        \caption{$V_{f}=30\%$.}\label{subfig:sideabovefiber30MII}
    \end{subfigure} ~
    \begin{subfigure}[b]{0.45\textwidth}
        %\includegraphics[width=\textwidth]{}
        \caption{$V_{f}=50\%$.}\label{subfig:sideabovefiber50MII}
    \end{subfigure}

    \begin{subfigure}[b]{0.45\textwidth}
        %\includegraphics[width=\textwidth]{}
        \caption{$V_{f}=60\%$.}\label{subfig:sideabovefiber60MII}
    \end{subfigure} ~
    \begin{subfigure}[b]{0.45\textwidth}
        %\includegraphics[width=\textwidth]{}
        \caption{$V_{f}=65\%$.}\label{subfig:sideabovefiber65MII}
    \end{subfigure}

\caption{Effect of the interaction between debonds appearing at regular intervals on Mode II ERR in a single-ply laminate with multiple layers of fibers at different levels of fiber volume fraction $V_{f}$.}\label{fig:sideabovefibersMII}
\end{figure}

\subsection{Comparison with the single fiber model with equivalent boundary conditions}

\textcolor{blue}{We compare the previous results with the corresponding models of single fibers with equivalent BC. We draw conclusions on the possibility of using a single fiber with equivalent BCs. By remembering the actual ply configurations the repeating elements are modeling, and observing that in the vertical direction no significant effect related to the presence of debonded or bonded fiber can be found, we conclude that debonds appearing in fibers aligned in the vertical direction are energetically equivalent, and thus different configurations of debonded/bonded fibers along the vertical direction have the same probability. It is thus likely, from the energetic point of view, that debonds form at the same time along fibers aligned vertically.}

\begin{figure}[!h]
\centering
    \begin{subfigure}[b]{0.45\textwidth}
        %\includegraphics[width=\textwidth]{}
        \caption{$V_{f}=30\%$.}\label{subfig:comparisonfree30MI}
    \end{subfigure} ~
    \begin{subfigure}[b]{0.45\textwidth}
        %\includegraphics[width=\textwidth]{}
        \caption{$V_{f}=50\%$.}\label{subfig:comparisonfree50MI}
    \end{subfigure}

    \begin{subfigure}[b]{0.45\textwidth}
        %\includegraphics[width=\textwidth]{}
        \caption{$V_{f}=60\%$.}\label{subfig:comparisonfree60MI}
    \end{subfigure} ~
    \begin{subfigure}[b]{0.45\textwidth}
        %\includegraphics[width=\textwidth]{}
        \caption{$V_{f}=65\%$.}\label{subfig:comparisonfree65MI}
    \end{subfigure}

\caption{Comparison of Mode I ERR between the single fiber model with free upper boundary and the multiple fibers model with fibers only on the side at different levels of fiber volume fraction $V_{f}$.}\label{fig:comparisonfreeMI}
\end{figure}

\begin{figure}[!h]
\centering
    \begin{subfigure}[b]{0.45\textwidth}
        %\includegraphics[width=\textwidth]{}
        \caption{$V_{f}=30\%$.}\label{subfig:comparisonfree30MII}
    \end{subfigure} ~
    \begin{subfigure}[b]{0.45\textwidth}
        %\includegraphics[width=\textwidth]{}
        \caption{$V_{f}=50\%$.}\label{subfig:comparisonfree50MII}
    \end{subfigure}

    \begin{subfigure}[b]{0.45\textwidth}
        %\includegraphics[width=\textwidth]{}
        \caption{$V_{f}=60\%$.}\label{subfig:comparisonfree60MII}
    \end{subfigure} ~
    \begin{subfigure}[b]{0.45\textwidth}
        %\includegraphics[width=\textwidth]{}
        \caption{$V_{f}=65\%$.}\label{subfig:comparisonfree65MII}
    \end{subfigure}

\caption{Comparison of Mode II ERR between the single fiber model with free upper boundary and the multiple fibers model with fibers only on the side at different levels of fiber volume fraction $V_{f}$.}\label{fig:comparisonfreeMII}
\end{figure}

\begin{figure}[!h]
\centering
    \begin{subfigure}[b]{0.45\textwidth}
        %\includegraphics[width=\textwidth]{}
        \caption{$V_{f}=30\%$.}\label{subfig:comparisoncoupling30MI}
    \end{subfigure} ~
    \begin{subfigure}[b]{0.45\textwidth}
        %\includegraphics[width=\textwidth]{}
        \caption{$V_{f}=50\%$.}\label{subfig:comparisoncoupling50MI}
    \end{subfigure}

    \begin{subfigure}[b]{0.45\textwidth}
        %\includegraphics[width=\textwidth]{}
        \caption{$V_{f}=60\%$.}\label{subfig:comparisoncoupling60MI}
    \end{subfigure} ~
    \begin{subfigure}[b]{0.45\textwidth}
        %\includegraphics[width=\textwidth]{}
        \caption{$V_{f}=65\%$.}\label{subfig:comparisoncoupling65MI}
    \end{subfigure}

\caption{Comparison of Mode I ERR between the single fiber model with coupling conditions along the upper boundary and the multiple fibers model with fibers above and both above and on the side at different levels of fiber volume fraction $V_{f}$.}\label{fig:comparisoncouplingMI}
\end{figure}

\begin{figure}[!h]
\centering
    \begin{subfigure}[b]{0.45\textwidth}
        %\includegraphics[width=\textwidth]{}
        \caption{$V_{f}=30\%$.}\label{subfig:comparisoncoupling30MII}
    \end{subfigure} ~
    \begin{subfigure}[b]{0.45\textwidth}
        %\includegraphics[width=\textwidth]{}
        \caption{$V_{f}=50\%$.}\label{subfig:comparisoncoupling50MII}
    \end{subfigure}

    \begin{subfigure}[b]{0.45\textwidth}
        %\includegraphics[width=\textwidth]{}
        \caption{$V_{f}=60\%$.}\label{subfig:comparisoncoupling60MII}
    \end{subfigure} ~
    \begin{subfigure}[b]{0.45\textwidth}
        %\includegraphics[width=\textwidth]{}
        \caption{$V_{f}=65\%$.}\label{subfig:comparisoncoupling65MII}
    \end{subfigure}

\caption{Comparison of Mode II ERR between the single fiber model with coupling conditions along the upper boundary and the multiple fibers model with fibers above and both above and on the side at different levels of fiber volume fraction $V_{f}$.}\label{fig:comparisoncouplingMII}
\end{figure}

\section{Conclusions \& Outlook}

\end{document}
