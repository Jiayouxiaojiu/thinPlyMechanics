\documentclass[review]{elsarticle}

\usepackage{amsmath}
\usepackage{subcaption}
\usepackage{lineno,hyperref}
\modulolinenumbers[5]

\journal{TBA}

%%%%%%%%%%%%%%%%%%%%%%%
%% Elsevier bibliography styles
%%%%%%%%%%%%%%%%%%%%%%%
%% To change the style, put a % in front of the second line of the current style and
%% remove the % from the second line of the style you would like to use.
%%%%%%%%%%%%%%%%%%%%%%%

%% Numbered
%\bibliographystyle{model1-num-names}

%% Numbered without titles
%\bibliographystyle{model1a-num-names}

%% Harvard
%\bibliographystyle{model2-names.bst}\biboptions{authoryear}

%% Vancouver numbered
%\usepackage{numcompress}\bibliographystyle{model3-num-names}

%% Vancouver name/year
%\usepackage{numcompress}\bibliographystyle{model4-names}\biboptions{authoryear}

%% APA style
%\bibliographystyle{model5-names}\biboptions{authoryear}

%% AMA style
%\usepackage{numcompress}\bibliographystyle{model6-num-names}

%% `Elsevier LaTeX' style
\bibliographystyle{elsarticle-num}
%%%%%%%%%%%%%%%%%%%%%%%

\begin{document}

\begin{frontmatter}

\title{Effect of uniform distributions of bonded and debonded fibers on the growth of the fiber/matrix interface crack in UD laminates with different fiber contents under transverse loading}
%\tnotetext[mytitlenote]{Fully documented templates are available in the elsarticle package on \href{http://www.ctan.org/tex-archive/macros/latex/contrib/elsarticle}{CTAN}.}

%% Group authors per affiliation:
%\author{Luca Di Stasio\fnref{myfootnote}}
%\address{Radarweg 29, Amsterdam}
%\fntext[myfootnote]{Since 1880.}

%% or include affiliations in footnotes:
\author[nancy,lulea]{Luca Di Stasio}
\author[lulea]{Janis Varna}
\author[nancy]{Zoubir Ayadi}
%\ead[url]{www.elsevier.com}

%\author[mysecondaryaddress]{Global Customer Service\corref{mycorrespondingauthor}}
%\cortext[mycorrespondingauthor]{Corresponding author}
%\ead{support@elsevier.com}

\address[nancy]{Universit\'e de Lorraine, EEIGM, IJL, 6 Rue Bastien Lepage, F-54010 Nancy, France}
\address[lulea]{Lule\aa\ University of Technology, University Campus, SE-97187 Lule\aa, Sweden}

\begin{abstract}

\end{abstract}

%\begin{keyword}
%\texttt{elsarticle.cls}\sep \LaTeX\sep Elsevier \sep template
%\MSC[2010] 00-01\sep  99-00
%\end{keyword}

\end{frontmatter}

\linenumbers

\section{Introduction}

\section{RVE models \& FE discretization}

\subsection{Models of Representative Volume Element(RVE)}

\begin{figure}[!h]
\centering
    \begin{subfigure}[b]{0.45\textwidth}
        %\includegraphics[width=\textwidth]{}
        \caption{A debonded fiber every 2 fully bonded ones.}\label{subfig:every2}
    \end{subfigure} ~
    \begin{subfigure}[b]{0.45\textwidth}
        %\includegraphics[width=\textwidth]{}
        \caption{Central debonded fiber with 1 fiber each side.}\label{subfig:1eachside}
    \end{subfigure}

    \begin{subfigure}[b]{0.45\textwidth}
        %\includegraphics[width=\textwidth]{}
        \caption{A debonded fiber every 4 fully bonded ones.}\label{subfig:every4}
    \end{subfigure} ~
    \begin{subfigure}[b]{0.45\textwidth}
        %\includegraphics[width=\textwidth]{}
        \caption{Central debonded fiber with 2 fibers each side.}\label{subfig:2eachside}
    \end{subfigure}

    \begin{subfigure}[b]{0.45\textwidth}
        %\includegraphics[width=\textwidth]{}
        \caption{A debonded fiber every 6 fully bonded ones.}\label{subfig:every6}
    \end{subfigure} ~
    \begin{subfigure}[b]{0.45\textwidth}
        %\includegraphics[width=\textwidth]{}
        \caption{Central debonded fiber with 3 fibers each side.}\label{subfig:3eachside}
    \end{subfigure}
\caption{Models of single-ply laminates with a single layer of fibers and debonds repeating at different distances (left column), and corresponding Representative Volume Elements (right column) with symmetry applied on the lower boundary line. The interface crack is represented in red.}\label{fig:fibersOnSideModels}
\end{figure}

\begin{figure}[!h]
\centering
    \begin{subfigure}[b]{0.45\textwidth}
        %\includegraphics[width=\textwidth]{}
        \caption{3 layers with a central line of debonded fibers.}\label{subfig:3layers}
    \end{subfigure} ~
    \begin{subfigure}[b]{0.45\textwidth}
        %\includegraphics[width=\textwidth]{}
        \caption{Central debonded fiber with 1 fiber above.}\label{subfig:1above}
    \end{subfigure}

    \begin{subfigure}[b]{0.45\textwidth}
        %\includegraphics[width=\textwidth]{}
        \caption{5 layers with a central line of debonded fibers.}\label{subfig:5layers}
    \end{subfigure} ~
    \begin{subfigure}[b]{0.45\textwidth}
        %\includegraphics[width=\textwidth]{}
        \caption{Central debonded fiber with 2 fibers above.}\label{subfig:1above}
    \end{subfigure}

    \begin{subfigure}[b]{0.45\textwidth}
        %\includegraphics[width=\textwidth]{}
        \caption{7 layers with a central line of debonded fibers.}\label{subfig:7layers}
    \end{subfigure} ~
    \begin{subfigure}[b]{0.45\textwidth}
        %\includegraphics[width=\textwidth]{}
        \caption{Central debonded fiber with 3 fibers above.}\label{subfig:1above}
    \end{subfigure}
\caption{Models of single-ply laminates with multiple layers of fibers and a central line of debonded fibers (left column), and corresponding Representative Volume Elements (right column) with symmetry applied on the lower boundary line. The interface crack is represented in red.}\label{fig:fibersOnTopModels}
\end{figure}

\begin{figure}[!h]
\centering
    \begin{subfigure}[b]{0.45\textwidth}
        %\includegraphics[width=\textwidth]{}
        \caption{3 layers with a debonded fiber every 2 fully bonded ones in the central line of fibers.}\label{subfig:3layersevery2}
    \end{subfigure} ~
    \begin{subfigure}[b]{0.45\textwidth}
        %\includegraphics[width=\textwidth]{}
        \caption{Central debonded fiber with 1 fiber on each side and 1 above.}\label{subfig:1eachside1above}
    \end{subfigure}

    \begin{subfigure}[b]{0.45\textwidth}
        %\includegraphics[width=\textwidth]{}
        \caption{3 layers with a debonded fiber every 4 fully bonded ones in the central line of fibers.}\label{subfig:3layersevery4}
    \end{subfigure} ~
    \begin{subfigure}[b]{0.45\textwidth}
        %\includegraphics[width=\textwidth]{}
        \caption{Central debonded fiber with 2 fibers on each side and 1 above.}\label{subfig:2eachside1above}
    \end{subfigure}

    \begin{subfigure}[b]{0.45\textwidth}
        %\includegraphics[width=\textwidth]{}
        \caption{5 layers with a debonded fiber every 4 fully bonded ones in the central line of fibers.}\label{subfig:5layersevery4}
    \end{subfigure} ~
    \begin{subfigure}[b]{0.45\textwidth}
        %\includegraphics[width=\textwidth]{}
        \caption{Central debonded fiber with 2 fibers on each side and 2 above.}\label{subfig:2eachside2above}
    \end{subfigure}

    \begin{subfigure}[b]{0.45\textwidth}
        %\includegraphics[width=\textwidth]{}
        \caption{3 layers with a debonded fiber every 6 fully bonded ones in the central line of fibers.}\label{subfig:3layersevery6}
    \end{subfigure} ~
    \begin{subfigure}[b]{0.45\textwidth}
        %\includegraphics[width=\textwidth]{}
        \caption{Central debonded fiber with 3 fibers on each side and 1 above.}\label{subfig:3eachside1above}
    \end{subfigure}
\caption{Models of single-ply laminates with multiple layers of fibers with debonds repeating at different distances in the central line of fibers (left column), and corresponding Representative Volume Elements (right column) with symmetry applied on the lower boundary line.}\label{fig:fibersOnSideAndTopModels}
\end{figure}

\subsection{Finite Element (FE) discretization}

\subsection{Validation of the model}

\section{Results \& Discussion}

\subsection{Effect of Fiber Volume Fraction}

\subsection{Interaction between debonds in UD laminates with a single layer of fibers}

\subsection{Influence of layers of fully bonded fibers on debond's growth in a centrally located line of debonded fibers}

\subsection{Interaction between debonds in UD laminates with multiple layers of fibers}

\subsection{Comparison with the single fiber model with equivalent boundary conditions}

\section{Conclusions \& Outlook}

\end{document}
