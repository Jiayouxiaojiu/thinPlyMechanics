\begin{longtable}{XXXXp{0.3\textwidth}p{0.3\textwidth}}
\bf{N.}&\bf{Status}&\bf{Reviewer}&\bf{Position}&\bf{Observation(s)}&\bf{Action(s)}\\
&\textcolor{red}{\xmark}\textrightarrow To-Do&&&&\\
&\textcolor{orange}{$\blacksquare$}\textrightarrow In Progress&&&&\\
&\textcolor{green}{\cmark}\textrightarrow Reviewed&&&&\\
\toprule
\bottomrule
1&\textcolor{green}{\cmark}&1&Page 8, lines 2--17&It is not clear what the word ``coupling'' really means within the context of the application of the boundary displacement.  For example, does the ``application of the coupling of horizontal displacements $u_{x}$ along the left and right had sides….'' Means simultaneous application of the horizontal displacement ux along the left and right hand sides…?
Similarly, what does the statement ``coupling of the vertical displacements $u_{z}$ is applied to the upper boundary…'' really mean?  What were the imposed boundary conditions for this particular loading state? Was the horizontal displacement applied simultaneously as the vertical displacement?
Linear distribution of horizontal displacement - what was the linear function for the displacement distribution used, as the choice of this will have a significant influence on the strain energy release rate.&The meaning of the different boundary conditions has been clarified by expressing them in the form of equations and improving the description of each one.\\
\midrule
2&\textcolor{green}{\cmark}&1&Pages 9, 16, and 19&Include the section numbers.&Section numbers were missing due to a misuse of the Latex journal template (sections are unnumbered). Corrected by referring to titles.\\
\midrule
3&\textcolor{green}{\cmark}&1&Page 9, lines 35-36&The applied horizontal displacement is chosen to correspond to a horizontal strain $\varepsilon_{x} = 1\%$.  However, there is no information about the applied vertical displacement $u_{z}$ for the other load cases.&The meaning of this constant vertical displacement and how it is evaluated are now explained in sub-section \textit{Introduction \& nomenclature} of section \textit{RVE models \& FE discretization}, when the different sets of boundary conditions are explained.\\
\midrule
4&\textcolor{green}{\cmark}&1&Figure 4&The results for $G_{I}$ for the load case $1 \times 1 - coupling$ suggest $G_{I}$ zero when $\Delta\theta > 80^{\circ}$. This is very surprising. The load case $1 \times 1 - coupling$ involves the application of vertical displacement $u_{z}$.  Thus, with increasing value of  $\Delta\theta$, $u_{z}$ will tend to have more opening mode effect on the debond. Thus one would expect $G_{I}$ to remain finite and positive, and to increase as $\Delta\theta$  increases beyond $45$ degrees. Can the authors comment on this and the difference between this expectation and their results?&An explanation of the reason why $G_{I}$ is equal to $0$ for $\Delta\theta\geq0$ in the $1 \times 1 - coupling$ case has been added to sub-section \textit{Effect of the proximity of the $0^{\circ}/90^{\circ}$ interface and of the thickness of $0^{\circ}$ layer on debond ERR for highly interactive debonds} of section \textit{Results \& Discussion}. For the sake of completeness, the explanation is reported here as well. \textit{``Notice that in the case of $1\times 1-coupling$, the upper surface moves by an amount of $u_{z}^{\nu}$ in the vertical direction, due to Poisson's effect, while remaining straight (see Equation~7). The value of $u_{z}^{\nu}$ is evaluated as part of the elastic solution and it results to be always negative. This agrees well with the expectation that, upon application of a tensile load in the $x$-direction, Poisson's effect causes a contraction of the cross-section normal to the $x$-axis, which translates into a negative $z$-displacement in the $x-z$ plane. In turn, the fiber-matrix interface is subjected to a compressive stress state for $\Delta\theta>80^{\circ}-90^{\circ}$, preventing debond growth in Mode I and corresponding in Figure~4 to the fact that $G_{I}=0$ for $\Delta\theta\geq80^{\circ}$.''}\\
\midrule
5&\textcolor{red}{\xmark}&1&Page 16&the difference between the energy release rate (ERR) for $n = 1$ and $n = 21$ (Figures 4 and 6; and Figures 5 and 7) was attributed to “strain magnification”.  As stated in the manuscript, the case with $n = 21$ is much stiffer than that with $n = 1$. The application of the same remote strain of $\varepsilon_{x} = 1\%$ to both cases, means the net average applied remote stress would be higher for $n = 21$ than for $n = 1$.  It is well known that ERR scale with the remotely applied load or stress. Should the comparison between the two cases not be made at the same average remote applied stress (not strain)? &\\
\midrule
\midrule
1&\textcolor{red}{\xmark}&2&Introduction and Conclusions&However, the authors could remark more clearly the novelty of their results versus the previous referenced ones.&\textit{In Introduction:}\\
2&\textcolor{green}{\cmark}&2&Page 2. Introduction: lines 7, 11, 22, 27 and 35.&Erroneous ‘?’ seem to appear.&The question marks appear in an incomplete Latex compilation in the place of references, figures, tables and sections' numbers. They were likely caused by an error of the journal's Latex compiler as they were absent in the local version. If they appear again, check the pdf provided by the authors.\\
3&\textcolor{green}{\cmark}&2&Page 2. Introduction: line 9.&‘At the lamina level, the use…’ should be ‘At the lamina level the use…’&Corrected according to reviewer's suggestion.\\
4&\textcolor{green}{\cmark}&2&Page 3. Introduction: lines 6, 17, 18, 19, 38 and 40.&Erroneous ‘?’ seem to appear.&The question marks appear in an incomplete Latex compilation in the place of references, figures, tables and sections' numbers. They were likely caused by an error of the journal's Latex compiler as they were absent in the local version. If they appear again, check the pdf provided by the authors.\\
5&\textcolor{green}{\cmark}&2&Page 4. Introduction: lines 5, 9, 17, 18, 20, and 38.&Erroneous ‘?’ seem to appear.&The question marks appear in an incomplete Latex compilation in the place of references, figures, tables and sections' numbers. They were likely caused by an error of the journal's Latex compiler as they were absent in the local version. If they appear again, check the pdf provided by the authors.\\
6&\textcolor{green}{\cmark}&2&Page 6. &Equation (2): An extra dot at the end seems to appear.&Removed.\\
7&\textcolor{green}{\cmark}&2&Page 7. Line 21.&The following modification is proposed: ‘ the laminates are assumed to be SUBJECTED..’ &Modified according to reviewer's suggestion.\\
8&\textcolor{red}{\xmark}&2&RVE models \& FE discretization: Figures 1 and 2.&Discontinuous vertical lines on the outer boundary should be desirable in order to reinforce the unlimited length of the model in the longitudinal (x) direction.&\\
9&\textcolor{red}{\xmark}&2&RVE models \& FE discretization: Figures 1 and 2.&How is real is to find horizontally aligned debonds in an actual cross ply composite (specially in the several rows model)? Is there any experimental base for this damage pattern? It could seem more realistic to consider vertically quasi-aligned debonds. Could the author envisage if the conclusions would be the same in this vertical pattern case?&\\
10&\textcolor{red}{\xmark}&2&Page 7. Line 44.&RUC nomenclature $n\times k-m\cdot t_{90^{\circ}}$ seems a bit complicated specially when used in sections where different combinations are continuously referred (see for instance pages 13 and 14).&\\
11&\textcolor{red}{\xmark}&2&Pages 8. Line 31.&$n=1$ does not seem an actual case from transverse cracking point of view. Additional clarification for the use of this model is recommended. &\\
12&\textcolor{red}{\xmark}&2&Pages 9. Line 23.&Numbers of sections are missed. &\\
13&\textcolor{red}{\xmark}&2&Finite Element (FE) discretization.&Since the properties employed are realistic and the results presented are not dimensionless it does not seem too logical to employ not real dimensions for the models (see for instance $R_{f}=1\mu m$ or cell sizes). The complete use of realistic parameters could provide additional conclusions for the $G$ results presented.&\\
14&\textcolor{green}{\cmark}&2&Page 10. Line 41.&Should CUSTOM be CUSTOMER?&To avoid confusion, \textit{``in a custom Python routine''} has been replaced by \textit{``in a Python routine developed by one of the authors''}.\\
15&\textcolor{green}{\cmark}&2&Page 11. Line 46.&Should ALBEIT be ALTHOUGH?&Yes, indeed: \textit{``albeit''} has been replaced with \textit{``although''}.\\
16&\textcolor{red}{\xmark}&2&RESULTS and Discussion Section.&In addition to $G_{I}$ and $G_{II}$, total $G$ graphs seem necessary to compute global effects in all cases presented.&\\
17&\textcolor{green}{\cmark}&2&Pages 16. Line 22.&Number of section is missed.&Section numbers were missing due to a misuse of the Latex journal template (sections are unnumbered). Corrected by referring to titles.\\
18&\textcolor{red}{\xmark}&2&Pages 18. Lines 40 and 41.&Number of section is missed.&\\
\midrule
\end{longtable}