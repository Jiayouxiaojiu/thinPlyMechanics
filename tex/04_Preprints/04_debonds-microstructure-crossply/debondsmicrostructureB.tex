\documentclass[review]{elsarticle}

\usepackage{amsmath}
\usepackage{dirtytalk}
\usepackage{subcaption}
\usepackage[usenames]{xcolor}
\usepackage{lineno,hyperref}
\modulolinenumbers[5]

\journal{TBA}

%%%%%%%%%%%%%%%%%%%%%%%
%% Elsevier bibliography styles
%%%%%%%%%%%%%%%%%%%%%%%
%% To change the style, put a % in front of the second line of the current style and
%% remove the % from the second line of the style you would like to use.
%%%%%%%%%%%%%%%%%%%%%%%

%% Numbered
%\bibliographystyle{model1-num-names}

%% Numbered without titles
%\bibliographystyle{model1a-num-names}

%% Harvard
%\bibliographystyle{model2-names.bst}\biboptions{authoryear}

%% Vancouver numbered
%\usepackage{numcompress}\bibliographystyle{model3-num-names}

%% Vancouver name/year
%\usepackage{numcompress}\bibliographystyle{model4-names}\biboptions{authoryear}

%% APA style
%\bibliographystyle{model5-names}\biboptions{authoryear}

%% AMA style
%\usepackage{numcompress}\bibliographystyle{model6-num-names}

%% `Elsevier LaTeX' style
\bibliographystyle{elsarticle-num}
%%%%%%%%%%%%%%%%%%%%%%%

\begin{document}

\begin{frontmatter}

\title{Energy release rate of the fiber/matrix interface crack in cross-ply $\left[0_{2kn}^{\circ},90_{n}^{\circ}\right]_{S}$ laminates under transverse loading: debond-bimaterial interface interaction}
%\tnotetext[mytitlenote]{Fully documented templates are available in the elsarticle package on \href{http://www.ctan.org/tex-archive/macros/latex/contrib/elsarticle}{CTAN}.}

%% Group authors per affiliation:
%\author{Luca Di Stasio\fnref{myfootnote}}
%\address{Radarweg 29, Amsterdam}
%\fntext[myfootnote]{Since 1880.}

%% or include affiliations in footnotes:
\author[nancy,lulea]{Luca Di Stasio}
\author[lulea]{Janis Varna}
\author[nancy]{Zoubir Ayadi}
%\ead[url]{www.elsevier.com}

%\author[mysecondaryaddress]{Global Customer Service\corref{mycorrespondingauthor}}
%\cortext[mycorrespondingauthor]{Corresponding author}
%\ead{support@elsevier.com}

\address[nancy]{Universit\'e de Lorraine, EEIGM, IJL, 6 Rue Bastien Lepage, F-54010 Nancy, France}
\address[lulea]{Lule\aa\ University of Technology, University Campus, SE-97187 Lule\aa, Sweden}

\begin{abstract}
\noindent
%\textcolor{purple}{{\em Priority}: 1}\\
%\textcolor{purple}{{\em Target journal(s)}: Composites Part B: Engineering, Composites Part A: Applied Science and Manufacturing, Composite Structures, Journal of Composite Materials, Composite Communications}\\
The effects of crack shielding, fiber content and ratio of $0^{\circ}$ to $90^{\circ}$ ply thickness on fiber/matrix debond growth in thin cross-ply laminates are investigated with Representative Volume Elements (RVEs) of different ordered microstructures. Debond growth is characterized by the estimation of the Energy Release Rates (ERRs) using the Virtual Crack Closure Technique (VCCT) and the J-integral. It is found that 
\end{abstract}

\begin{keyword}
Polymer-matrix Composites (PMCs)\sep Thin-ply\sep Transverse Failure \sep Debonding \sep Finite Element Analysis (FEA)
\end{keyword}


\end{frontmatter}

\linenumbers

\section{Introduction}

Since the development of the \emph{spred tow} technology or \say{FUKUI method} (from the Japanese prefecture that hosted its invention)~\cite{Kawabe2008,Kawabe2008en}, significant efforts have been directed toward the characterization of \emph{thin-ply} laminates~\cite{Sasayama2003,Yamaguchi2005,Tsai2005,Sihn2007,Yokozeki2008,Yokozeki2010,Saito2012,Arteiro2013,Arteiro2014,Amacher2014,Guillamet2014,Huang2018,Cugnoni2018} and their application to mission-critical structures in the aerospace sector~\cite{Moon2011,Kim2017,Kopp2017,McCarville2018}.\\


\section{RVE models \& FE discretization}

\subsection{Models of Representative Volume Element(RVE)}

\textcolor{blue}{We start by describing the different idealized micro-structures considered and the corresponding repeating element or RVE used to model them. Fig.~\ref{fig:fibersOnSideModels}, Fig.~\ref{fig:fibersOnTopModels} and Fig.~\ref{fig:fibersOnSideAndTopModels}}

\begin{figure}[!h]
\centering
    \begin{subfigure}[b]{0.45\textwidth}
        %\includegraphics[width=\textwidth]{}
        \caption{A debonded fiber every 2 fully bonded ones.}\label{subfig:every2}
    \end{subfigure} ~
    \begin{subfigure}[b]{0.45\textwidth}
        %\includegraphics[width=\textwidth]{}
        \caption{Central debonded fiber with 1 fiber each side.}\label{subfig:1eachside}
    \end{subfigure}

    \begin{subfigure}[b]{0.45\textwidth}
        %\includegraphics[width=\textwidth]{}
        \caption{A debonded fiber every 4 fully bonded ones.}\label{subfig:every4}
    \end{subfigure} ~
    \begin{subfigure}[b]{0.45\textwidth}
        %\includegraphics[width=\textwidth]{}
        \caption{Central debonded fiber with 2 fibers each side.}\label{subfig:2eachside}
    \end{subfigure}

    \begin{subfigure}[b]{0.45\textwidth}
        %\includegraphics[width=\textwidth]{}
        \caption{A debonded fiber every 6 fully bonded ones.}\label{subfig:every6}
    \end{subfigure} ~
    \begin{subfigure}[b]{0.45\textwidth}
        %\includegraphics[width=\textwidth]{}
        \caption{Central debonded fiber with 3 fibers each side.}\label{subfig:3eachside}
    \end{subfigure}
\caption{Models of $\left[0^{\circ}_{n}, 90^{\circ}\right]_{S}$ laminates in which the central $90^{\circ}$ ply possesses a single layer of fibers and debonds repeating at different distances (left column), and corresponding Representative Volume Elements (right column) with symmetry applied on the lower boundary line. The interface crack is represented in red.}\label{fig:fibersOnSideModels}
\end{figure}

\begin{figure}[!h]
\centering
    \begin{subfigure}[b]{0.45\textwidth}
        %\includegraphics[width=\textwidth]{}
        \caption{3 layers with a central line of debonded fibers.}\label{subfig:3layers}
    \end{subfigure} ~
    \begin{subfigure}[b]{0.45\textwidth}
        %\includegraphics[width=\textwidth]{}
        \caption{Central debonded fiber with 1 fiber above.}\label{subfig:1above}
    \end{subfigure}

    \begin{subfigure}[b]{0.45\textwidth}
        %\includegraphics[width=\textwidth]{}
        \caption{5 layers with a central line of debonded fibers.}\label{subfig:5layers}
    \end{subfigure} ~
    \begin{subfigure}[b]{0.45\textwidth}
        %\includegraphics[width=\textwidth]{}
        \caption{Central debonded fiber with 2 fibers above.}\label{subfig:1above}
    \end{subfigure}

    \begin{subfigure}[b]{0.45\textwidth}
        %\includegraphics[width=\textwidth]{}
        \caption{7 layers with a central line of debonded fibers.}\label{subfig:7layers}
    \end{subfigure} ~
    \begin{subfigure}[b]{0.45\textwidth}
        %\includegraphics[width=\textwidth]{}
        \caption{Central debonded fiber with 3 fibers above.}\label{subfig:1above}
    \end{subfigure}
\caption{Models of $\left[0^{\circ}_{n}, 90^{\circ}\right]_{S}$ laminates in which the central $90^{\circ}$ ply possesses a central line of debonded fibers (left column), and corresponding Representative Volume Elements (right column) with symmetry applied on the lower boundary line. The interface crack is represented in red.}\label{fig:fibersOnTopModels}
\end{figure}

\begin{figure}[!h]
\centering
    \begin{subfigure}[b]{0.45\textwidth}
        %\includegraphics[width=\textwidth]{}
        \caption{3 layers with a debonded fiber every 2 fully bonded ones in the central line of fibers.}\label{subfig:3layersevery2}
    \end{subfigure} ~
    \begin{subfigure}[b]{0.45\textwidth}
        %\includegraphics[width=\textwidth]{}
        \caption{Central debonded fiber with 1 fiber on each side and 1 above.}\label{subfig:1eachside1above}
    \end{subfigure}

    \begin{subfigure}[b]{0.45\textwidth}
        %\includegraphics[width=\textwidth]{}
        \caption{3 layers with a debonded fiber every 4 fully bonded ones in the central line of fibers.}\label{subfig:3layersevery4}
    \end{subfigure} ~
    \begin{subfigure}[b]{0.45\textwidth}
        %\includegraphics[width=\textwidth]{}
        \caption{Central debonded fiber with 2 fibers on each side and 1 above.}\label{subfig:2eachside1above}
    \end{subfigure}

    \begin{subfigure}[b]{0.45\textwidth}
        %\includegraphics[width=\textwidth]{}
        \caption{5 layers with a debonded fiber every 4 fully bonded ones in the central line of fibers.}\label{subfig:5layersevery4}
    \end{subfigure} ~
    \begin{subfigure}[b]{0.45\textwidth}
        %\includegraphics[width=\textwidth]{}
        \caption{Central debonded fiber with 2 fibers on each side and 2 above.}\label{subfig:2eachside2above}
    \end{subfigure}

    \begin{subfigure}[b]{0.45\textwidth}
        %\includegraphics[width=\textwidth]{}
        \caption{3 layers with a debonded fiber every 6 fully bonded ones in the central line of fibers.}\label{subfig:3layersevery6}
    \end{subfigure} ~
    \begin{subfigure}[b]{0.45\textwidth}
        %\includegraphics[width=\textwidth]{}
        \caption{Central debonded fiber with 3 fibers on each side and 1 above.}\label{subfig:3eachside1above}
    \end{subfigure}
\caption{Models of$\left[0^{\circ}_{n}, 90^{\circ}\right]_{S}$ laminates in which the central $90^{\circ}$ ply possesses multiple layers of fibers with debonds repeating at different distances in the central line of fibers (left column), and corresponding Representative Volume Elements (right column) with symmetry applied on the lower boundary line.}\label{fig:fibersOnSideAndTopModels}
\end{figure}

\begin{figure}[!h]
\centering
    \begin{subfigure}[b]{0.45\textwidth}
        %\includegraphics[width=\textwidth]{}
        \caption{Single layer of debonded fibers inside a cross-ply laminates.}\label{subfig:singlelayerdebfibers}
    \end{subfigure} ~
    \begin{subfigure}[b]{0.45\textwidth}
        %\includegraphics[width=\textwidth]{}
        \caption{Element with a single debonded fiber and, on the top surface, coupled vertical displacement and linearly distributed horizontal displacement.}\label{subfig:free}
    \end{subfigure}

\caption{Models of $\left[0^{\circ}_{n}, 90^{\circ}\right]_{S}$ laminates in which fibers belonging to the central $90^{\circ}$ ply are all debonded  (left column), and corresponding Representative Volume Elements (right column) with symmetry applied on the lower boundary line.}\label{fig:allFibersDebonded}
\end{figure}

\subsection{Finite Element (FE) discretization}

\textcolor{blue}{We describe the model implemented: schematic + description of parameters, formulation (LEFM, frictionless contact, VCCT, J-Integral), implementation of BCs, mesh. Fig.~\ref{fig:FEmodel}}

\begin{figure}[!h]
\centering
    \begin{subfigure}[b]{0.45\textwidth}
        %\includegraphics[width=\textwidth]{}
        \caption{Schematic of the model with its main parameters.}\label{subfig:modelschem}
    \end{subfigure} ~
    \begin{subfigure}[b]{0.45\textwidth}
        %\includegraphics[width=\textwidth]{}
        \caption{Detail of the mesh in the crack tip's neighborhood.}\label{subfig:meshdetail}
    \end{subfigure}

\caption{Details and main parameters of the Finite Element model.}\label{fig:FEmodel}
\end{figure}

\textcolor{blue}{We mention the validation of the model with respect to BEM results by referring to the other paper.}

\section{Results \& Discussion}

\subsection{Effect of Fiber Volume Fraction}

\textcolor{blue}{The effect is similar for all the different BC cases, it's enough to show some of them to exemplify. $G_{I}$ in Fig.~\ref{fig:volumefractionMI}, $G_{II}$ in Fig.~\ref{fig:volumefractionMII}.}

\textcolor{purple}{Graphics of ERR vs $\Delta\theta$, one curve for each $V_{f}$, one graphic for each selected BC. Selected BC: free, coupling, some examples with fibers (see captions).}

\begin{figure}[!h]
\centering
    \begin{subfigure}[b]{0.45\textwidth}
        %\includegraphics[width=\textwidth]{}
        \caption{Single fiber model with free boundary on top.}\label{subfig:volfracfreeMI}
    \end{subfigure} ~
    \begin{subfigure}[b]{0.45\textwidth}
        %\includegraphics[width=\textwidth]{}
        \caption{Single fiber model with coupling of vertical displacements along the upper boundary.}\label{subfig:volfraccouplingMI}
    \end{subfigure}

    \begin{subfigure}[b]{0.45\textwidth}
        %\includegraphics[width=\textwidth]{}
        \caption{1 fiber each side.}\label{subfig:volfrac1eachsideMI}
    \end{subfigure} ~
    \begin{subfigure}[b]{0.45\textwidth}
        %\includegraphics[width=\textwidth]{}
        \caption{1 fiber above.}\label{subfig:volfrac1aboveMI}
    \end{subfigure}

    \begin{subfigure}[b]{0.45\textwidth}
        %\includegraphics[width=\textwidth]{}
        \caption{5 fibers each side.}\label{subfig:volfrac5eachsideMI}
    \end{subfigure} ~
    \begin{subfigure}[b]{0.45\textwidth}
        %\includegraphics[width=\textwidth]{}
        \caption{5 fibers above.}\label{subfig:volfrac5aboveMI}
    \end{subfigure}

    \begin{subfigure}[b]{0.45\textwidth}
        %\includegraphics[width=\textwidth]{}
        \caption{10 fibers each side.}\label{subfig:volfrac10eachsideMI}
    \end{subfigure} ~
    \begin{subfigure}[b]{0.45\textwidth}
        %\includegraphics[width=\textwidth]{}
        \caption{10 fibers above.}\label{subfig:volfrac10aboveMI}
    \end{subfigure}

    \begin{subfigure}[b]{0.45\textwidth}
        %\includegraphics[width=\textwidth]{}
        \caption{1 fiber each side, 1 above.}\label{subfig:volfrac1eachside1aboveMI}
    \end{subfigure} ~
    \begin{subfigure}[b]{0.45\textwidth}
        %\includegraphics[width=\textwidth]{}
        \caption{3 fibers each side, 1 above.}\label{subfig:volfrac3eachside1aboveMI}
    \end{subfigure}

    \begin{subfigure}[b]{0.45\textwidth}
        %\includegraphics[width=\textwidth]{}
        \caption{2 fibers each side, 2 above.}\label{subfig:volfrac2eachside2aboveMI}
    \end{subfigure} ~
    \begin{subfigure}[b]{0.45\textwidth}
        %\includegraphics[width=\textwidth]{}
        \caption{5 fibers each side, 2 above.}\label{subfig:volfrac5eachside2aboveMI}
    \end{subfigure}

\caption{A view of the effect of fiber volume fraction on Mode I ERR across different models.}\label{fig:volumefractionMI}
\end{figure}

\begin{figure}[!h]
\centering
    \begin{subfigure}[b]{0.45\textwidth}
        %\includegraphics[width=\textwidth]{}
        \caption{Single fiber model with free boundary on top.}\label{subfig:volfracfreeMII}
    \end{subfigure} ~
    \begin{subfigure}[b]{0.45\textwidth}
        %\includegraphics[width=\textwidth]{}
        \caption{Single fiber model with coupling of vertical displacements along the upper boundary.}\label{subfig:volfraccouplingMII}
    \end{subfigure}

    \begin{subfigure}[b]{0.45\textwidth}
        %\includegraphics[width=\textwidth]{}
        \caption{1 fiber each side.}\label{subfig:volfrac1eachsideMII}
    \end{subfigure} ~
    \begin{subfigure}[b]{0.45\textwidth}
        %\includegraphics[width=\textwidth]{}
        \caption{1 fiber above.}\label{subfig:volfrac1aboveMII}
    \end{subfigure}

    \begin{subfigure}[b]{0.45\textwidth}
        %\includegraphics[width=\textwidth]{}
        \caption{5 fibers each side.}\label{subfig:volfrac5eachsideMII}
    \end{subfigure} ~
    \begin{subfigure}[b]{0.45\textwidth}
        %\includegraphics[width=\textwidth]{}
        \caption{5 fibers above.}\label{subfig:volfrac5aboveMII}
    \end{subfigure}

    \begin{subfigure}[b]{0.45\textwidth}
        %\includegraphics[width=\textwidth]{}
        \caption{10 fibers each side.}\label{subfig:volfrac10eachsideMII}
    \end{subfigure} ~
    \begin{subfigure}[b]{0.45\textwidth}
        %\includegraphics[width=\textwidth]{}
        \caption{10 fibers above.}\label{subfig:volfrac10aboveMII}
    \end{subfigure}

    \begin{subfigure}[b]{0.45\textwidth}
        %\includegraphics[width=\textwidth]{}
        \caption{1 fiber each side, 1 above.}\label{subfig:volfrac1eachside1aboveMII}
    \end{subfigure} ~
    \begin{subfigure}[b]{0.45\textwidth}
        %\includegraphics[width=\textwidth]{}
        \caption{3 fibers each side, 1 above.}\label{subfig:volfrac3eachside1aboveMII}
    \end{subfigure}

    \begin{subfigure}[b]{0.45\textwidth}
        %\includegraphics[width=\textwidth]{}
        \caption{2 fibers each side, 2 above.}\label{subfig:volfrac2eachside2aboveMII}
    \end{subfigure} ~
    \begin{subfigure}[b]{0.45\textwidth}
        %\includegraphics[width=\textwidth]{}
        \caption{5 fibers each side, 2 above.}\label{subfig:volfrac5eachside2aboveMII}
    \end{subfigure}

\caption{A view of the effect of fiber volume fraction on Mode II ERR across different models.}\label{fig:volumefractionMII}
\end{figure}

\subsection{Interaction between debonds in a $90^{\circ}$ ply with a single layer of fibers inside a $\left[0^{\circ}_{n}, 90^{\circ}\right]_{S}$ laminate}

\textcolor{blue}{We start with a simpler (2 parameters: number of fibers in the horizontal directions + bounding ply thickness) but more extreme model: central $90^{\circ}$ ply with one line of fibers. What's the effect on $G_{I}$ and $G_{II}$? What's the effect of $0^{\circ}$ ply's thicknesses? Reference to Kies strain magnification. $G_{I}$ in Fig.~\ref{fig:sidefibersMI}, $G_{II}$ in Fig.~\ref{fig:sidefibersMII}.}

\textcolor{purple}{One graphic for each $V_{f}$ (30\%,50\%,60\%,65\%) and thickness ratio (1, 10), one curve for each case of fibers on the side (1, 2, 3, 5, 10, 50, 100) + curve for equivalent BC ()(Fig.~\ref{fig:sidefibersMI}, Fig.~\ref{fig:sidefibersMII}). Focus is effect of debond distribution in the horizontal direction.}\\

\begin{figure}[!h]
\centering
    \begin{subfigure}[b]{0.45\textwidth}
        %\includegraphics[width=\textwidth]{}
        \caption{$V_{f}=30\%$, $\frac{t_{0^{\circ}}}{t_{90^{\circ}}}=1$.}\label{subfig:sidefiber30MIthick1}
    \end{subfigure} ~
    \begin{subfigure}[b]{0.45\textwidth}
        %\includegraphics[width=\textwidth]{}
         \caption{$V_{f}=30\%$, $\frac{t_{0^{\circ}}}{t_{90^{\circ}}}=10$.}\label{subfig:sidefiber30MIthick10}
    \end{subfigure}

   \begin{subfigure}[b]{0.45\textwidth}
        %\includegraphics[width=\textwidth]{}
        \caption{$V_{f}=50\%$, $\frac{t_{0^{\circ}}}{t_{90^{\circ}}}=1$.}\label{subfig:sidefiber50MIthick1}
    \end{subfigure} ~
    \begin{subfigure}[b]{0.45\textwidth}
        %\includegraphics[width=\textwidth]{}
         \caption{$V_{f}=50\%$, $\frac{t_{0^{\circ}}}{t_{90^{\circ}}}=10$.}\label{subfig:sidefiber50MIthick10}
    \end{subfigure}

    \begin{subfigure}[b]{0.45\textwidth}
        %\includegraphics[width=\textwidth]{}
        \caption{$V_{f}=60\%$, $\frac{t_{0^{\circ}}}{t_{90^{\circ}}}=1$.}\label{subfig:sidefiber60MIthick1}
    \end{subfigure} ~
    \begin{subfigure}[b]{0.45\textwidth}
        %\includegraphics[width=\textwidth]{}
        \caption{$V_{f}=60\%$, $\frac{t_{0^{\circ}}}{t_{90^{\circ}}}=10$.}\label{subfig:sidefiber60MIthick10}
    \end{subfigure}

    \begin{subfigure}[b]{0.45\textwidth}
        %\includegraphics[width=\textwidth]{}
        \caption{$V_{f}=65\%$, $\frac{t_{0^{\circ}}}{t_{90^{\circ}}}=1$.}\label{subfig:sidefiber65MIthick1}
    \end{subfigure} ~
    \begin{subfigure}[b]{0.45\textwidth}
        %\includegraphics[width=\textwidth]{}
        \caption{$V_{f}=65\%$, $\frac{t_{0^{\circ}}}{t_{90^{\circ}}}=10$.}\label{subfig:sidefiber65MIthick10}
    \end{subfigure}

\caption{Effect of the interaction between debonds appearing at regular intervals on Mode I ERR in a $\left[0^{\circ}_{n}, 90^{\circ}\right]_{S}$ laminates in which the central $90^{\circ}$ ply possesses a single layer of fibers at different levels of fiber volume fraction $V_{f}$.}\label{fig:sidefibersMI}
\end{figure}

\begin{figure}[!h]
\centering
   \begin{subfigure}[b]{0.45\textwidth}
        %\includegraphics[width=\textwidth]{}
        \caption{$V_{f}=30\%$, $\frac{t_{0^{\circ}}}{t_{90^{\circ}}}=1$.}\label{subfig:sidefiber30MIIthick1}
    \end{subfigure} ~
    \begin{subfigure}[b]{0.45\textwidth}
        %\includegraphics[width=\textwidth]{}
         \caption{$V_{f}=30\%$, $\frac{t_{0^{\circ}}}{t_{90^{\circ}}}=10$.}\label{subfig:sidefiber30MIIthick10}
    \end{subfigure}

   \begin{subfigure}[b]{0.45\textwidth}
        %\includegraphics[width=\textwidth]{}
        \caption{$V_{f}=50\%$, $\frac{t_{0^{\circ}}}{t_{90^{\circ}}}=1$.}\label{subfig:sidefiber50MIIthick1}
    \end{subfigure} ~
    \begin{subfigure}[b]{0.45\textwidth}
        %\includegraphics[width=\textwidth]{}
         \caption{$V_{f}=50\%$, $\frac{t_{0^{\circ}}}{t_{90^{\circ}}}=10$.}\label{subfig:sidefiber50MIIthick10}
    \end{subfigure}

    \begin{subfigure}[b]{0.45\textwidth}
        %\includegraphics[width=\textwidth]{}
        \caption{$V_{f}=60\%$, $\frac{t_{0^{\circ}}}{t_{90^{\circ}}}=1$.}\label{subfig:sidefiber60MIIthick1}
    \end{subfigure} ~
    \begin{subfigure}[b]{0.45\textwidth}
        %\includegraphics[width=\textwidth]{}
        \caption{$V_{f}=60\%$, $\frac{t_{0^{\circ}}}{t_{90^{\circ}}}=10$.}\label{subfig:sidefiber60MIIthick10}
    \end{subfigure}

    \begin{subfigure}[b]{0.45\textwidth}
        %\includegraphics[width=\textwidth]{}
        \caption{$V_{f}=65\%$, $\frac{t_{0^{\circ}}}{t_{90^{\circ}}}=1$.}\label{subfig:sidefiber65MIIthick1}
    \end{subfigure} ~
    \begin{subfigure}[b]{0.45\textwidth}
        %\includegraphics[width=\textwidth]{}
        \caption{$V_{f}=65\%$, $\frac{t_{0^{\circ}}}{t_{90^{\circ}}}=10$.}\label{subfig:sidefiber65MIIthick10}
    \end{subfigure}

\caption{Effect of the interaction between debonds appearing at regular intervals on Mode II ERR in a single-ply laminate with a single layer of fibers at different levels of fiber volume fraction $V_{f}$.}\label{fig:sidefibersMII}
\end{figure}

\textcolor{purple}{One graphic for each $V_{f}$ (30\%,50\%,60\%,65\%) and selected cases of fibers on the side (1, 3,  10),  one curve for thickness ratio (1, 10) + curve for corresponding UD model + curve for equivalent BC (vertical displacement coupling+linear horizontal displacement)(Fig.~\ref{fig:sidefibersthicknessMI},  Fig.~\ref{fig:sidefibersthicknessMII}). Focus is effect of thickness of bounding plies.}\\

\begin{figure}[!h]
\centering
    \begin{subfigure}[b]{0.3\textwidth}
        %\includegraphics[width=\textwidth]{}
        \caption{$V_{f}=30\%$, 1 fiber on each side.}\label{subfig:sidefiber30MIcase1}
    \end{subfigure} ~
   \begin{subfigure}[b]{0.3\textwidth}
        %\includegraphics[width=\textwidth]{}
        \caption{$V_{f}=30\%$, 3 fibers on each side.}\label{subfig:sidefiber30MIcase2}
    \end{subfigure} ~
\begin{subfigure}[b]{0.3\textwidth}
        %\includegraphics[width=\textwidth]{}
        \caption{$V_{f}=30\%$, 10 fibers on each side.}\label{subfig:sidefiber30MIcase3}
    \end{subfigure}

    \begin{subfigure}[b]{0.3\textwidth}
        %\includegraphics[width=\textwidth]{}
        \caption{$V_{f}=50\%$, 1 fiber on each side.}\label{subfig:sidefiber50MIcase1}
    \end{subfigure} ~
   \begin{subfigure}[b]{0.3\textwidth}
        %\includegraphics[width=\textwidth]{}
        \caption{$V_{f}=50\%$, 3 fibers on each side.}\label{subfig:sidefiber50MIcase2}
    \end{subfigure} ~
\begin{subfigure}[b]{0.3\textwidth}
        %\includegraphics[width=\textwidth]{}
        \caption{$V_{f}=50\%$, 10 fibers on each side.}\label{subfig:sidefiber50MIcase3}
    \end{subfigure}

    \begin{subfigure}[b]{0.3\textwidth}
        %\includegraphics[width=\textwidth]{}
        \caption{$V_{f}=50\%$, 1 fiber on each side.}\label{subfig:sidefiber60MIcase1}
    \end{subfigure} ~
   \begin{subfigure}[b]{0.3\textwidth}
        %\includegraphics[width=\textwidth]{}
        \caption{$V_{f}=50\%$, 3 fibers on each side.}\label{subfig:sidefiber60MIcase2}
    \end{subfigure} ~
\begin{subfigure}[b]{0.3\textwidth}
        %\includegraphics[width=\textwidth]{}
        \caption{$V_{f}=50\%$, 10 fibers on each side.}\label{subfig:sidefiber60MIcase3}
    \end{subfigure}

    \begin{subfigure}[b]{0.3\textwidth}
        %\includegraphics[width=\textwidth]{}
        \caption{$V_{f}=65\%$, 1 fiber on each side.}\label{subfig:sidefiber65MIcase1}
    \end{subfigure} ~
   \begin{subfigure}[b]{0.3\textwidth}
        %\includegraphics[width=\textwidth]{}
        \caption{$V_{f}=65\%$, 3 fibers on each side.}\label{subfig:sidefiber65MIcase2}
    \end{subfigure} ~
\begin{subfigure}[b]{0.3\textwidth}
        %\includegraphics[width=\textwidth]{}
        \caption{$V_{f}=65\%$, 10 fibers on each side.}\label{subfig:sidefiber65MIcase3}
    \end{subfigure}

\caption{Effect of $0^{\circ}$ ply's thickness on the interaction between debonds appearing at regular intervals on Mode I ERR in a $\left[0^{\circ}_{n}, 90^{\circ}\right]_{S}$ laminate in which the central $90^{\circ}$ ply possesses a single layer of fibers at different levels of fiber volume fraction $V_{f}$.}\label{fig:sidefibersthicknessMI}
\end{figure}

\begin{figure}[!h]
\centering
    \begin{subfigure}[b]{0.3\textwidth}
        %\includegraphics[width=\textwidth]{}
        \caption{$V_{f}=30\%$, 1 fiber on each side.}\label{subfig:sidefiber30MIIcase1}
    \end{subfigure} ~
   \begin{subfigure}[b]{0.3\textwidth}
        %\includegraphics[width=\textwidth]{}
        \caption{$V_{f}=30\%$, 3 fibers on each side.}\label{subfig:sidefiber30MIIcase2}
    \end{subfigure} ~
\begin{subfigure}[b]{0.3\textwidth}
        %\includegraphics[width=\textwidth]{}
        \caption{$V_{f}=30\%$, 10 fibers on each side.}\label{subfig:sidefiber30MIIcase3}
    \end{subfigure}

    \begin{subfigure}[b]{0.3\textwidth}
        %\includegraphics[width=\textwidth]{}
        \caption{$V_{f}=50\%$, 1 fiber on each side.}\label{subfig:sidefiber50MIIcase1}
    \end{subfigure} ~
   \begin{subfigure}[b]{0.3\textwidth}
        %\includegraphics[width=\textwidth]{}
        \caption{$V_{f}=50\%$, 3 fibers on each side.}\label{subfig:sidefiber50MIIcase2}
    \end{subfigure} ~
\begin{subfigure}[b]{0.3\textwidth}
        %\includegraphics[width=\textwidth]{}
        \caption{$V_{f}=50\%$, 10 fibers on each side.}\label{subfig:sidefiber50MIIcase3}
    \end{subfigure}

    \begin{subfigure}[b]{0.3\textwidth}
        %\includegraphics[width=\textwidth]{}
        \caption{$V_{f}=50\%$, 1 fiber on each side.}\label{subfig:sidefiber60MIIcase1}
    \end{subfigure} ~
   \begin{subfigure}[b]{0.3\textwidth}
        %\includegraphics[width=\textwidth]{}
        \caption{$V_{f}=50\%$, 3 fibers on each side.}\label{subfig:sidefiber60MIIcase2}
    \end{subfigure} ~
\begin{subfigure}[b]{0.3\textwidth}
        %\includegraphics[width=\textwidth]{}
        \caption{$V_{f}=50\%$, 10 fibers on each side.}\label{subfig:sidefiber60MIIcase3}
    \end{subfigure}

    \begin{subfigure}[b]{0.3\textwidth}
        %\includegraphics[width=\textwidth]{}
        \caption{$V_{f}=65\%$, 1 fiber on each side.}\label{subfig:sidefiber65MIIcase1}
    \end{subfigure} ~
   \begin{subfigure}[b]{0.3\textwidth}
        %\includegraphics[width=\textwidth]{}
        \caption{$V_{f}=65\%$, 3 fibers on each side.}\label{subfig:sidefiber65MIIcase2}
    \end{subfigure} ~
\begin{subfigure}[b]{0.3\textwidth}
        %\includegraphics[width=\textwidth]{}
        \caption{$V_{f}=65\%$, 10 fibers on each side.}\label{subfig:sidefiber65MIIcase3}
    \end{subfigure}

\caption{Effect of $0^{\circ}$ ply's thickness on the interaction between debonds appearing at regular intervals on Mode II ERR in a $\left[0^{\circ}_{n}, 90^{\circ}\right]_{S}$ laminate in which the central $90^{\circ}$ ply possesses a single layer of fibers at different levels of fiber volume fraction $V_{f}$.}\label{fig:sidefibersthicknessMII}
\end{figure}

\subsection{Interaction between layers of fully bonded fibers and a centrally located line of debonded fibers in a $90^{\circ}$ ply inside a $\left[0^{\circ}_{n}, 90^{\circ}\right]_{S}$ laminate}

\textcolor{blue}{We then move to a ply with multiple lines of fibers and only debonded fibers in the central one (2 parameters: number of fibers in vertical direction + bounding ply thickness, a bit closer to real plies).  $G_{I}$ in Fig.~\ref{fig:abovefibersMI}, $G_{II}$ in Fig.~\ref{fig:abovefibersMII}.}

\textcolor{purple}{One graphic for each $V_{f}$ (30\%,50\%,60\%,65\%) and thickness ratio (1, 10), one curve for each case of fibers on top (1, 2, 3, 5, 10, 50, 100) + curve for equivalent BC (Fig.~\ref{fig:abovefibersMI}, Fig.~\ref{fig:abovefibersMII}). Focus is effect of debond distribution in the vertical direction.}\\

\begin{figure}[!h]
\centering
    \begin{subfigure}[b]{0.45\textwidth}
        %\includegraphics[width=\textwidth]{}
        \caption{$V_{f}=30\%$, $\frac{t_{0^{\circ}}}{t_{90^{\circ}}}=1$.}\label{subfig:abovefiber30MIthick1}
    \end{subfigure} ~
    \begin{subfigure}[b]{0.45\textwidth}
        %\includegraphics[width=\textwidth]{}
         \caption{$V_{f}=30\%$, $\frac{t_{0^{\circ}}}{t_{90^{\circ}}}=10$.}\label{subfig:abovefiber30MIthick10}
    \end{subfigure}

   \begin{subfigure}[b]{0.45\textwidth}
        %\includegraphics[width=\textwidth]{}
        \caption{$V_{f}=50\%$, $\frac{t_{0^{\circ}}}{t_{90^{\circ}}}=1$.}\label{subfig:abovefiber50MIthick1}
    \end{subfigure} ~
    \begin{subfigure}[b]{0.45\textwidth}
        %\includegraphics[width=\textwidth]{}
         \caption{$V_{f}=50\%$, $\frac{t_{0^{\circ}}}{t_{90^{\circ}}}=10$.}\label{subfig:abovefiber50MIthick10}
    \end{subfigure}

    \begin{subfigure}[b]{0.45\textwidth}
        %\includegraphics[width=\textwidth]{}
        \caption{$V_{f}=60\%$, $\frac{t_{0^{\circ}}}{t_{90^{\circ}}}=1$.}\label{subfig:abovefiber60MIthick1}
    \end{subfigure} ~
    \begin{subfigure}[b]{0.45\textwidth}
        %\includegraphics[width=\textwidth]{}
        \caption{$V_{f}=60\%$, $\frac{t_{0^{\circ}}}{t_{90^{\circ}}}=10$.}\label{subfig:abovefiber60MIthick10}
    \end{subfigure}

    \begin{subfigure}[b]{0.45\textwidth}
        %\includegraphics[width=\textwidth]{}
        \caption{$V_{f}=65\%$, $\frac{t_{0^{\circ}}}{t_{90^{\circ}}}=1$.}\label{subfig:abovefiber65MIthick1}
    \end{subfigure} ~
    \begin{subfigure}[b]{0.45\textwidth}
        %\includegraphics[width=\textwidth]{}
        \caption{$V_{f}=65\%$, $\frac{t_{0^{\circ}}}{t_{90^{\circ}}}=10$.}\label{subfig:abovefiber65MIthick10}
    \end{subfigure}

\caption{Influence of layers of fully bonded fibers on debond's growth in Mode I ERR in a centrally located line of debonded fibers at different levels of fiber volume fraction $V_{f}$ and thickness ratios.}\label{fig:abovefibersMI}
\end{figure}

\begin{figure}[!h]
\centering
   \begin{subfigure}[b]{0.45\textwidth}
        %\includegraphics[width=\textwidth]{}
        \caption{$V_{f}=30\%$, $\frac{t_{0^{\circ}}}{t_{90^{\circ}}}=1$.}\label{subfig:abovefiber30MIIthick1}
    \end{subfigure} ~
    \begin{subfigure}[b]{0.45\textwidth}
        %\includegraphics[width=\textwidth]{}
         \caption{$V_{f}=30\%$, $\frac{t_{0^{\circ}}}{t_{90^{\circ}}}=10$.}\label{subfig:abovefiber30MIIthick10}
    \end{subfigure}

   \begin{subfigure}[b]{0.45\textwidth}
        %\includegraphics[width=\textwidth]{}
        \caption{$V_{f}=50\%$, $\frac{t_{0^{\circ}}}{t_{90^{\circ}}}=1$.}\label{subfig:abovefiber50MIIthick1}
    \end{subfigure} ~
    \begin{subfigure}[b]{0.45\textwidth}
        %\includegraphics[width=\textwidth]{}
         \caption{$V_{f}=50\%$, $\frac{t_{0^{\circ}}}{t_{90^{\circ}}}=10$.}\label{subfig:abovefiber50MIIthick10}
    \end{subfigure}

    \begin{subfigure}[b]{0.45\textwidth}
        %\includegraphics[width=\textwidth]{}
        \caption{$V_{f}=60\%$, $\frac{t_{0^{\circ}}}{t_{90^{\circ}}}=1$.}\label{subfig:abovefiber60MIIthick1}
    \end{subfigure} ~
    \begin{subfigure}[b]{0.45\textwidth}
        %\includegraphics[width=\textwidth]{}
        \caption{$V_{f}=60\%$, $\frac{t_{0^{\circ}}}{t_{90^{\circ}}}=10$.}\label{subfig:abovefiber60MIIthick10}
    \end{subfigure}

    \begin{subfigure}[b]{0.45\textwidth}
        %\includegraphics[width=\textwidth]{}
        \caption{$V_{f}=65\%$, $\frac{t_{0^{\circ}}}{t_{90^{\circ}}}=1$.}\label{subfig:abovefiber65MIIthick1}
    \end{subfigure} ~
    \begin{subfigure}[b]{0.45\textwidth}
        %\includegraphics[width=\textwidth]{}
        \caption{$V_{f}=65\%$, $\frac{t_{0^{\circ}}}{t_{90^{\circ}}}=10$.}\label{subfig:abovefiber65MIIthick10}
    \end{subfigure}

\caption{Influence of layers of fully bonded fibers on debond's growth in Mode II ERR in a centrally located line of debonded fibers at different levels of fiber volume fraction $V_{f}$ and thickness ratios.}\label{fig:abovefibersMII}
\end{figure}

\textcolor{purple}{One graphic for each $V_{f}$ (30\%,50\%,60\%,65\%) and selected cases of fibers on top (1, 3,  10),  one curve for thickness ratio (1, 10) + curve for corresponding UD model + curve for equivalent BC (vertical displacement coupling+linear horizontal displacement)(Fig.~\ref{fig:abovefibersthicknessMI},  Fig.~\ref{fig:abovefibersthicknessMII}). Focus is effect of thickness of bounding plies.}\\

\begin{figure}[!h]
\centering
    \begin{subfigure}[b]{0.3\textwidth}
        %\includegraphics[width=\textwidth]{}
        \caption{$V_{f}=30\%$, 1 fiber on each side.}\label{subfig:abovefiber30MIcase1}
    \end{subfigure} ~
   \begin{subfigure}[b]{0.3\textwidth}
        %\includegraphics[width=\textwidth]{}
        \caption{$V_{f}=30\%$, 3 fibers on each side.}\label{subfig:abovefiber30MIcase2}
    \end{subfigure} ~
\begin{subfigure}[b]{0.3\textwidth}
        %\includegraphics[width=\textwidth]{}
        \caption{$V_{f}=30\%$, 10 fibers on each side.}\label{subfig:abovefiber30MIcase3}
    \end{subfigure}

    \begin{subfigure}[b]{0.3\textwidth}
        %\includegraphics[width=\textwidth]{}
        \caption{$V_{f}=50\%$, 1 fiber on each side.}\label{subfig:abovefiber50MIcase1}
    \end{subfigure} ~
   \begin{subfigure}[b]{0.3\textwidth}
        %\includegraphics[width=\textwidth]{}
        \caption{$V_{f}=50\%$, 3 fibers on each side.}\label{subfig:abovefiber50MIcase2}
    \end{subfigure} ~
\begin{subfigure}[b]{0.3\textwidth}
        %\includegraphics[width=\textwidth]{}
        \caption{$V_{f}=50\%$, 10 fibers on each side.}\label{subfig:abovefiber50MIcase3}
    \end{subfigure}

    \begin{subfigure}[b]{0.3\textwidth}
        %\includegraphics[width=\textwidth]{}
        \caption{$V_{f}=50\%$, 1 fiber on each side.}\label{subfig:abovefiber60MIcase1}
    \end{subfigure} ~
   \begin{subfigure}[b]{0.3\textwidth}
        %\includegraphics[width=\textwidth]{}
        \caption{$V_{f}=50\%$, 3 fibers on each side.}\label{subfig:abovefiber60MIcase2}
    \end{subfigure} ~
\begin{subfigure}[b]{0.3\textwidth}
        %\includegraphics[width=\textwidth]{}
        \caption{$V_{f}=50\%$, 10 fibers on each side.}\label{subfig:abovefiber60MIcase3}
    \end{subfigure}

    \begin{subfigure}[b]{0.3\textwidth}
        %\includegraphics[width=\textwidth]{}
        \caption{$V_{f}=65\%$, 1 fiber on each side.}\label{subfig:abovefiber65MIcase1}
    \end{subfigure} ~
   \begin{subfigure}[b]{0.3\textwidth}
        %\includegraphics[width=\textwidth]{}
        \caption{$V_{f}=65\%$, 3 fibers on each side.}\label{subfig:abovefiber65MIcase2}
    \end{subfigure} ~
\begin{subfigure}[b]{0.3\textwidth}
        %\includegraphics[width=\textwidth]{}
        \caption{$V_{f}=65\%$, 10 fibers on each side.}\label{subfig:abovefiber65MIcase3}
    \end{subfigure}

\caption{Effect of $0^{\circ}$ ply's thickness on the influence of layers of fully bonded fibers on debond's growth in Mode I ERR in a centrally located line of debonded fibers in the central $90^{\circ}$ ply of a $\left[0^{\circ}_{n}, 90^{\circ}\right]_{S}$ laminate at different levels of fiber volume fraction $V_{f}$.}\label{fig:abovefibersthicknessMI}
\end{figure}

\begin{figure}[!h]
\centering
    \begin{subfigure}[b]{0.3\textwidth}
        %\includegraphics[width=\textwidth]{}
        \caption{$V_{f}=30\%$, 1 fiber on each side.}\label{subfig:abovefiber30MIIcase1}
    \end{subfigure} ~
   \begin{subfigure}[b]{0.3\textwidth}
        %\includegraphics[width=\textwidth]{}
        \caption{$V_{f}=30\%$, 3 fibers on each side.}\label{subfig:abovefiber30MIIcase2}
    \end{subfigure} ~
\begin{subfigure}[b]{0.3\textwidth}
        %\includegraphics[width=\textwidth]{}
        \caption{$V_{f}=30\%$, 10 fibers on each side.}\label{subfig:abovefiber30MIIcase3}
    \end{subfigure}

    \begin{subfigure}[b]{0.3\textwidth}
        %\includegraphics[width=\textwidth]{}
        \caption{$V_{f}=50\%$, 1 fiber on each side.}\label{subfig:abovefiber50MIIcase1}
    \end{subfigure} ~
   \begin{subfigure}[b]{0.3\textwidth}
        %\includegraphics[width=\textwidth]{}
        \caption{$V_{f}=50\%$, 3 fibers on each side.}\label{subfig:abovefiber50MIIcase2}
    \end{subfigure} ~
\begin{subfigure}[b]{0.3\textwidth}
        %\includegraphics[width=\textwidth]{}
        \caption{$V_{f}=50\%$, 10 fibers on each side.}\label{subfig:abovefiber50MIIcase3}
    \end{subfigure}

    \begin{subfigure}[b]{0.3\textwidth}
        %\includegraphics[width=\textwidth]{}
        \caption{$V_{f}=50\%$, 1 fiber on each side.}\label{subfig:abovefiber60MIIcase1}
    \end{subfigure} ~
   \begin{subfigure}[b]{0.3\textwidth}
        %\includegraphics[width=\textwidth]{}
        \caption{$V_{f}=50\%$, 3 fibers on each side.}\label{subfig:abovefiber60MIIcase2}
    \end{subfigure} ~
\begin{subfigure}[b]{0.3\textwidth}
        %\includegraphics[width=\textwidth]{}
        \caption{$V_{f}=50\%$, 10 fibers on each side.}\label{subfig:abovefiber60MIIcase3}
    \end{subfigure}

    \begin{subfigure}[b]{0.3\textwidth}
        %\includegraphics[width=\textwidth]{}
        \caption{$V_{f}=65\%$, 1 fiber on each side.}\label{subfig:abovefiber65MIIcase1}
    \end{subfigure} ~
   \begin{subfigure}[b]{0.3\textwidth}
        %\includegraphics[width=\textwidth]{}
        \caption{$V_{f}=65\%$, 3 fibers on each side.}\label{subfig:abovefiber65MIIcase2}
    \end{subfigure} ~
\begin{subfigure}[b]{0.3\textwidth}
        %\includegraphics[width=\textwidth]{}
        \caption{$V_{f}=65\%$, 10 fibers on each side.}\label{subfig:abovefiber65MIIcase3}
    \end{subfigure}

\caption{Effect of $0^{\circ}$ ply's thickness on the influence of layers of fully bonded fibers on debond's growth in Mode I ERR in a centrally located line of debonded fibers in the central $90^{\circ}$ ply of a $\left[0^{\circ}_{n}, 90^{\circ}\right]_{S}$ laminate at different levels of fiber volume fraction $V_{f}$.}\label{fig:abovefibersthicknessMII}
\end{figure}

\subsection{Interaction of debonds within a $90^{\circ}$ ply with multiple layers of fibers inside a $\left[0^{\circ}_{n}, 90^{\circ}\right]_{S}$ laminate}

\textcolor{blue}{Finally models that are closer to real laminates and are more complex (3 parameters: number of fibers along the horizontal direction + number of layers in the vertical one +  bounding ply thickness).  $G_{I}$ in Fig.~\ref{fig:sideabovefibersMI}, $G_{II}$ in Fig.~\ref{fig:sideabovefibersMII}.}

\textcolor{purple}{One graphic for each $V_{f}$ (30\%,50\%,60\%,65\%) and thickness ratio (1, 10), one curve for each selected case of fibers on side and on top ([n. on side, n. on top]: [1,1], [2,1], [2,2], [5,1], [5,5], [10,1], [10,10]) + curve for equivalent BC (Fig.~\ref{fig:abovefibersMI}, Fig.~\ref{fig:sideabovefibersMII}). Focus is effect of debond distribution in the horizontal and vertical direction.}\\

\begin{figure}[!h]
\centering
    \begin{subfigure}[b]{0.45\textwidth}
        %\includegraphics[width=\textwidth]{}
        \caption{$V_{f}=30\%$, $\frac{t_{0^{\circ}}}{t_{90^{\circ}}}=1$.}\label{subfig:sideabovefiber30MIthick1}
    \end{subfigure} ~
    \begin{subfigure}[b]{0.45\textwidth}
        %\includegraphics[width=\textwidth]{}
         \caption{$V_{f}=30\%$, $\frac{t_{0^{\circ}}}{t_{90^{\circ}}}=10$.}\label{subfig:sideabovefiber30MIthick10}
    \end{subfigure}

   \begin{subfigure}[b]{0.45\textwidth}
        %\includegraphics[width=\textwidth]{}
        \caption{$V_{f}=50\%$, $\frac{t_{0^{\circ}}}{t_{90^{\circ}}}=1$.}\label{subfig:sideabovefiber50MIthick1}
    \end{subfigure} ~
    \begin{subfigure}[b]{0.45\textwidth}
        %\includegraphics[width=\textwidth]{}
         \caption{$V_{f}=50\%$, $\frac{t_{0^{\circ}}}{t_{90^{\circ}}}=10$.}\label{subfig:sideabovefiber50MIthick10}
    \end{subfigure}

    \begin{subfigure}[b]{0.45\textwidth}
        %\includegraphics[width=\textwidth]{}
        \caption{$V_{f}=60\%$, $\frac{t_{0^{\circ}}}{t_{90^{\circ}}}=1$.}\label{subfig:sideabovefiber60MIthick1}
    \end{subfigure} ~
    \begin{subfigure}[b]{0.45\textwidth}
        %\includegraphics[width=\textwidth]{}
        \caption{$V_{f}=60\%$, $\frac{t_{0^{\circ}}}{t_{90^{\circ}}}=10$.}\label{subfig:sideabovefiber60MIthick10}
    \end{subfigure}

    \begin{subfigure}[b]{0.45\textwidth}
        %\includegraphics[width=\textwidth]{}
        \caption{$V_{f}=65\%$, $\frac{t_{0^{\circ}}}{t_{90^{\circ}}}=1$.}\label{subfig:sideabovefiber65MIthick1}
    \end{subfigure} ~
    \begin{subfigure}[b]{0.45\textwidth}
        %\includegraphics[width=\textwidth]{}
        \caption{$V_{f}=65\%$, $\frac{t_{0^{\circ}}}{t_{90^{\circ}}}=10$.}\label{subfig:sideabovefiber65MIthick10}
    \end{subfigure}

\caption{Effect of the interaction of debonds within a $90^{\circ}$ ply with multiple layers of fibers on debond's growth in Mode I ERR at different levels of fiber volume fraction $V_{f}$ and thickness ratios.}\label{fig:sideabovefibersMI}
\end{figure}

\begin{figure}[!h]
\centering
   \begin{subfigure}[b]{0.45\textwidth}
        %\includegraphics[width=\textwidth]{}
        \caption{$V_{f}=30\%$, $\frac{t_{0^{\circ}}}{t_{90^{\circ}}}=1$.}\label{subfig:sideabovefiber30MIIthick1}
    \end{subfigure} ~
    \begin{subfigure}[b]{0.45\textwidth}
        %\includegraphics[width=\textwidth]{}
         \caption{$V_{f}=30\%$, $\frac{t_{0^{\circ}}}{t_{90^{\circ}}}=10$.}\label{subfig:sideabovefiber30MIIthick10}
    \end{subfigure}

   \begin{subfigure}[b]{0.45\textwidth}
        %\includegraphics[width=\textwidth]{}
        \caption{$V_{f}=50\%$, $\frac{t_{0^{\circ}}}{t_{90^{\circ}}}=1$.}\label{subfig:sideabovefiber50MIIthick1}
    \end{subfigure} ~
    \begin{subfigure}[b]{0.45\textwidth}
        %\includegraphics[width=\textwidth]{}
         \caption{$V_{f}=50\%$, $\frac{t_{0^{\circ}}}{t_{90^{\circ}}}=10$.}\label{subfig:sideabovefiber50MIIthick10}
    \end{subfigure}

    \begin{subfigure}[b]{0.45\textwidth}
        %\includegraphics[width=\textwidth]{}
        \caption{$V_{f}=60\%$, $\frac{t_{0^{\circ}}}{t_{90^{\circ}}}=1$.}\label{subfig:sideabovefiber60MIIthick1}
    \end{subfigure} ~
    \begin{subfigure}[b]{0.45\textwidth}
        %\includegraphics[width=\textwidth]{}
        \caption{$V_{f}=60\%$, $\frac{t_{0^{\circ}}}{t_{90^{\circ}}}=10$.}\label{subfig:sideabovefiber60MIIthick10}
    \end{subfigure}

    \begin{subfigure}[b]{0.45\textwidth}
        %\includegraphics[width=\textwidth]{}
        \caption{$V_{f}=65\%$, $\frac{t_{0^{\circ}}}{t_{90^{\circ}}}=1$.}\label{subfig:sideabovefiber65MIIthick1}
    \end{subfigure} ~
    \begin{subfigure}[b]{0.45\textwidth}
        %\includegraphics[width=\textwidth]{}
        \caption{$V_{f}=65\%$, $\frac{t_{0^{\circ}}}{t_{90^{\circ}}}=10$.}\label{subfig:sideabovefiber65MIIthick10}
    \end{subfigure}

\caption{Effect of the interaction of debonds within a $90^{\circ}$ ply with multiple layers of fibers on debond's growth in Mode II ERR at different levels of fiber volume fraction $V_{f}$ and thickness ratios.}\label{fig:sideabovefibersMII}
\end{figure}

\textcolor{purple}{One graphic for each $V_{f}$ (30\%,50\%,60\%,65\%) and selected cases of fibers on side and on top ([1,1], [5,1], [5,5]),  one curve for thickness ratio (1, 10) + curve for corresponding UD model + curve for equivalent BC (vertical displacement coupling+linear horizontal displacement)(Fig.~\ref{fig:sideabovefibersthicknessMI},  Fig.~\ref{fig:sideabovefibersthicknessMII}). Focus is effect of thickness of bounding plies.}\\

\begin{figure}[!h]
\centering
    \begin{subfigure}[b]{0.3\textwidth}
        %\includegraphics[width=\textwidth]{}
        \caption{$V_{f}=30\%$, 1 fiber on each side.}\label{subfig:sideabovefiber30MIcase1}
    \end{subfigure} ~
   \begin{subfigure}[b]{0.3\textwidth}
        %\includegraphics[width=\textwidth]{}
        \caption{$V_{f}=30\%$, 3 fibers on each side.}\label{subfig:sideabovefiber30MIcase2}
    \end{subfigure} ~
\begin{subfigure}[b]{0.3\textwidth}
        %\includegraphics[width=\textwidth]{}
        \caption{$V_{f}=30\%$, 10 fibers on each side.}\label{subfig:sideabovefiber30MIcase3}
    \end{subfigure}

    \begin{subfigure}[b]{0.3\textwidth}
        %\includegraphics[width=\textwidth]{}
        \caption{$V_{f}=50\%$, 1 fiber on each side.}\label{subfig:sideabovefiber50MIcase1}
    \end{subfigure} ~
   \begin{subfigure}[b]{0.3\textwidth}
        %\includegraphics[width=\textwidth]{}
        \caption{$V_{f}=50\%$, 3 fibers on each side.}\label{subfig:sideabovefiber50MIcase2}
    \end{subfigure} ~
\begin{subfigure}[b]{0.3\textwidth}
        %\includegraphics[width=\textwidth]{}
        \caption{$V_{f}=50\%$, 10 fibers on each side.}\label{subfig:sideabovefiber50MIcase3}
    \end{subfigure}

    \begin{subfigure}[b]{0.3\textwidth}
        %\includegraphics[width=\textwidth]{}
        \caption{$V_{f}=50\%$, 1 fiber on each side.}\label{subfig:sideabovefiber60MIcase1}
    \end{subfigure} ~
   \begin{subfigure}[b]{0.3\textwidth}
        %\includegraphics[width=\textwidth]{}
        \caption{$V_{f}=50\%$, 3 fibers on each side.}\label{subfig:sideabovefiber60MIcase2}
    \end{subfigure} ~
\begin{subfigure}[b]{0.3\textwidth}
        %\includegraphics[width=\textwidth]{}
        \caption{$V_{f}=50\%$, 10 fibers on each side.}\label{subfig:sideabovefiber60MIcase3}
    \end{subfigure}

    \begin{subfigure}[b]{0.3\textwidth}
        %\includegraphics[width=\textwidth]{}
        \caption{$V_{f}=65\%$, 1 fiber on each side.}\label{subfig:sideabovefiber65MIcase1}
    \end{subfigure} ~
   \begin{subfigure}[b]{0.3\textwidth}
        %\includegraphics[width=\textwidth]{}
        \caption{$V_{f}=65\%$, 3 fibers on each side.}\label{subfig:sideabovefiber65MIcase2}
    \end{subfigure} ~
\begin{subfigure}[b]{0.3\textwidth}
        %\includegraphics[width=\textwidth]{}
        \caption{$V_{f}=65\%$, 10 fibers on each side.}\label{subfig:sideabovefiber65MIcase3}
    \end{subfigure}

\caption{Effect of $0^{\circ}$ ply's thickness on debond's growth in Mode I ERR within the $90^{\circ}$ ply with multiple layers of fibers of a $\left[0^{\circ}_{n}, 90^{\circ}\right]_{S}$ laminate at different levels of fiber volume fraction $V_{f}$.}\label{fig:sideabovefibersthicknessMI}
\end{figure}

\begin{figure}[!h]
\centering
    \begin{subfigure}[b]{0.3\textwidth}
        %\includegraphics[width=\textwidth]{}
        \caption{$V_{f}=30\%$, 1 fiber on each side.}\label{subfig:sideabovefiber30MIIcase1}
    \end{subfigure} ~
   \begin{subfigure}[b]{0.3\textwidth}
        %\includegraphics[width=\textwidth]{}
        \caption{$V_{f}=30\%$, 3 fibers on each side.}\label{subfig:sideabovefiber30MIIcase2}
    \end{subfigure} ~
\begin{subfigure}[b]{0.3\textwidth}
        %\includegraphics[width=\textwidth]{}
        \caption{$V_{f}=30\%$, 10 fibers on each side.}\label{subfig:sideabovefiber30MIIcase3}
    \end{subfigure}

    \begin{subfigure}[b]{0.3\textwidth}
        %\includegraphics[width=\textwidth]{}
        \caption{$V_{f}=50\%$, 1 fiber on each side.}\label{subfig:sideabovefiber50MIIcase1}
    \end{subfigure} ~
   \begin{subfigure}[b]{0.3\textwidth}
        %\includegraphics[width=\textwidth]{}
        \caption{$V_{f}=50\%$, 3 fibers on each side.}\label{subfig:sideabovefiber50MIIcase2}
    \end{subfigure} ~
\begin{subfigure}[b]{0.3\textwidth}
        %\includegraphics[width=\textwidth]{}
        \caption{$V_{f}=50\%$, 10 fibers on each side.}\label{subfig:sideabovefiber50MIIcase3}
    \end{subfigure}

    \begin{subfigure}[b]{0.3\textwidth}
        %\includegraphics[width=\textwidth]{}
        \caption{$V_{f}=50\%$, 1 fiber on each side.}\label{subfig:sideabovefiber60MIIcase1}
    \end{subfigure} ~
   \begin{subfigure}[b]{0.3\textwidth}
        %\includegraphics[width=\textwidth]{}
        \caption{$V_{f}=50\%$, 3 fibers on each side.}\label{subfig:sideabovefiber60MIIcase2}
    \end{subfigure} ~
\begin{subfigure}[b]{0.3\textwidth}
        %\includegraphics[width=\textwidth]{}
        \caption{$V_{f}=50\%$, 10 fibers on each side.}\label{subfig:sideabovefiber60MIIcase3}
    \end{subfigure}

    \begin{subfigure}[b]{0.3\textwidth}
        %\includegraphics[width=\textwidth]{}
        \caption{$V_{f}=65\%$, 1 fiber on each side.}\label{subfig:sideabovefiber65MIIcase1}
    \end{subfigure} ~
   \begin{subfigure}[b]{0.3\textwidth}
        %\includegraphics[width=\textwidth]{}
        \caption{$V_{f}=65\%$, 3 fibers on each side.}\label{subfig:sideabovefiber65MIIcase2}
    \end{subfigure} ~
\begin{subfigure}[b]{0.3\textwidth}
        %\includegraphics[width=\textwidth]{}
        \caption{$V_{f}=65\%$, 10 fibers on each side.}\label{subfig:sideabovefiber65MIIcase3}
    \end{subfigure}

\caption{Effect of $0^{\circ}$ ply's thickness on debond's growth in Mode II ERR within the $90^{\circ}$ ply with multiple layers of fibers of a $\left[0^{\circ}_{n}, 90^{\circ}\right]_{S}$ laminate at different levels of fiber volume fraction $V_{f}$.}\label{fig:sideabovefibersthicknessMII}
\end{figure}

\section{Conclusions \& Outlook}

\section*{Acknowledgements}

Luca Di Stasio gratefully acknowledges the support of the European School of Materials (EUSMAT) through the DocMASE Doctoral Programme and the European Commission through the Erasmus Mundus Programme.

\bibliography{refs}

\end{document}
