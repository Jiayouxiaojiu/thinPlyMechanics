\documentclass{standalone}

%----------------------------------------------------------------------------------------------%
%                                 Packages and basic declarations
%----------------------------------------------------------------------------------------------%

\usepackage{tikz}
\usepackage{verbatim}
\usepackage{pgf}
\usepackage{tikz}
\usepackage{mathrsfs}
\usepackage{tkz-euclide}
\usetkzobj{all} 
\usetikzlibrary{arrows}

%----------------------------------------------------------------------------------------------%
%                                          INPUT PARAMETERS
%----------------------------------------------------------------------------------------------%

\def\Rf{1}
\def\Vff{0.5}
\def\tratio{0.75}
\def\meshfacone{0.2}
\def\meshfactwo{0.75}
\def\meshfacthree{0.5}
\def\thetavalue{0}
\def\deltatheta{50}

%----------------------------------------------------------------------------------------------%
%                               Definition of dependent parameters
%----------------------------------------------------------------------------------------------%

\def\pivalue{3.141592653589793238462643383279502884197169399375105820974944592307816406286}

\newcommand{\half}[1]{
       0.5*#1
       }

\pgfmathsetmacro\sqrtthree{sqrt(3)}

\pgfmathsetmacro\equitriheight{0.5*sqrt(3)}

\pgfmathsetmacro\l{0.5*\Rf*sqrt(\pivalue/\Vff)}

\pgfmathsetmacro\symmsyml{\l/4}

\pgfmathsetmacro\symmsymh{\symmsyml*\equitriheight}

\def\domlim{1.28*\l}
\def\loadlim{1.15*\l}
\def\loadlabel{0.2*\Rf}
\def\cornerlabel{1.077*\l}

\def\thetabot{\thetavalue-\deltatheta}
\def\thetaup{\thetavalue+\deltatheta}

\def\thetabotsymm{180-\deltatheta}
\def\thetaupsymm{180+\deltatheta}

\def\thetahalfbot{\thetavalue-0.5*\deltatheta}
\def\thetahalfup{\thetavalue+0.5*\deltatheta}

\def\thetaround{360+\thetavalue-\deltatheta}
\def\thetadraw{0.25*\thetavalue}

\def\xM{0.9*\costheta*\Rf}
\def\yM{0.9*\sintheta*\Rf}

\pgfmathsetmacro\cosfourtyfive{cos(45)}
\pgfmathsetmacro\sinfourtyfive{sin(45)}

\pgfmathsetmacro\costheta{cos(\thetavalue)}
\pgfmathsetmacro\sintheta{sin(\thetavalue)}

\pgfmathsetmacro\costhetabot{cos(\thetabot)}
\pgfmathsetmacro\sinthetabot{sin(\thetabot)}

\pgfmathsetmacro\costhetaup{cos(\thetaup)}
\pgfmathsetmacro\sinthetaup{sin(\thetaup)}

\pgfmathsetmacro\costhetahalfbot{cos(\thetahalfbot)}
\pgfmathsetmacro\sinthetahalfbot{sin(\thetahalfbot)}

\pgfmathsetmacro\costhetahalfup{cos(\thetahalfup)}
\pgfmathsetmacro\sinthetahalfup{sin(\thetahalfup)}
  
\pgfmathsetmacro\yloadarrowone{\l+(\loadlim-\l)*0.2}
\pgfmathsetmacro\yloadarrowtwo{\l+2*(\loadlim-\l)*0.2}
\pgfmathsetmacro\yloadarrowthree{\l+3*(\loadlim-\l)*0.2}
\pgfmathsetmacro\yloadarrowfour{\l+4*(\loadlim-\l)*0.2}

\pgfmathsetmacro\ILsquared{(\costhetabot-\costhetaup)*(\costhetabot-\costhetaup)+(\sinthetabot-\sinthetaup)*(\sinthetabot-\sinthetaup))}
\pgfmathsetmacro\IMsquared{(\costhetabot-0.9*\costheta)*(\costhetabot-0.9*\costheta)+(\sinthetabot-0.9*\sintheta)*(\sinthetabot-0.9*\sintheta)}
\pgfmathsetmacro\IL{sqrt(\ILsquared)}
\pgfmathsetmacro\IM{sqrt(\IMsquared)}
\pgfmathsetmacro\angleM{asin(0.5*\IL/\IM)}

\def\crackstartangle{\thetavalue-\angleM}
\def\crackstopangle{\thetavalue+\angleM}

\pgfmathsetmacro\meshradiusone{\meshfactwo*\Rf}
\pgfmathsetmacro\meshradiustwo{\Rf+\meshfacthree*(\l-\Rf)}

%----------------------------------------------------------------------------------------------%
%----------------------------------------------------------------------------------------------%
%                                            DOCUMENT STARTS
%----------------------------------------------------------------------------------------------%
%----------------------------------------------------------------------------------------------%

\begin{document}

%----------------------------------------------------------------------------------------------%
%                Single RVE with applied constant strain, only debonded
%----------------------------------------------------------------------------------------------%
 
\begin{tikzpicture}[scale=1.3,cap=round,x=1cm,y=1cm]

\tikzstyle{axes}=[]

\draw[dashed,line width=0.75mm] (0,0)--(\costhetabot,\sinthetabot);
\draw[dashed,line width=0.75mm] (0,0)--(\costhetaup,\sinthetaup);

\foreach \x in {0}{
% debond on the right
\draw[line width=1.25mm] (\costhetaup+\x*4*\l,\sinthetaup)arc (\thetaup:\thetaround:\Rf);
\draw[draw=red,line width=1.25mm](\costhetaup+\x*4*\l,\sinthetaup) arc(\thetaup:\thetabot:\Rf);
\tkzDefPoint(\costhetaup+\x*4*\l,\sinthetaup){A}\tkzDefPoint(1.2*\Rf+\x*4*\l,0){B}\tkzDefPoint(\costhetabot+\x*4*\l,\sinthetabot){C}
\tkzCircumCenter(A,B,C)\tkzGetPoint{O}
\tkzDrawArc[draw=red,line width=1.25mm](O,C)(A)
}


\foreach \x in {-2,1}{
% debond on the left
\draw[line width=1.25mm] (2*\l+\x*4*\l-\costhetaup,-\sinthetaup)arc (-\thetaup:\thetaround:-\Rf);
\draw[draw=red,line width=1.25mm](2*\l+\x*4*\l-\costhetaup,\sinthetaup) arc(180+\thetabot:180-\thetaup:\Rf);
\tkzDefPoint(2*\l+\x*4*\l-\costhetaup,\sinthetaup){A}\tkzDefPoint(2*\l+\x*4*\l-1.2*\Rf,0){B}\tkzDefPoint(2*\l+\x*4*\l-\costhetabot,\sinthetabot){C}
\tkzCircumCenter(A,B,C)\tkzGetPoint{O}
\tkzDrawArc[draw=red,line width=1.25mm](O,A)(C)
}

\foreach \x in {0}{


% fully bonded fibers +/- 1
\draw[line width=1.25mm] (2*\l+\Rf,0+\x*2*\l,0)arc (0:360:\Rf);
\draw[line width=1.25mm] (-2*\l+\Rf,0+\x*2*\l,0)arc (0:360:\Rf);

% fully bonded fibers +/- 2
\draw[line width=1.25mm] (4*\l+\Rf,0+\x*2*\l,0)arc (0:360:\Rf);
\draw[line width=1.25mm] (-4*\l+\Rf,0+\x*2*\l,0)arc (0:360:\Rf);


% fully bonded fibers +/- 4
%\draw[line width=1.25mm] (8*\l+\Rf,0+\x*2*\l,0)arc (0:360:\Rf);
%\draw[line width=1.25mm] (-8*\l+\Rf,0+\x*2*\l,0)arc (0:360:\Rf);
}

\foreach \x in {-1,1}{

% center fiber
\draw[line width=1.25mm] (\Rf,0+\x*2*\l,0)arc (0:360:\Rf);

% fully bonded fibers +/- 1
\draw[line width=1.25mm] (2*\l+\Rf,0+\x*2*\l,0)arc (0:360:\Rf);
\draw[line width=1.25mm] (-2*\l+\Rf,0+\x*2*\l,0)arc (0:360:\Rf);

% fully bonded fibers +/- 2
\draw[line width=1.25mm] (4*\l+\Rf,0+\x*2*\l,0)arc (0:360:\Rf);
\draw[line width=1.25mm] (-4*\l+\Rf,0+\x*2*\l,0)arc (0:360:\Rf);

% fully bonded fibers +/- 3
\draw[line width=1.25mm] (6*\l+\Rf,0+\x*2*\l,0)arc (0:360:\Rf);
\draw[line width=1.25mm] (-6*\l+\Rf,0+\x*2*\l,0)arc (0:360:\Rf);

% fully bonded fibers +/- 4
%\draw[line width=1.25mm] (8*\l+\Rf,0+\x*2*\l,0)arc (0:360:\Rf);
%\draw[line width=1.25mm] (-8*\l+\Rf,0+\x*2*\l,0)arc (0:360:\Rf);
}

\begin{scope}[style=axes]
  \draw[->,line width=0.75mm] (-7.2*\l,0) -- (7.3*\l,0) node[right] {\Huge$\underline{i}, x$};
  \draw[->,line width=0.75mm] (0,-9.2*\l) -- (0,9.3*\l) node[above] {\Huge$\underline{k}, z$};
\end{scope}

\pgfmathsetmacro\ltot{\l+\tratio*\l}


\draw[line width=1.25mm] (-7*\l,3*\l) -- (7*\l,3*\l) ;
\draw[line width=1.25mm] (-7*\l,-3*\l) -- (7*\l,-3*\l) ;
\draw[line width=1.25mm] (7*\l,3*\l) -- (7*\l,-3*\l);
\draw[line width=1.25mm] (-7*\l,3*\l) -- (-7*\l,-3*\l);

\draw[line width=1.25mm] (7*\l,3*\l) -- (7*\l,9*\l);
\draw[line width=1.25mm] (-7*\l,3*\l) -- (-7*\l,9*\l);
\draw[line width=1.25mm] (7*\l,-3*\l) -- (7*\l,-9*\l);
\draw[line width=1.25mm] (-7*\l,-3*\l) -- (-7*\l,-9*\l);

\draw[line width=1.25mm] (-7*\l,9*\l) -- (7*\l,9*\l) ;
\draw[line width=1.25mm] (-7*\l,-9*\l) -- (7*\l,-9*\l) ;

\draw[dashed,blue, line width=1.25mm] (3*\l,0) -- (3*\l,9*\l) ;
\draw[dashed,blue, line width=1.25mm] (-3*\l,0) -- (-3*\l,9*\l) ;
\draw[dashed,blue, line width=1.25mm] (-3*\l,0) -- (3*\l,0) ;
\draw[dashed,blue, line width=1.25mm] (-3*\l,9*\l) -- (3*\l,9*\l) ;

%bottom symmetry
\foreach \x in {0,...,12}{
\draw[dashed,blue, line width=1.25mm](-3*\l+\x*2*\symmsyml,0)--(-3*\l+\x*2*\symmsyml-0.5*\symmsyml,-\symmsymh)--(-3*\l+\x*2*\symmsyml-0.5*\symmsyml+\symmsyml,-\symmsymh)--(-3*\l+\x*2*\symmsyml,0);
\draw[dashed,blue, line width=1.25mm] (-3*\l+\x*2*\symmsyml,-\symmsymh-0.25*\symmsyml,0)arc (0:360:0.25*\symmsyml);
\draw[dashed,blue, line width=1.25mm] (-3*\l+\x*2*\symmsyml+0.5*\symmsyml,-\symmsymh-0.25*\symmsyml,0)arc (0:360:0.25*\symmsyml);}

%left symmetry
\foreach \x in {0,...,18}{
\draw[dashed,blue, line width=1.25mm](-3*\l,\x*2*\symmsyml)--(-3*\l-\symmsymh,\x*2*\symmsyml-0.5*\symmsyml)--(-3*\l-\symmsymh,\x*2*\symmsyml+0.5*\symmsyml)--(-3*\l,\x*2*\symmsyml);
\draw[dashed,blue, line width=1.25mm] (-3*\l-\symmsymh,\x*2*\symmsyml-0.25*\symmsyml,0)arc (0:360:0.25*\symmsyml);
\draw[dashed,blue, line width=1.25mm] (-3*\l-\symmsymh,\x*2*\symmsyml+0.25*\symmsyml,0)arc (0:360:0.25*\symmsyml);}

%right symmetry
\foreach \x in {0,...,18}{
\draw[dashed,blue, line width=1.25mm](3*\l,\x*2*\symmsyml)--(3*\l+\symmsymh,\x*2*\symmsyml-0.5*\symmsyml)--(3*\l+\symmsymh,\x*2*\symmsyml+0.5*\symmsyml)--(3*\l,\x*2*\symmsyml);
\draw[dashed,blue, line width=1.25mm] (3*\l+\symmsymh+0.5*\symmsyml,\x*2*\symmsyml-0.25*\symmsyml,0)arc (0:360:0.25*\symmsyml);
\draw[dashed,blue, line width=1.25mm] (3*\l+\symmsymh+0.5*\symmsyml,\x*2*\symmsyml+0.25*\symmsyml,0)arc (0:360:0.25*\symmsyml);}

\draw[blue,line width=1.25mm] (-3*\l,-9.25*\l) -- (3*\l,-9.25*\l) ;
\draw[blue,line width=1.25mm] (-3*\l,-9.15*\l) -- (-3*\l,-9.35*\l) ;
\draw[blue,line width=1.25mm] (3*\l,-9.15*\l) -- (3*\l,-9.35*\l) ;

\draw[blue,line width=1.25mm] (-7.25*\l,-3*\l) -- (-7.25*\l,3*\l) ;
\draw[blue,line width=1.25mm] (-7.15*\l,3*\l) -- (-7.35*\l,3*\l) ;
\draw[blue,line width=1.25mm] (-7.15*\l,-3*\l) -- (-7.35*\l,-3*\l) ;

\draw[blue,line width=1.25mm] (8.25*\l,3*\l) -- (8.25*\l,9*\l) ;
\draw[blue,line width=1.25mm] (8.15*\l,3*\l) -- (8.35*\l,3*\l) ;
\draw[blue,line width=1.25mm] (8.15*\l,9*\l) -- (8.35*\l,9*\l) ;

\draw[blue,line width=1.25mm] (8.25*\l,-3*\l) -- (8.25*\l,-9*\l) ;
\draw[blue,line width=1.25mm] (8.15*\l,-3*\l) -- (8.35*\l,-3*\l) ;
\draw[blue,line width=1.25mm] (8.15*\l,-9*\l) -- (8.35*\l,-9*\l) ;

\draw[blue,line width=1.25mm] (8.25*\l,-3*\l) -- (8.25*\l,3*\l) ;

\draw[->,line width=1.25mm] (5*\l,6*\l) -- (6*\l,6*\l)node[right] {\Huge$\mathbf{0^{\circ}}$};
\draw[->,line width=1.25mm] (5*\l,-6*\l) -- (6*\l,-6*\l)node[right] {\Huge$\mathbf{0^{\circ}}$};

\draw (0,-9.35*\l) node[blue,left,below] {\Huge$\mathbf{n=3}$};
\draw (-7.3*\l,0) node[blue,above,rotate=90] {\Huge$\mathbf{k=3}$};

\draw (8.3*\l,6*\l) node[blue,below,rotate=90] {\Huge$\mathbf{p\cdot t_{90^{\circ}}}$};
\draw (8.3*\l,-6*\l) node[blue,below,rotate=90] {\Huge$\mathbf{p\cdot t_{90^{\circ}}}$};
\draw (8.3*\l,0) node[blue,below,rotate=90] {\Huge$\mathbf{k\cdot2L}$};

\end{tikzpicture}

\end{document}