%------------------------------------------------------------------------------------------%
%------------------------------------------------------------------------------------------%
%------------------------------------------------------------------------------------------%
%                                      FILE BEGINS
%------------------------------------------------------------------------------------------%
%------------------------------------------------------------------------------------------%
%------------------------------------------------------------------------------------------%

%------------------------------------------------------------------------------------------%
%------------------------------------------------------------------------------------------%
%                                    DOCUMENT CLASS
%------------------------------------------------------------------------------------------%
%------------------------------------------------------------------------------------------%
\documentclass[a4paper]{jpconf}

%------------------------------------------------------------------------------------------%
%------------------------------------------------------------------------------------------%
%                                       PACKAGES
%------------------------------------------------------------------------------------------%
%------------------------------------------------------------------------------------------%
\usepackage{amsmath}
\usepackage{booktabs}
\usepackage{cite}
\usepackage{float}
\usepackage{graphicx}
\usepackage[caption=false]{subfig}
\usepackage[makeroom]{cancel}

%------------------------------------------------------------------------------------------%
%------------------------------------------------------------------------------------------%
%                                    DOCUMENT BEGINS
%------------------------------------------------------------------------------------------%
%------------------------------------------------------------------------------------------%
\begin{document}

%------------------------------------------------------------------------------------------%
%                                        HEADER
%------------------------------------------------------------------------------------------%

\title{On the meaning of the normalisation parameter $G_{0}$ and the normalised Energy Release Rates in the fiber/matrix interface problem}

\author{Luca Di Stasio$^{1,2,a}$ , Janis Varna$^{2,b}$ and Zoubir Ayadi$^{1,c}$ }

\address{$^{1}$SI2M, IJL, EEIGM, Universit\'e de Lorraine, 6 Rue Bastien Lepage, F-54010 Nancy, France\\$^{2}$Division of Polymer Engineering, Lule\aa\ University of Technology, SE-97187 Lule\aa , Sweden }

{\vspace*{5pt}\address{E-mail: $^{a}$luca.di-stasio@univ-lorraine.fr, $^{b}$janis.varna@ltu.se, $^{c}$zoubir.ayadi@univ-lorraine.fr}}
%\address{$^{a}$luca.di-stasio@univ-lorraine.fr}, \address{$^{b}$janis.varna@ltu.se}, \address{$^{c}$zoubir.ayadi@univ-lorraine.fr}

%------------------------------------------------------------------------------------------%
%                                       ABSTRACT
%------------------------------------------------------------------------------------------%

\begin{abstract}

\end{abstract}

%------------------------------------------------------------------------------------------%
%                                   List of acronyms
%------------------------------------------------------------------------------------------%

\section*{List of acronyms}

\begin{tabular}{ll}
BEM &  Boundary Element Method\\
ERR &  Energy Release Rate (here used as synonim of SERR)\\
FEM &  Finite Element Method\\
SERR &  Strain Energy Release Rate\\
VCCT &  Virtual Crack Closure Technique\\
\end{tabular}
%\end{table}

%------------------------------------------------------------------------------------------%
%                                   List of symbols
%------------------------------------------------------------------------------------------%

\section*{List of symbols}

%\begin{table}[!h]
\begin{tabular}{lcl}

\end{tabular}
%\end{table}

\clearpage
%------------------------------------------------------------------------------------------%
%                             Intro
%------------------------------------------------------------------------------------------%

\section{Introduction}

The numerical analysis of the fiber/matrix interface has focused on the determination of mode mixity through the calculation of mode I and mode II energy release rates. In early papers, where the problem was solved in the complex domain by means of the Airy function and conformal transformations, it was shown that the stress field was due to oscillate in a very small region close to the crack tip. Thus, the evalution of stresses and stress intensity factors represents an ardous obstacle when to the discrete counterpart of the problem. \\
Due to their definition as rates of energy change, the calculation of ERRs fits better into the frameworks of discrete procedures, whether the Finite Element Method (FEM) or the Boundary Element Method (BEM). Different variations of the same principle has been derived over the years, namely the Crack Closure Technique, the Crack Closure Integral, the Virtual Crack Closure Technique and the Virtual Crack Closure Integral. The idea at the core of these methods is that, given that the crack is propagating in a linear elastic medium, the energy released by the creation of a unit area of crack's surfaces is equal to the work needed to close the new created surfaces back together.\\
Energy Release Rates have been so far reported in a normalized form, where a reference energy release rate $G_{0}$ is used as normalization parameter. In \cite{sevilla one fiber}, the authors claim that the use of

\begin{equation}
\label{eq:G0}
G_{0} = \frac{1+k_{m}}{8\mu_{m}}\sigma_{0}^{2}R_{f}\pi
\end{equation}

would make the results comparable between different material systems. In equation \ref{eq:G0}, $k_{m}$ is the Kolosov constant for the matrix, which is equal to $3-4\nu$ for plane strain and $\frac{3-\nu}{1+\nu}$ for plane stress conditions, $\mu_{m}$ is the shear modulus of the matrix, $\sigma_{0}$ is the applied stress at the boundary and $R_{f}$ is the radius of the inclusion. \\
In \cite{sevilla two fiber}, the same normalization parameter is used to analyze the effect of a neighbouring fiber on the Energy Release Rates. A similar use can be found in \cite{Linqi} It seems that the first apperance of $G_{0}$ can be retrieved in \cite{Toya}, soon followed by \cite{handbook of SIF} where a parametric study of Toya's analytical results is performed and tabulated. However, in Toya the normalization is performed for the maximum? stress at the crack tip, and its actual formulation is \textit{formula from toya}.

The question thus arises: what is the meaning of $G_{0}$? And consequently, what is the meaning of the normalized Energy Release Rates? How does the selection of this peculiar normalization value make results comparable across different material systems and ply geoemetries? In this brief note we will try to answer these questions.

%------------------------------------------------------------------------------------------%
%                             G0
%------------------------------------------------------------------------------------------%

\section{Analysis of $G_{0}$}

Consider an infinite plate made of an isotropic homogeneous material with a central crack of length $2a$ in its center. We assume that the material is the same as matrix constituent of a composite, and thus we will use the pedix $m$ to identify the properties. Under an applied remote traction $\sigma_{0}$ perpendicular to the crack faces, the mode I Stress Intensity Factor (SIF) reads \cite{find ref}

\begin{equation}
\label{eq:cc-plate-k1}
K_{I}^{CC} = \sigma_{0}\sqrt{\pi a}.
\end{equation}

Across the literature on Linear Elastic Fracture Mechanics, in textbooks and handbooks, the expression in eq. \ref{eq:cc-plate-k1} has assumed the role of reference for the expression of Stress Intensity Factors for all possible modes of fracture and geometries. Thus, the SIF for any other arbitrary geometrical configuration of crack and sorrounding structure can be expressed as

\begin{equation}
K_{mode} = \beta K_{ref}\quad\text{where}\quad K_{ref}=K_{I}^{CC}=\sigma_{0}\sqrt{\pi a}
\end{equation}

or, equivalently,

\begin{equation}
K_{mode} = \beta\sigma_{0}\sqrt{\pi a}
\end{equation}

where $\beta$ is the shape factor. Thus, evaluation of the SIF for a new configuration reduces to the determination of its shape factor.\\
The corresponding energy release rate can be computed to be

\begin{equation}
G_{ref}=G_{I}^{CC}=\frac{\left(K_{I}^{CC}\right)^{2}}{E_{m}^{*}}.
\end{equation}

For plane strain,

\begin{equation}
E_{m}^{*}=\frac{E_{m}}{1-\nu^{2}}\quad\text{and thus}\quad G_{ref}=\frac{1-\nu^{2}}{E_{m}}\sigma_{0}^{2}\pi a,
\end{equation}

while for plane stress,

\begin{equation}
E_{m}^{*}=E_{m}\quad\text{and thus}\quad G_{ref}=\frac{\sigma_{0}^{2}\pi a}{E_{m}}.
\end{equation}

The energy release rate for the generic configuration thus reads

\begin{equation}
G_{mode}=\beta G_{ref},
\end{equation}

which particularizes to $G_{mode}=\beta\frac{1-\nu^{2}}{E_{m}}\sigma_{0}^{2}\pi a $ for plane strain, and $ G_{mode}=\beta\frac{\sigma_{0}^{2}\pi a}{E_{m}} $ for plane stress.\\
A homogeneous isotropic material has only $2$ independent parameters that determine its elastic behavior; thus it holds

\begin{equation}
G_{m}=\frac{E_{m}}{2\left(1+\nu_{m}\right)}.
\end{equation}

Let us consider once again the reference energy release rate for plane strain states and rework its expression

\begin{equation}
\label{eq:gref-planestrain}
\begin{aligned}
G_{ref}=&\frac{1-\nu_{m}^{2}}{E_{m}}\sigma_{0}^{2}\pi a=\\
=&\frac{\left(1-\nu_{m}\right)\left(1+\nu_{m}\right)}{E_{m}}\sigma_{0}^{2}\pi a=\\
=&\frac{\left(1-\nu_{m}\right)}{2G_{m}}\sigma_{0}^{2}\pi a=\\
=&\frac{4}{4}\frac{\left(1-\nu_{m}\right)}{2G_{m}}\sigma_{0}^{2}\pi a=\\
=&\frac{1+3-4\nu_{m}}{8G_{m}}\sigma_{0}^{2}\pi a=\\
=&\frac{1+k_{m}}{8G_{m}}\sigma_{0}^{2}\pi a
\end{aligned}
\end{equation}

where $k_{m}=3-4\nu_{m}$ is the Kolosov's constant for plane strain.\\
Similarly, for plane stress

\begin{equation}
\label{eq:gref-planestress}
\begin{aligned}
G_{ref}=&\frac{\sigma_{0}^{2}\pi a}{E_{m}}=\\
=&\frac{\sigma_{0}^{2}\pi a}{2G_{m}\left(1+\nu_{m}\right)}=\\
=&\frac{4}{4}\frac{1}{2G_{m}\left(1+\nu_{m}\right)}\sigma_{0}^{2}\pi a\\
=&\frac{1}{8G_{m}}\frac{4+\nu_{m}-\nu_{m}}{1+\nu_{m}}\sigma_{0}^{2}\pi a\\
=&\frac{1}{8G_{m}}\left(\frac{1+\nu_{m}}{1+\nu_{m}}+\frac{3-\nu_{m}}{1+\nu_{m}}\right)\sigma_{0}^{2}\pi a=\\
=&\frac{1+\frac{3-\nu_{m}}{1+\nu_{m}}}{8G_{m}}\sigma_{0}^{2}\pi a=\\%
=&\frac{1+k_{m}}{8G_{m}}\sigma_{0}^{2}\pi a
\end{aligned}
\end{equation}

where $k_{m}=\frac{3-\nu_{m}}{1+\nu_{m}}$ is the Kolosov's constant for plane stress.\\
Recalling now eq. \ref{eq:G0} and comparing with eqs. \ref{eq:gref-planestrain} and \ref{eq:gref-planestress}, we can now identify a clear physical meaning: $G_{0}$ is the mode I energy release rate for a central crack of total length equal to the inclusion diameter $2R_{f}$ in an infinite plate made of the same material as the matrix. Furthermore, a historical purpose can be identified, as this formulation of $G_{0}$ allow for the expression of the energy release rates for the fiber/matrix interface problem as

\begin{equation}
G_{I}=\beta_{I} G_{ref}\qquad G_{II}=\beta_{II} G_{ref},
\end{equation}

where $\beta_{I}$ and $\beta_{II}$ are respectively the mode I and mode II shape parameters and can now be identified as the functions $\frac{G_{I}}{G_{0}}$ and $\frac{G_{II}}{G_{0}}$ that represent a main part of the results of the fiber/matrix interface problem.

%------------------------------------------------------------------------------------------%
%                             	Critique and alternative formulations
%------------------------------------------------------------------------------------------%

\section{A new formulation of $G_{0}$ for a multi-scale analysis of transverse cracking}

%\clearpage
%------------------------------------------------------------------------------------------%
%                             CONCLUSIONS AND PERSPECTIVES
%------------------------------------------------------------------------------------------%

%\section{Conclusions}


%------------------------------------------------------------------------------------------%
%                                   ACKNOWLEDGEMENTS
%------------------------------------------------------------------------------------------%
%\ack
%Luca Di Stasio gratefully acknowledges the support of the European School of Materials (EUSMAT) through the DocMASE Doctoral Programme and the European Commission through the Erasmus Mundus Programme.
%\newpage
%------------------------------------------------------------------------------------------%
%                                      REFERENCES
%------------------------------------------------------------------------------------------%
\section*{References}
\begin{thebibliography}{9}
%\bibitem{author:year}author surname author initials (up to 10) year title {\it Journal} {\bf vol} (issue) pages
% FEM
\bibitem{Griffiths:1994}Griffiths  R. 1994 Stiffness matrix of the four-node quadrilateral element in closed form {\it Int. J. Numer. Meth. Eng.} {\bf 57} (2) 109--143
% LEFM
\bibitem{Krueger:2004}Krueger R. 2004 Virtual crack closure technique: History, approach, and applications {\it Appl. Mech. Rev.} {\bf 57} (2) 109--143
\bibitem{abaqus:2016} ABAQUS 2016 ABAQUS 2016 Analysis User's Manual {\it Online Documentation Help: Dassault Syst\'emes}
\bibitem{Rice:1968}Rice J. R. 1968 A Path Independent Integral and the Approximate Analysis of Strain Concentration by Notches and Cracks {\it J. Appl. Mech.} {\bf 35} 379--386

%\bibitem{author:year}
\end{thebibliography}

%------------------------------------------------------------------------------------------%
%------------------------------------------------------------------------------------------%
%                                    DOCUMENT ENDS
%------------------------------------------------------------------------------------------%
%------------------------------------------------------------------------------------------%
\end{document}

%------------------------------------------------------------------------------------------%
%------------------------------------------------------------------------------------------%
%------------------------------------------------------------------------------------------%
%                                      FILE ENDS
%------------------------------------------------------------------------------------------%
%------------------------------------------------------------------------------------------%
%------------------------------------------------------------------------------------------%
