\documentclass[review]{elsarticle}

\usepackage{amsmath}
\usepackage{lineno,hyperref}
\modulolinenumbers[5]

\journal{Report}

%%%%%%%%%%%%%%%%%%%%%%%
%% Elsevier bibliography styles
%%%%%%%%%%%%%%%%%%%%%%%
%% To change the style, put a % in front of the second line of the current style and
%% remove the % from the second line of the style you would like to use.
%%%%%%%%%%%%%%%%%%%%%%%

%% Numbered
%\bibliographystyle{model1-num-names}

%% Numbered without titles
%\bibliographystyle{model1a-num-names}

%% Harvard
%\bibliographystyle{model2-names.bst}\biboptions{authoryear}

%% Vancouver numbered
%\usepackage{numcompress}\bibliographystyle{model3-num-names}

%% Vancouver name/year
%\usepackage{numcompress}\bibliographystyle{model4-names}\biboptions{authoryear}

%% APA style
%\bibliographystyle{model5-names}\biboptions{authoryear}

%% AMA style
%\usepackage{numcompress}\bibliographystyle{model6-num-names}

%% `Elsevier LaTeX' style
\bibliographystyle{elsarticle-num}
%%%%%%%%%%%%%%%%%%%%%%%

\begin{document}

\begin{frontmatter}

\title{A set of criteria for the prediction of initiation and propagation of transverse cracks}
%\tnotetext[mytitlenote]{Fully documented templates are available in the elsarticle package on \href{http://www.ctan.org/tex-archive/macros/latex/contrib/elsarticle}{CTAN}.}

%% Group authors per affiliation:
%\author{Luca Di Stasio\fnref{myfootnote}}
%\address{Radarweg 29, Amsterdam}
%\fntext[myfootnote]{Since 1880.}

%% or include affiliations in footnotes:
\author[nancy,lulea]{Luca Di Stasio}
\author[lulea]{Janis Varna}
\author[nancy]{Zoubir Ayadi}
%\ead[url]{www.elsevier.com}

%\author[mysecondaryaddress]{Global Customer Service\corref{mycorrespondingauthor}}
%\cortext[mycorrespondingauthor]{Corresponding author}
%\ead{support@elsevier.com}

\address[nancy]{Universit\'e de Lorraine, EEIGM, IJL, 6 Rue Bastien Lepage, F-54010 Nancy, France}
\address[lulea]{Lule\aa\ University of Technology, University Campus, SE-97187 Lule\aa, Sweden}

\begin{abstract}

\end{abstract}

%\begin{keyword}
%\texttt{elsarticle.cls}\sep \LaTeX\sep Elsevier \sep template
%\MSC[2010] 00-01\sep  99-00
%\end{keyword}

\end{frontmatter}

\linenumbers

\section{Normalization function}

\begin{equation}
G_{0}=G_{0}\left(\varepsilon_{0},V_{f},E_{1f},E_{2f},E_{m},\nu_{12f},\nu_{23f},\nu_{m},G_{12f},G_{23f}\right)
\end{equation}

Given the elastic properties of the transversely isotropic UD ply $E_{1},E_{2},nu_{12},nu_{23}$, for a $90^{\circ}$ ply under transverse tension the cross section along the direction of the load coincides with the plane of transversal isotropy. It is thus possible, for a system in plane strain, to define equivalent isotropic Young's modulus and Poisson's ratio as follows. The effective Young's modulus and Poisson's ratio in plane strain in the plane of isotropy are defined as

\begin{equation}\label{eq:equiplanestraintransiso}
E^{*}=\frac{E_{2}}{1-\nu_{21}\nu_{12}}\qquad\nu^{*}=\frac{\nu_{23}+\nu_{21}\nu_{12}}{1+\nu_{23}}
\end{equation}

\begin{equation}
\begin{split}
G_{0}=\frac{\sigma_{0}^{2}}{E^{*}}\pi R_{f}\quad&\text{for a stress or force controlled test}\\[5pt]
G_{0}=E^{*}\varepsilon_{0}^{2}\pi R_{f}\quad&\text{for a strain or displacement controlled test}
\end{split}
\end{equation}

\section{Boundary conditions}

The ratio of maximum radial and tangential crack displacements with respect to the free case (single repeating element or single fiber layer ply?) can be considered as proxies for the effect of boundary conditions

\begin{equation}
\frac{u^{BC}_{r,max}}{u^{free}_{r,max}},\frac{u^{BC}_{\theta,max}}{u^{free}_{\theta,max}}
\end{equation}

\section{Initiation of fiber-matrix debonds}

\section{Propagation of fiber-matrix debonds}

\begin{equation}
\frac{G_{I}}{G_{0}}=\begin{cases}
A_{\delta}\left(V_{f}\right)\log\left(\delta\right)&+A_{\Delta\theta}\left(V_{f},\frac{u^{BC}_{r,max}}{u^{free}_{r,max}},\frac{u^{BC}_{\theta,max}}{u^{free}_{\theta,max}}\right)\sin\left(B_{\Delta\theta}\Delta\theta+C_{\Delta\theta}\right)+D\\
&B_{\Delta\theta}\Delta\theta_{max}\left(V_{f},\frac{u^{BC}_{r,max}}{u^{free}_{r,max}},\frac{u^{BC}_{\theta,max}}{u^{free}_{\theta,max}}\right)+C_{\Delta\theta}=\frac{\pi}{2}\\
&B_{\Delta\theta}\Delta\theta_{CZ}\left(\frac{u^{BC}_{r,max}}{u^{free}_{r,max}},\frac{u^{BC}_{\theta,max}}{u^{free}_{\theta,max}}\right)+C_{\Delta\theta}=\pi\\
\end{cases}
\end{equation}

\section{Transition to collective mesoscopic behavior}

\end{document}
